\beginsong {La complainte de l'homme en blanc} [
ititle= {Complainte de l'homme en blanc, (La)},
tu={Pour faire un homme (Hugues Aufray)}]

\beginverse
Si le Pap' s'inscrivait à l'université,
On pourrait parier sans risquer l'anathème
Qu'il choisirait de venir à l'ULB
Pour enfin y recevoir le baptême.
\endverse

\beginchorus
\textbf {Refrain}
\quater {C'est la complainte de l'homme en blanc !}
\endchorus

\beginverse
Il goûterait ainsi aux plaisirs de la chair,
Sans devoir pour cela à chaqu'messe y monter ;
Terminés les sermons, terminées les prières,
Pas b'soin d'tout ça pour nous faire un bébé.
\endverse

\beginverse
Le temps s'rait révolu de chanter des cantiques
Et d'inutilement secouer l'encensoir,
Il pourrait tout à l'aise fermer sa boutique
Pour avec nous guindailler tous les soirs.
\endverse

\beginverse
On l'imaginerait sur le char du CP,
Canonisant notre ami Théodore,
Il pourrait raconter à tout'la chrétienté
Qu'il a enfin découvert le folklore.
\endverse

\beginverse
Le touriste visitant la ville sacrée
Entendra la complainte du grand homme en blanc,
Pleurant ces occasions à tout jamais gâchées,
Derrièr'les tristes murs du Vatican.
\endverse

\endsong
