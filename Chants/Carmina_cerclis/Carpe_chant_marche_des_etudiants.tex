\beginsong{Marche des étudiants\footnotemark}[
  ititle={Marche des étudiants},
  tu={Les Gueux (P. : Paul Vanderborght, 1919)}]

\footnotetext{Ce titre était renseigné sous \emph{Chant de Étudiants} dans les Fleurs du Mâle-Geuzenliederboek (1967)}
  
\beginverse
Nous sommes ceux qu'anime la folie
Et qui s'en vont ivres de Liberté ;
Nous faisons guerr' à la mélancolie
Ou la cachons sous des cris de gaieté.
Bourgeois sans feu, votre vie est banale :
Les préjugés guident vos fronts tremblants ;
\bisdeux{Chez nous, l'on a l'humeur paradoxale}{Le cœur léger, et le gosier brûlant.}
\endverse

\beginverse
Des vieux gaulois nous gardons la mémoire
En les chantant perchés sur nos tonneaux ;
Si le bourgeois veut nous payer à boire,
Nous le suivrons jusqu'au fond des caveaux.
Fraternité, tu nais entre les verres ;
Ami, buvons à la Fraternité !
\bisdeux{Haro ! Haro sur les mines sévères !}{Pourquoi Bacchus n'est-il pas député ?}
\endverse

\beginverse
Si nous avons parfois la bourse plate,
Nous possédons bien des cœurs de trottins ;
Car, en amour, nous sommes des pirates
Braquant partout leurs regards assassins.
Souvent, pourtant, nous devons en rabattre
De nos grands airs de riche Don Juan :
\bisdeux{Dans les bouquins nous allons nous ébattre}{Pour oublier les suppôts de Satan.}
\endverse

\beginverse
Quand nous serons amis de doctes sages,
Nous sourirons doucement au passé
En regrettant, malgré tout, ce bel âge
D'enthousi-asme à jamais effacé.
Alors, tirant sur nos vieilles bouffardes,
Nous redirons à mi-voix nos chansons ;
\bisdeux{Elles étaient peut-être un peu gaillardes}{Mais on hurlait si bien à l'unisson !}
\endverse

\endsong
