\beginsong {Chant d'AGRO de l'ULB\footnotemark}[
ititle= {Chant d'AGRO de l'ULB},
ititle= {Dès que la bière coulera},
tu= {Dès que le vent soufflera (Renaud) (P: Touffe Decostre)}]

\footnotetext {Autre titre: Dès que la bière coulera.}

\beginverse
C'est pas l'homme qui prend la bièr',
C'est la bièr' qui prend l'homme.
Moi, la bière, elle m'a pris,
Je m' souviens, à l'Unif.
J'ai troqué mes cheveux
Et mon passé sérieux
Contr' un' penn' ULB
Et un vieux tablier.
J'ai déserté les crasses
Qui m' disaient : " Sois prudent,
La bière, c'est dégueulasse ;
Les comitards pissent dedans ! "
\endverse

\beginchorus
\textbf {Refrain}
Dès que la bière coulera, je reguindaill'ra.
Dès que les bières couleront, nous reguindaill'rons...
\endchorus

\beginverse
C'est pas l'homme qui prend la bière,
C'est la bière qui prend l'homme.
Moi, la bière, elle m'a pris
Au cercl' AGRO., tant pis...
J'ai eu si mal au coeur
Devant un fût tari,
Qu' j' suis parti avant l'heure,
N'était mêm' pas minuit.
J' me suis cogné partout,
J'ai dormi dans des draps souillés,
Ça m'a coûté des sous,
C'est la guindaille, c'est l' pied !
\endverse

\beginverse
C'est pas l'homme qui prend la bière,
C'est la bière qui prend l'homme.
Mais elle prend pas la femme
Qui préfère le champagne.
La mienne m'attend, à tort,
À la fin du T.D.
Mais l'amour est bien mort
Dans ses yeux délavés.
Elle n'a mêm' pas la cuite,
J' comprends pas elle pleure
Son homme qui la quitte ;
La bière, c'est son malheur !
\endverse

\beginverse
C'est pas l'homme qui prend la bière,\footnote {Ce couplet ne fait pas partie de la version originale, et est donc par là-même, apocryphe. Néanmoins,il est quand même chanté dans les cantus ; c'est là, la seule raison de sa présence dans ce chansonnier.}
C'est la bière qui prend l'homme.
Moi, la bière, elle m'a pris
Au cercl' AGRO., tant pis...
Je ferai le tour du monde,
Pour boire à chaque échoppe.
Dans tous les bars du monde,
Je sifflerai ma chope.
De Tokyo à Panam'(e),
Je foutrai le boxon
Jamais aucun barman
N'oubliera mon surnom.
\endverse

\beginverse
C'est pas l'homme qui prend la bière,
C'est la bière qui prend l'homme.
Moi, la bière, elle m'a pris
Je m' souviens à l'Unif.
Ne pleure plus ma mère
Ton fils est un poivrot,
Ne pleure plus mon père
Il vit sur son tonneau.
Regardez votr' enfant,
Il est rentré bourré,
Je sais, c'est pas marrant
Mais il a guindaillé.
\endverse

\endsong
