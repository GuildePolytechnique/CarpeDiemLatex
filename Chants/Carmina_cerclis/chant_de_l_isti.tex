\beginsong {Chant de l'I.S.T.I. (Bxl)}[
ititle= {Chant de l'ISTI},
tu= {Hymne à la joie (Ludwig von Beethoven)}]

\beginverse
Si, partout, on nous envie
Pour nos femmes et nos bons vits,
Si nos fûts sont toujours vides :
C'est que nous sommes de l'ISTI.
\endverse

\beginchorus
\textbf {Refrain}
Pennes, femmes, et bonnes bières,
C'est ce qui compte dans la vie.
Sur les calotins, on chie ;
Ce sont tous des petits zizis.
\endchorus

\beginverse
Traducteurs et interprètes
Se retrouvent au café.
Dans la joie et dans les dettes,
À l'Antique, on peut s' saouler.
\endverse

\beginverse
Et dans toutes les guindailles,
Ils nous entendront clamer
Les vertus et les ripailles
De ceux qui les font chi-er.
\endverse

\beginverse
Si un soir, dans les rues sombres,
Vous entendez ce chant-ci,
Vite mettez-vous dans l'ombre,
Ce sont les gens de l'ISTI.
\endverse

\endsong
