\beginsong {Chant de Solvay ULB (C.\$)} [
ititle= {Chant de Solvay ULB},
tu= {Les housards de la garde}]

\beginverse
C'est durant toutes nos folles ivresses
Que nous nous livrons à bien des méfaits,
Car nous voulons dissiper la tristesse
De l'avenir que la vie nous promet.
\endverse

\beginchorus
\textbf {Refrain}
Verre à la main, chantons notre jeunesse,
Ecout' bourgeois qui nous prend pour des fous :
C'est à Solvay qu'on fête la Vadrouille
Jusques à l'aub' nous buvons comm' des trous.
\endchorus

\beginverse
Nous adorons nos charmantes amies
Et restons près d'elles jusqu'au matin
Mais, malgré tout cet amour qui nous lie,
Nous ne laiss'rons pas tomber les copains.
\endverse

\beginverse
Et si parfois des esprits par trop sages
Disaient : " Bientôt vous le regretterez,
Vous abusez trop de votre jeune âge,
Ce n'est pas ainsi qu'il faut s'amuser. "
\endverse

\beginchorus
\textbf {Dernier refrain}
Verre à la main, nous leur rétorquerons :
" C'est à Solvay qu'on fête les orgies.
Ne craignant pas la suit' de nos folies,
Il nous faut la femm', la bière, la chanson.
Verre à la main, nous passons par la vie,
Verre à la main gai'ment nous la quitt'rons. "
\endchorus

\endsong
