\beginsong {La ballade des estomacs tourmentés \footnotemark} [
tu= {La ballade des gens heureux (Gérard Lenorman)},
ititle= {Ballade des estomacs tourmentés, la}]

\footnotetext {Gerbir or not gerbir ; Festival de la chanson estudiantine CP ULB, 1988.}

\beginverse
Si votre estomac se trouve ballotté
Si la veille vous avez trop guindaillé
Acceptez donc la dégueulade
La dégueulade peut soulager.
\endverse

\beginverse
Les gros morceaux à l'entrée du cardia.
Se bouscul'nt pour sortir d' l'estomac
De l'oesophage l'escalade
En dégueulade se termin'ra
\endverse

\beginverse
Tiens dis'nt les frites, rev'là les amygdales
Et la dent creus', bientôt ce s'ra l' final
Allons vit' sortir en promenade
La dégueulade c'est carnaval
\endverse

\beginverse
Les spaghettis ressortent par le nez
Et en pluie retomb' sur le pavé
Avouez que la dégueulade
De bell's cascades peut nous donner
\endverse

\beginverse
Roter, peter, chier ou bien vomir
Tout' éjection provoque du plaisir
Mais tout en tête du hit-parade
La dégueulade me guérit
\endverse

\beginverse
Vous est-il seul'ment déjà arrivé
De dégueuler sur votre dulcinée
Pour les coeurs qui batt'nt la chamade
La dégueulade c'est pas le pied
\endverse

\beginverse
Et quand on a bien dégueulé partout
Dedans on peut alors fair' des remous
On y ferait nager des naillades
La dégueulade tell'ment c'est doux
\endverse

\beginverse
Et si la nourritur' est bien mâchée
L'aspect en lisse et bien régulier
On mangerait bien de cett' panade
La dégueulade c'est bon c'est gai
\endverse

\beginverse
Et pour ceux qui ont horreur des crachats
Ou qui sent'nt leur estomac raplapla
Guindaillez à la limonade
La dégueulade vous épargnera.
\endverse

\endsong