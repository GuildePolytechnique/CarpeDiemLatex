\beginsong{La ballade du mutant\footnotemark}[
  ititle={Ballade du mutant, La},
  tu={Malheur à celui qui blesse un enfant (Enrico Macias)}]

\footnotetext{Corporatio Bruxellensis, ULB ; Festival de la chanson estudiantine CP ULB, 1981.}
  
\beginverse
Il est né un soir près d'un' central' nucléaire
D'un pèr' alcoolique et d'un' mèr' éthéromane
Il avait trois jambes, de longs bras tous ve-erts
Son grand nez tout jaun' luisait comm' un' banane
\endverse

\beginchorus
\textbf{Refrain}
Qu'il soit vert ou bleu depuis sa naissance
Il a les yeux roug's, il est plein d'excroissances
Qu'il soit asthmatique, goitreux ou rampant
Malheur à celui qui blesse un mutant.
\endchorus

\beginverse
Dans l'institution où l'on plaça le p'tit chauve
Il faisait bien rir' avec sa douzain' de doigts
Il faut reconnaître qu'une main tout' mauve
Ça n'est pas courant sur la têt' d'un p'tit gars.
\endverse

\beginverse
Il y'avait des jours où c'était dur pour l' pauvr' gosse
Quand avec un' sonde il fallait l'alimenter
Car je n' vous l'ai pas dit, mais en plus d' sa bosse
Le pauvre chéri était paralysé.
\endverse

\beginverse
Et quand il eut l'âge enfin d'aller voir les filles1
Qu'il voulut sortir sa queue en form' d' tir'-bouchon
Sa petit' peau flasqu' é-tait moll' et sans vie
Et sa couille uniqu' avait l'air d'un ballon.
\endverse

\endsong
