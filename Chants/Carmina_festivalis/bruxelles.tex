\beginsong {Bruxelles \footnotemark} [
ititle= {Bruxelles}]

\footnotetext { PK, VUB ; Festival de la chanson estudiantine CP ULB, 1984. Auteurs de la partie néerlandophone :
W. Heynen et Wannes Van De Velde pour l'originale \emph{Ik wil deze nacht in de straten verdwalen}. "Het
beste van Wannes Van De Velde" - 1989 kompilatie Polygram Brussel - Compact Disc AAD Philips 838
762-2.}

\beginchorus
\textbf {Refrain I}
Je veux me prom'ner dans les rues de Bruxelles,
Les bruits de cette ville me rendent amoureux,
Venez voir comm' toutes les putes sont belles,
Vous y trouverez un accueil chaleureux.
\endchorus

\beginverse
Sous la lumière des grands réverbères
On voit un couple s'aimer tendrement
Dans une autre ruelle, une scène cruelle,
Deux sales mecs, au poing, se rentrent dedans.
\endverse

\beginverse
Les étudiants sont en train de guindailler
Dans les bistrots, dans les cafés,
Et dehors, dans le froid, un clochard solitaire
Cherche une place pour dormir par terre.
\endverse

\beginchorus
\textbf {Refrain II}
Ik wil deze nacht in de straten verdwalen,
De klank van de stad maakt mijn ziel amoureus
Al heb ik geen geld om plezier te betalen,
Ik vind wel een vrouwke naar mijne keus.
\endchorus

\beginverse
Onder de glans van de manestralen,
Wordt heel onze wereld een huwelijksbed,
Ga mee naar de kroegen vol wijnen en matrozen
Vergeet uwe na-am en al de rest.
\endverse

\beginverse
Laat ons dan samen de wereld verteren,
Met klinkede glazen vol franse wijn,
Zingt mee met de mensen, dat hebben ze geren,
En laat deze nacht nooit een einde zijn.
\endverse

\endsong