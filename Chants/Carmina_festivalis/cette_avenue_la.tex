\beginsong {Cette avenue-là\footnotemark} [
ititle={Cette avenue-là},
tu= {Cette année-là (interprétée par Claude François)}]

\footnotetext {Les nanas de Léonard et les clodos, ULB ; Festival de la chanson estudiantine CP ULB, 1988. Non
chantée par annulation montoise du festival [cfr Le bétail montois (Guilde Polytechnique 1989)].}

\beginverse
Cett' av'nue-là (cett' av'nue-là)
Je me souviens de la première fois
J' la descendais, je n' la connaissais pas
Oh ! Quelle av'nue cette av'nue-là (cett' av'nue-là)
Je n' sais pourquoi (je n' sais pourquoi)
Par des étudiants je fus abordé
Et de sale bleu c'est moi qu'ils ont traités
Je ne comprenais pas pourquoi (non pas pourquoi)
\endverse

\beginchorus
C'est là (là)
Que je subis mon premier luigi ... en public
Et là (là)
J'ai compris ce que c'était un scar.
\endchorus

\beginverse
Cett' av'nue-là (cett' av'nue-là)
Bord' un endroit que vous n'ignorez pas
Le foyer vous n'y échappez pas
Quel abreuvoir cett' endroit-là (cett' endroit-là)
Mes années là (mes années là)
J'en suis sorti assez souvent bourré
Kriek, brun', ou blanche, rien n'avait de secret
Oh ! Qu'est-ce que j'y ai guindaillé (ai guindaillé)
\endverse

\beginchorus
De là (là)
Je me traînais jusqu'à tous les TD ... enivré
J' voulais (ouais)
Que la nuit n'en finisse pas !
\endchorus

\beginverse
Cett' av'nue-là (cett' av'nue-là)
Menait tout droit au kot(e) des bleuettes
Et tous les soirs je leur faisais leur fête
Oh ! Quel foutoir cet endroit-là (cet endroit-là)
Cette av'nue-là (cett' av'nue-là)
Oh ! Ça jamais je n' pourrais l'oublier
Car ma jeunesse c'est elle qui l'a marquée
Et dans mon coeur elle est gravée (elle est gravée)
\endverse

\beginchorus
C'est là (là)
Qu'à chaque St-Vé on brûlait tous les chars dans le noirs
Et nous (nous)
Les students on n' demandait qu'à boire !
\endchorus

\beginverse
Cette av'nue-là (cett' av'nue-là)
Il n'y en a qu'une elle se trouve à l'ULB
Sortant d'ici vous la reconnaîtrez
Sans aucun doutes ... c'est Paul Héger
\endverse

\endsong