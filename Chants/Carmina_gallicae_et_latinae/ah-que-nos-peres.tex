\beginsong {Ah! Que nos pères étaient heureux\footnotemark}[
ititle= {Ah! Que nos pères étaient heureux}]

\footnotetext{Origine : Haute Bourgogne.}

\beginverse
\bis {Ah! Que nos pèr's étaient heureux}
Quand ils étaient à table,
\bis {Le vin coulait à côté d'eux}
Ça leur était fort agréable
\endverse

\beginchorus
\textbf {Refrain}
Et ils buvaient à leurs tonneaux 
\bis {Comme des trous.}
\bis{Morbleu ! Bien autrement que nous !}
\endchorus

\beginverse
\bis {Ils n'avaient ni riches buffets}
Ni verres de Venise,
\bis {Mais ils avaient des gobelets}
Aussi grands que leur barbe grise.
\endverse

\beginverse
\bis {Ils ne savaient ni le latin}
Ni la théosophie
\bis {Mais ils avaient le goût du vin}
C'était là leur philosophie
\endverse

\beginverse
\bis {Quand ils avaient quelque chagrin}
Ou quelque maladie,
\bis {Ils plantaient là le médecin}
L'apothicair', sa pharmacie.
\endverse

\beginverse
\bis {Et quand le petit dieu d'Amour}
Leur envoyait quelque donzelle
\bis {Sans peur, sans feinte et sans détour}
Ils plantaient là la demoiselle
\endverse

\beginverse
\bis {Celui qui planta le provin}
Au beau pays de France
\bis {Dans le flot du rubis divin}
Sut planter là notre espérance.
\endverse

\beginchorus
\textbf {Dernier refrain}
Amis buvons à nos tonneaux 
\bis {Comme des trous.}
\bis {Morbleu ! L'avenir est à nous !}
\endchorus

\endsong