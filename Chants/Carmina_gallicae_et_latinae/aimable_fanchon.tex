\beginsong {L'aimable Fanchon\footnotemark} [
tu= {Amour, laisse gronder ta mère (XVIIème sicècle)},
ititle= {L'aimable Fanchon},
ititle= {Fanchon}]

\footnotetext {Autre titre: Fanchon. C'est une chanson de garnison, attribuée à Antoine Charles Louis, comte de Lasalle, qui l'aurait composée au soir de la bataille de Marengo (1800). Cette chanson est devenue chanson à boire par la transformation du parrain Allemand en parrain Bourguignon, et par l'omission du dernier couplet. L'"Ordre du 101" a repris cette chanson comme chant d'ordre.}

\beginverse
Amis, il faut faire une pau-ause,
J'aperçois l'ombre d'un bouchon,\footnote {Nom populaire du cabaret.}
Buvons à l'aimable Fanchon,
Chantons pour elle quelque cho-ose.
\endverse

\beginchorus
\textbf {Refrain}
Ah! que son entretien est dous,
Qu'elle a de mérit' et de gloire.
\terdeux{Elle aime à rir', elle aime à boire,} {Elle aime à chanter comme nous.}
\bis {Oui, comme nous.}
\endchorus

\beginverse
Fanchon, quoique bonne chrétie-enne,
Fut baptisée avec du vin.
Un Bour-guignon fut son parrain,
Une Bretonne sa marrai-aine.
\endverse

\beginverse
Fanchon préfère la grilla-ade
A d'autres mets plus délicats.
Son teint pren un nouvel éclat
Quand on lui sert une rasa-ade.
\endverse

\beginverse
Fanchon ne se montre crue-elle
Que quand on lui parle d'amour.
Mais, moi, si je lui fais la cour,
C'est pour m'enivrer avec e-elle.
\endverse

\beginverse
Un jour, le voisin La Grena-ade
Lui mit la main dans le corset;
Elle ré-pondit par un soufflet
Sur le museau du camara-ade.
\endverse

\endsong