\beginsong {Alexandre\footnotemark}[
ititle= {Alexandre}]

\footnotetext {Air à boire du XVème siècle. Une version plus correcte de cette chanson est en cours de recherche. Les vers 7 et 8 de chaque couplet sont notés selon la version de la chorale de l'ULB.}

\beginverse
Alexandre, dont le nom
A rempli la terre,
N'aimait pas tant le canon
Qu'il faisait le verre.
Si le grand Mars des guerriers
S'est acquis tant des lauriers,
Que devons, -vons, -vons,
Que pouvons, -vons, -vons,
Que devos,
Que pouvons
Que devons-nous faire
Sinon de bien boère?
\endverse

\beginverse
Quand la mer rouge apparût
Aux yeux de Grégoire,
Aussitôt ce buveur crut
Qu'il n'avait qu'à boire.
Moïse fut bien plus fin
Voyant que ce n'était vin;
Il la pa-, pa-, pa-,
Il la -sa, -sa, -sa,
Il la pa-,
Il la -sa,
Il la passa toute,
Sans en boire goutte.
\endverse

\beginverse
Le bonhomme Gédéon
Faisait des merveilles,
Aussi n'usait sédition
Rien que des bouteilles.
Servons-nous donc, aujourd'hui,
Des bouteilles comme lui
\bis {Et faisons, -sons, -sons,}
\bis {Et faisons}
Et faisons la guerre
A grands coups de verre.
\endverse

\beginverse
Loth, qui fut homme de bien,
Se plaisait à boère,
Dieu ne lui en disait rien,
Il le laissait faire.
Et puis quand il était saoûl,
Il s'endormait comme nous,
\bis {Dans un' ca-, ca-, ca-}
\bis {Dans un' ca-}
Dans une caverne
Près de la taverne
\endverse

\beginverse
Noé, pendant qu'il vivait,
Patriarche digne,
Savait bien comm' on buvait
Du fruit de la vigne;
De peur qu'il ne but de l'eau
Dieu lui fit faire un bateau
Pour trouver, -ver, -ver,
Pour chercher, -cher, -cher,
Pour trouver,
Pour chercher,
Pour trouver refuge,
Au temps du déluge.
\endverse

\endsong