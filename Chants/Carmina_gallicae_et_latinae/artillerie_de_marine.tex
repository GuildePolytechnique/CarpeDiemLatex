\beginsong{L'artillerie de marine\footnotemark}[
  ititle={Artillerie de marine, L'},
  ititle={Trou de mon cul, Le}]

\footnotetext{Autre titre : \emph{Le trou de mon cul}. Les français servent \emph{Le jour de l'An} en guise d'introït à cette chanson.}

\beginverse
Tous les obus de la marine
Sont si bien faits et si pointus
Qu'ils entreraient sans vaseline
\bis{Dans l' trou d' mon cul}
\endverse

\beginchorus
\textbf{Refrain}
L'artill'rie d' marine, voilà mes amours
Et je l'aimerai, je l'aimerai sans cesse
L'artill'rie d' marine, voilà mes amours
Et je l'aimerai, je l'aimerai toujours.
\endchorus

\beginverse
L' adjudant-chef qu' est de service
A une sale gueul' si mal foutue
Qu'on la prendrait sans plus d' malice
\bis{Pour l' trou d' mon cul}
\endverse

\beginverse
J'ai fait trois ans de gymnastique
Et non jamais, j' n'ai jamais pu,
Poser un baiser sympathique
\bis{Sur l' trou d' mon cul}
\endverse

\beginverse
A mon dernier voyage en Chine
Un mandarin gras et dodu
Voulut mettre sa grosse pine
\bis{Dans l' trou d' mon cul}
\endverse

\beginverse
J'ai fait trois fois le tour du monde
Dans mes voyages, j' n'ai jamais vu
Une chose aussi parfait'ment ronde
\bis{Que l' trou d' mon cul}
\endverse

\beginverse
De Singapour jusqu'à Formose
J' n'ai jamais vu, non jamais vu,
J' n'ai jamais vu chose aussi rose
\bis{Que l' trou d' mon cul}
\endverse

\beginverse
J'ai visité des capitales,
Et non jamais, j' n'ai jamais vu,
Un' chose aussi parfait'ment sale
\bis{Que l' trou d' mon cul}
\endverse

\beginverse
Si j' suis entré dans la méd'cine
C'est qu' les clystères sont si pointus,
Qu'ils entreraient comme une pine
\bis{Dans l' trou d' mon cul}
\endverse

\beginverse
Si j' suis entré dans l'art dentaire
C'est qu' les tire-nerfs sont si menus
Qu' j' m'en mettrais une bonne douzaine
\bis{Dans l' trou d' mon cul}
\endverse

\beginverse
Quand j' serai un vieux qu' a la tremblote
Et que d' baiser, je n' pourrai plus,
J'irai chez Jeanne ou chez Charlotte
M' fair' fair' des langues
Dans l' trou d' mon cul.
\endverse

\endsong
