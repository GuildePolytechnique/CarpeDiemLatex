\beginsong {L'Artilleur de Metz\footnotemark}[
ititle={ Artilleur de Metz, L'}]

\footnotetext {\emph {Pourrait dater de la restauration} (04/1815 - 07/1830) \emph {ou le refrain pourrait être inspiré du duo de basses du deuxième acte de la pièce d'opera} \textbf { I puritani} \emph{de} \textbf { Bellini, Suoni la tromba}}

\beginverse
Quand l'artilleur de Metz
Arriv' en garnison,
Toutes les femm's de Metz
Se fout'nt les doigts dans l' con
Pour préparer l' chemin
A l'artilleur rupin
Qui leur foutra demain
Sa pin' dans le vagin
\endverse

\beginchorus
\textbf {Refrain}
Artilleurs, mes chers frères,
A sa santé buvons un verre
\bisdeux {Et répétons ce gai refrain :}{Viv'nt les artilleurs, les femm's et le bon vin !}
\endchorus

\beginverse
Quand l'artilleur de Metz
Demand' une faveur,
Toutes les femm's de Metz
L'accord'nt avec ardeur
Et le mari cornard
Voit l'artilleur chicard
Baiser également
La fill' et la maman.
\endverse

\beginverse
Quand l'artilleur de Metz
Quitte sa garnison
Toutes les femm's de Metz
Se mett'nt à leur balcon
Pour saluer l' départ
De l'artilleur chicard
Qui leur a tant foutu
Sa pin' dans l' trou du cul
\endverse

\endsong