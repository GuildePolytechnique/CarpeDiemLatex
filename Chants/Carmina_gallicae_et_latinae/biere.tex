\beginsong{La bière\footnotemark}[
  ititle={Bière, La}]

\footnotetext{Auteur : Antoine Clesse (forgeron-poète montois).}
  
\beginverse
Elle a vraiment d'une bière flamande
L'air avenant, l'éclat et la douceur.
Joyeux Wallons, elle nous affriande
Et le Faro trouv' en elle une soeur.
\endverse

\beginchorus
\textbf{Refrain}
À plein verre, mes bons amis,
En la buvant, il faut chanter la bière.
À plein verre, mes bons amis,
Il faut chanter la bière du pays.
\endchorus

\beginverse
Voyez là-bas la kermesse en délire :
Les pots sont pleins, jouez ménétriers !
Quels jeux bruyants et quels éclats de rire !
Ce sont encor' "Les Flamands" de Teniers.
\endverse

\beginverse
Aux souverains, portant tout haut leurs plaintes,
Bourgeois jaloux des droits de la cité,
Nos francs aïeux, tout en vidant leur pinte,
Fondaient les arts avec la liberté.
\endverse

\beginverse
Quand leurs tribuns, à l'attitud' altière,
Faisaient sonner le tocsin des beffrois,
Tous ces fumeurs, tous ces buveurs de bière,
Savaient combattre et mourir pour leurs droits.
\endverse

\beginverse
Belges, chantons à ce refrain à boire !
Peintres, guerriers qui nous illustrent tous,
Géants couchés dans leur linceul de gloire,
Vont s'éveiller, pour redir' avec nous.
\endverse

\beginverse
Salut à toi, bière limpid' et blonde !
Je tiens mon verre, et le bonheur en main.
Ah ! J'en voudrais verser à tout le monde,
Pour le bonheur de tout le genre humain.
\endverse

\endsong
