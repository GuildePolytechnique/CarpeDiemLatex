\beginsong{Les biroutes}[
  ititle={Biroutes, Les}]
  
\beginverse
\bis{In djou qué dj' n'avou rin à fai}
\bis{D' j'ai composé pou' m'n amus'min}
Avu m' gross' biroute en main
En' bell' canson su les biroutes.
\textit{Parlé :} Petit ballet, coquet, discret
\endverse

\beginchorus
\textbf{Refrain}
Dansez, voltigez, les biroutes,
Traderidera ha, ha, traderidera
Ah ! Qué plaisi' d'avou en' gross' biroute !
Ah ! Qué plaisi' d' pouvou s'in servi' eyè sin capote !
\endchorus

\beginverse
\bis{En' société vint dè s' former}
\bis{On y admet tous les d' jon' gins}
Dè dix-huit à septante sept ans
Pourvu qu'i's eussent en' gross' biroute.
\textit{Parlé :} Petit ballet, coquet, secret
\endverse

\beginverse
\bis{Quin l' société sèra prospère}
\bis{Nos akat'rons in biau drapiau}
Avu en' gross' biroute in waut
Eyè l' monde dira : "Què bell' biroute."
\textit{Parlé :} Petit ballet, coquet, matrimonial
\endverse

\beginverse
\bis{Quin l' présidin i' s' marira}
\bis{Nos s'rons tertout à s' mariatche}
Avu en' gross' boit' dè ciratche
Eyè nos noircirons s' biroute.
\textit{Parlé :} Petit ballet, coquet, funèbre
\endverse

\beginverse
\bis{Quin l' présidin i' s' morira}
\bis{Nos s'rons tertout à s' n'intermin}
Avu nos gross' biroutes in main
Eyè nos f'rons braire nos biroutes.
\textit{Parlé :} Petit ballet, coquet, patriotique
\endverse

\beginverse
\bis{Quin les Flamins nos attaqu'rons}
\bis{Nos s'rons tertou d'vé l' frontière}
Avu nos gross' biroutes in l'air
Nos les maqu'rons à coups d' biroutes.
\endverse

\endsong
