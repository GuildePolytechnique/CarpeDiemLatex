\beginsong {La bourguignonne\footnotemark}[
ititle={Bourguignonne, La},
ititle={Joyeux enfants de la Bourgogne}]

\footnotetext {Autre titre : \emph{Joyeux enfants de la Bourgogne}. A remarquer que le refrain actuel est peu différent de l'original qui, lui, se trouve dans le "Petit Bitu" (1993).}

\beginverse
C'est dans une vigne
Que j'ai vu le jour ;
Ma mère était digne
De tout mon amour ;
Depuis ma naissance
Elle m'a nourri ;
En reconnaissance
Mon coeur la chérit.
\endverse

\beginchorus 
\textbf {Refrain}
\bisquatre {Joyeux enfants de la Bourgogne} {Je n'ai jamais eu de guignon ;} {Quand je vois rougir ma trogne } {Je suis fier d'être Bourguignon !}
\endchorus

\beginverse
Toujours ma bouteille
À côté de moi,
Buvant sous la treille,
Plus heureux qu'un roi,
Jamais je n' m'embrouille
Car chaque matin
Je me débarbouille
Dans un verr' de vin.
\endverse

\beginverse
Madère et champagne,
Approchez un peu,
Et vous, vins d'Espagne
Malgré tous vos feux,
Amis de l'ivrogne
Réclamez vos droits
Devant la Bourgogne :
Saluez trois fois !
\endverse

\beginverse
Ma femm' est aimable
Et sur ses appas
Quand je sors de table
Je ne m'endors pas
Je lui dis : " Mignonne,
Je plains ton destin. "
Mais ma bourguignonne
Jamais ne s'en plaint.
\endverse

\beginverse
Je veux qu'on enterre,
Quand je serai mort,
Près de moi un verre
Empli jusqu'au bord.
J' veux êtr' dans ma cave
Tout près de mon vin
Dans un' pose grave
Le nez sous l' robin.
\endverse

\endsong