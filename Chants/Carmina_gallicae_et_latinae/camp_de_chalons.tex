\beginsong {Le camp de Châlons\footnotemark} [
ititle= {Camp de Châlons, Le},
ititle= {En revenant de Charenton},
ititle= {Marie-Suzon}]

footnotetext {Autres titres : \emph{En revenant de Charenton}; la chanson commence alors par ce titre (in 69 Chansons d'Étudiants, 1984), \emph{Marie-Suzon}. Allusion est faite au camp militaire de Châlons (1859), dans la Champagne, ce qui pourrait nous la faire dater de la seconde moitié du XIXème siècle.}

\beginverse
En revenant du camp d' Châlons
La faridondaine, la faridondon\footnote { Variante :" \emph{Bringuedezingue, bringuedezon}" ou "\emph{Bringuedezingue, la faridondaine}"}
J'ai rencontré Marie-Suzon.
\endverse

\beginchorus
\textbf {Refrain}
Tortille, broquille marchand de guenilles
À cheval sur la fille, enculant la famille
Le père, la mère, la vieill', et le vieux !
Vinaigr' et moutard' et chapeau de cocu,
Prends ton nez, ta barb' et fous ça dans mon cul
Tap' ton cul contre le mien,
Va t' fair' foutre, moi j'en reviens
Où ça ?
Par derrièr' la maison.
\bis {Et allons en vendange, les raisons sont bons}
Et fous ton nez dans le trou de mon
Bringu'dezingue, la faridondaine
Bringu'dezingue, la faridondon.
\endchorus

\beginverse
J'ai rencontré Marie-Suzon
La faridondaine, la faridondon
J' la fis asseoir sur le gazon.
\endverse

\beginverse
... En m'asseyant, je vis son con.
\endverse

\beginverse
... Il était noir comm' du charbon.
\endverse

\beginverse
... Et tout couvert de morpi-ons.
\endverse

\beginverse
... Il y'en avait cinq cent millions.
\endverse

\beginverse
... Qui défilaient par escadrons.
\endverse

\beginverse
... Comm' les soldats d' Napoléon.
\endverse

\beginverse
... Et moi, comm' un foutu cochon.
\endverse

\beginverse
... J'ai baisé la Marie-Suzon.
\endverse

\endsong