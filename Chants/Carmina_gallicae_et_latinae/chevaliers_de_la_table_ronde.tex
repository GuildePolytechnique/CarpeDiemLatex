\beginsong {Chevaliers de la table ronde}[
ititle= {Chevaliers de la table ronde}]

\beginverse
\bisdeux {Chevaliers de la table ronde,} {Goûtons voir si le vin est bon.}
\bistrois {Goûtons voir, oui, oui, oui,} {Goûtons voir, non, non, non,} {Goûtons voir si le vin est bon}
\endverse

\beginverse
\bisdeux {J'en boirai cinq à six bouteilles} {Une femme sur les genoux,}
Une femme, oui, oui, oui, ...
\endverse

\beginverse
\bisdeux {Et si le tonneau se débonde\footnote {Variante : S'il est bon, s'il est agréable}} {J'en boirai jusqu'à mon plaisir}
J'en boirai, oui, oui, oui, ...
\endverse

\beginverse
\bisdeux {Et s'il en reste quelques gouttes} {Ce sera pour nous rafraîchir}
Ce sera, oui, oui, oui, ...
\endverse

\beginverse
\bisdeux {Mais voici qu'on frapp' à la porte} {Je crois bien que c'est le mari,}
Je crois bien, oui, oui, oui, ...
\endverse

\beginverse
\bisdeux {Si c'est lui, que le diable l'emporte} {Car il vient troubler mon plaisir,}
Car il vient, oui, oui, oui, ...
\endverse

\beginverse
\bisdeux {Si je meurs, je veux qu'on m'enterre} {Dans une cave où y'a du bon vin,}
Dans une cave, oui, oui, oui, ...
\endverse

\beginverse
\bisdeux {Les deux pieds contre la muraille} {Et la têt' sous le robinet}
Et la têt', oui, oui, oui, ...
\endverse

\beginverse
\bisdeux {Et mes os de cette manière} {Resteront, imbibés de vin}
Resteront, oui, oui, oui, ...
\endverse

\beginverse
\bisdeux {Et les quatre plus grands ivrognes} {Porteront les quatr' coins du drap}
Porteront, oui, oui, oui, ...
\endverse

\beginverse
\bisdeux {Pour donner le discours d'usage,} {On prendra le bistrot du coin.}
On prendra, oui, oui, oui, ...
\endverse

\beginverse
\bisdeux {Sur ma tomb', je veux qu'on inscrive :} {"Ici-gît le Roi des buveurs."}
Ici gît, oui, oui, oui, ...
\endverse

\beginverse
\bisdeux {La morale de cett' histoire} {Est qu'il faut boir' avant d' mourir}
Est qu'il faut, oui, oui, oui, ...
\endverse

\endsong