\beginsong {Le con et la bouteille\footnotemark} [
ititle= {Con et la bouteille (La)},
tu= {Les coquilles}]

\footnotetext {Connue depuis le XVIIème siècle, une version de cette chanson figure dans "Le Panier aux Ordures" (1878).}

\beginverse
Nargue des pédants et des sots
Qui viennent chagriner notr' âme !
Que fit Dieu pour guérir nos maux ?
Les vieux vins et les jeunes femmes.
Il créa pour notre bonheur
Le sexe et le jus de la treille
Aussi je vais en son honneur
\bis {Chanter les cons et les bouteilles !}
\endverse

\beginverse
Dans l'Olympe, séjour des dieux
On boit, on patine des fesses,
Et le nectar délici-eux
N'est que le foutre des déesses.
Si j'y vais, jamais Apollon
Ne charmera plus mon oreille ;
De Vénus, je saisis le con,
\bis {De Bacchus, je prends la bouteille !}
\endverse

\beginverse
Dans les bassinets féminins
Quand on a trop brûlé d'amorces,
Quelques bouteilles de vieux vins
Au vit rendent toute sa force.
Amis, plus on boit, plus on fout ;
Un buveur décharge à merveille
Aussi le vin pour dire tout
\bis {C'est du foutre mis en bouteille.}
\endverse

\beginverse
On ne peut pas toujours bander ;
Du vit, le temps borne l'usage.
On se fatigue à décharger
Mais, amis, on boit à tout âge.
Quant aux vieillards, aux froids couillons,
Qu'ils utilisent mieux leurs veilles ;
Quand on n' peut plus boucher de cons,
\bis {On débouche au moins des bouteilles !}
\endverse

\beginverse
Mais, hélas ! Depuis bien longtemps,
Pour punir nos fautes maudites,
Le Bon Dieu fit les cons trop grands
Et les bouteilles trop petites.
Grand Dieu, fais, nous t'en supplions,
Par quelque nouvelle merveille,
Toujours trouver le fond du con
\bis {Jamais celui de la bouteille !}
\endverse

\endsong
