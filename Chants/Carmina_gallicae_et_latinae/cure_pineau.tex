\beginsong {Le curé Pineau\footnotemark} [
ititle= {Curé Pineau (le)},
ititle= {Curé Pinot (le)}]

\footnotetext {Le curé Pinot. L'air original est assez différent, une fois n'est pas coutume, de celui chanté
actuellement dans les réunions d'étudiants. N.B. : Les couplets en italique ne sont pas chantés en Belgique.}

\beginverse
Je m'en vais vous conter l'histoire
De Pineau curé d' chez nous,
\bistrois {Pineau cu-, papa,} {Pineau cu-, maman,} {Pineau curé de chez nous.}
\endverse

\beginverse
Monsieur l' curé a un parterre\footnote { Variante : \emph{des platt's-bandes}}
Il en cultive les fleurs,
\bistrois {Il en cul-, papa,} {Il en cul-, maman,} {Il en cultive des fleurs.}
\endverse

\beginverse
Monsieur l' curé a des calottes
Des calottes de drap noir, ...
\endverse

\beginverse
Monsieur l' curé a un' fontaine
Au bord d'elle, il vient s'asseoir, ...
\endverse

\beginverse
Monsieur l' curé, il mont' en chaire
Son gros vicaire le suit, ...
\endverse

\beginverse
Monsieur l' curé a un carrosse
Ses roues pèt'nt sur le pavé, ...
\endverse

\beginverse
Monsieur l' curé dit au vicaire\footnote {Originale : \emph{Monsieur l' curé qu' aime la nature Dit : " Sortons observer l' couchant. "}}
Sortons observer l' couchant, ...
\endverse

\beginverse
\textit{Monsieur l' curé a une vieill' cloche}
\textit{Il la branl' trois fois par jour, ...}
\endverse

\beginverse
\textit{M'sieur l' curé a un enfant d' chœur(e)}
\textit{C'est un compagnon de Jésus, ...}
\endverse

\beginverse
\textit{Monsieur l' curé a une chasuble,}
\textit{Il l'enfile tous les matins, ...}
\endverse

\beginverse
\textit{Monsieur l' curé fait l'élevage}
\textit{Des lapines et des lapins, ...}
\endverse

\beginverse
\textit{Monsieur l' curé aime les Anglaises}
\textit{Pour leurs singularités, ...}
\endverse

\beginverse
\textit{Monsieur l' curé aime les Russes}
\textit{Pour leur kummel délicieux, ...}
\endverse

\beginverse
\textit{Celui qui fit cette chanson -on}
\textit{C'est Pineau, curé d' chez nous, ...}
\endverse

\endsong