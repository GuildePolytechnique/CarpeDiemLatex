\beginsong {La tour de Londres\footnotemark} [
ititle= {Tour de Londres (la)},
ititle= {Tour de Nantes (la)}]

\footnotetext {Parodie de la chanson Dans les prisons de Nantes. Autre titres : \emph{La tour de Nantes}. \emph{Dans la tour de Londres}.}

\beginverse
Dans une tour de Londres
\bis {Là-haut,}
Dans une tour de Londres
\bis {Y'avait un prisonnier.}
\endverse

\beginverse
Il n'y voyait personne
\bis {Là-haut,}
Il n'y voyait personne
\bis {Que la fill' du geôlier.}
\endverse

\beginverse
Un jour, il lui demande ...
\bis {La clef du cabinet. }
\endverse

\beginverse
Il s'assit sur le trône ...
\bis {Et se mit à chi-er.}
\endverse

\beginverse
En attendant qu' ça sèche ...
\bis {Il se mit à chanter.}
\endverse

\beginverse
J'emmerde la police ...
\bis {Et la maréchaussée.}
\endverse

\beginverse
Les gendarm's l'entendirent ...
\bis {Et vinr'nt le trucider.}
\endverse

\beginverse
La moral' de l'histoire ...
Est qu'il faut pas chi-er
Sans avoir du papier.
\endverse

\endsong