\beginsong {De profundis morpionibus\footnotemark} [
ititle= {De profundis morpionibus},
ititle= {Mort, l'apparition et les obsèques du capitaine Morpion (la)},
tu= {Marche funèbre (M. : M. Reyer, 1852)}]

\footnotetext {La première version de la chanson, \emph{la Mort, l'Apparition et les Obsèques du Capitaine Morpion}, a été publiée en 1864 dans Le Parnasse Satyrique du XIXe siècle. L'auteur était Théophile Gautier. C'est cette première version qui figure en italique dans ce recueil - mis à part le refrain. Une seconde version en 13 couplets suivit en 1866 dans Le Nouveau Parnasse Satyrique du XIXème siècle. Une publication de 19 couplets apparu en 1913 dans l'Anthologie Hospitalière et Latinesque en donnant le titre \emph{le Combat des Poux et des Morpions}. Le titre De profundis morpionibus apparut entre 1866 et 1911. Quant à cette version, elle compte 33 couplets.}

\beginverse
O ! Muse prête-moi ta lyre,
Afin qu'en vers je puisse dire
Un des combat les plus fameux,
Qui s'est déroulé sous les cieux.
\endverse

\beginchorus
\textbf {Refrain}
De profundis morpionibus
\bis {Tra, la, la, la, la, la, ... }
\endchorus

\beginverse
Un jour de fet' comm' Saint'-Thérèse,
A Saint'-Gudul' chantait la messe
Elle sentit soudainement
Un énorme chatouillement.
\endverse

\beginverse
\textit {Cent milles poux de forte taille
Sur la motte ont livré bataille
A nombre égal de morpi-ons
Portant écus et morions.}
\endverse

\beginverse
Dans un bouzin de tous les diables,
Le choc fut si épouvantable
Qu' les femm's enceint's en accouchant
Chiaient d' la merde au lieu d'enfants.
\endverse

\beginverse
La bataille fut gigantesque,
Tous les morpions mourur'nt ou presque
à l'exception des plus trapus
Qui s'accrochèr'nt aux poils du cul.
\endverse

\beginverse
Le général, nouvel Enée,
Sortant des rangs de son armée,
A son rival, beau chevalier,
Propose un combat singulier.
\endverse

\beginverse
C'est un général plein d'audace
Descendant de l'antique race
Des morpi-ons que Mars donna
A Vénus quand il la baisa.
\endverse

\beginverse
Un morpi-on motocycliste,
Prenant la raie du cul pour piste
Dans un virage dérapa
Et dans la merde s'enlisa.
\endverse

\beginverse
Monté sur une pair' d'échasses
Un vieux morpion que l'on pourchasse,
Sur une motte trébucha,
Les yeux au ciel il expira.
\endverse

\beginverse
Puis au plus fort de la bataille,
Soudain frappé par la mitraille,
Le maréchal des morpi-ons
Tomba mort à l'entrée du con.
\endverse

\beginverse
Un morpion de nobl' origine,
Qui revenait du bout d' la pine,
Levant sa lance s'écria :
"Le morpion meurt, mais n' se rend pas !"
\endverse

\beginverse
Et ils bouchent tout' la fente,
Que les morpions morts ensanglantent
Et la vallée du cul au con
Était jonchée de morpi-ons.
\endverse

\beginverse
Et pour reprendre l'avantage,
Les morpions luttaient avec rage ;
Mais leurs efforts fur'nt superflus,
Les poux gardèrent le dessus.
\endverse

\beginverse
A cheval sur une roupette,
Tenant à la main sa lorgnette,
Le capitaine des morpions
Examinait les positions.
\endverse

\beginverse
Soudain, voyant plier son aile,
Il dit à ses troupes fidèles :
" Ah ! Mes amis ! Nous somm's foutus,
Piquons un' charge au fond du cul. "
\endverse

\beginverse
\textit {Transpercé malgré sa cuirasse
Faite d'une écaille de crasse,
Le capitaine Morpi-on
Est tombé mort au bord du con.}
\endverse

\beginverse
\textit {En vain la foule désolée,
Pour lui dresser un mausolée
Pendant huit jours chercha son corps.
L'abîme ne rend pas les morts !}
\endverse

\beginverse
\textit {Un soir, au bord de la ravine,
Ruisselant de foutre et d'urine,
On vit un fantôme tout nu
A cheval sur un poil de cul.}
\endverse

\beginverse
\textit {C'était l'ombre du capitaine
Dont la carcasse de vers pleine
Par défaut d'inhumati-on
Sentait le maroill's et l'arpion.}
\endverse

\beginverse
\textit {Devant cette ombre qui murmure,
Triste, faute de sépulture,
Tous les morpi-ons font serment
De lui él'ver un monument.}
\endverse

\beginverse
En vain l'on chercha sa dépouille
Sur la pine et sur les deux couilles.
On ne trouva qu'un bout de queue
Qu'un sabre avait coupé en deux.
\endverse

\beginverse
\textit {On l'a recouvert d'une toile
Où de l'honneur brille l'étoile
Comme au convoi d'un général
Où d'un garde nati-onal.}
\endverse

\beginverse
\textit {Son cheval à pied l'accompagne :
Quatre morpi-ons grands d'Espagne
La larme à l'oeil, l'écharpe au bras,
Tiennent les quatre coins du drap.}
\endverse

\beginverse
\textit {On lui bâtit un cénotaphe
Où l'on grava cette épitaphe :
Ci-gît un morpi-on de coeur,
Mort vaillamment au champ d'honneur.}
\endverse

\beginverse
Douze des plus jolies morpionnes
Portèr'nt en pleurant des couronnes
De fleurs blanch's et de poils du cul
Qu'avait tant aimé le vaincu.
\endverse

\beginverse
Restés un peu plus en arrière,
Assis en rond sur leur derrière,
La crott' au cul, la larm' à l'oeil,
Tous les morpions étaient en deuil.
\endverse

\beginverse
Au bord du profond précipice,
On rangea les morpions novices
Ils défilèr'nt en escadrons
En faisant sonner leurs clairons.
\endverse

\beginverse
Tandis que la foule en détresse,
Tout en pleurant disait la messe,
L'adversaire de l'onguent gris
Monta tout droit au Paradis.
\endverse

\beginverse
Sur une couill' grosse et velue,
On érigea une statue
Au capitaine des morpions,
Mort bravement au fond d'un con.
\endverse

\beginverse
Et l'on en fit une relique
Que l'on mit dans un' basilique
Pour que les futurs bataillons
Sachent comment meurt un morpion.
\endverse

\beginverse
Depuis ce jour, on voit dans l'ombre,
A la porte d'un caveau sombre,
Quatre morpions de noir vêtus,
Montant la garde au trou du cul.
\endverse

\beginverse
Depuis ce temps dans la vallée,
On entend des bruits de mêlée,
Les ombres des morpions vaincus
Hant'nt à jamais les poils du cul.
\endverse

\beginverse
Et parfois par les soirs de brume,
Quand sur la terr' se lèv' la lune,
On voit les âmes des morpions
Voltiger sur les poils du con.

\textbf{FIN}
\endverse

\endsong