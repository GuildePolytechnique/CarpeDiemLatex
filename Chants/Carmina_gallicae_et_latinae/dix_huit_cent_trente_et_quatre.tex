\beginsong {1834\footnotemark}[
ititle= {Dix-huit cent trente quatre},
ititle= {Chanson du 150ième anniversaire de l'ULB}]

\footnotetext {Autre titre : \emph{Chanson du 150ème anniversaire de l'ULB}. Auteurs : Éric Saintrond - Corinne Fievet ; Concours UAE de la chanson du 150ème anniversaire de l'ULB.}

\beginverse
Dix-huit cent trente quatre,
Malines s'installant
Se réservant la carte
De notr' enseignement
Seul' une poignée d'hommes
Bien vite a réagi
A ces marchands de Rome
Qui vend'nt un paradis
\endverse

\beginchorus
\textbf {Refrain}
150 ans déjà, il leur en a fallu du cran
150 ans déjà, contre ce clergé si puissant
150 ans déjà, qu'est née notr' Université
150 ans de droit, d'humour et de fraternité.
\endchorus

\beginverse
Dix-huit cent trente quatre,
Malines et puis Louvain
Le mouton suit son pâtre
Il choisit son destin
Mais Bruxell's sur ses gardes
Veillant la liberté
Se défend de la harde
Et crée notr' ULB.
\endverse

\beginverse
Dix-huit cent trente quatre,
Verhaegen et consort
Un siècle nous en écarte
Mais ils ne sont pas morts !
Car tout ce que nos frères
Ont construit de leurs mains
Jamais une prière
N'en causera la fin.
\endverse

\beginverse
Dix-huit cent trente quatre,
Vérité à la science
Que chacun joue ses cartes
Gar' à l'intolérance !
Car le mât de cocagne
Où pend'nt leurs saint's pensées
S'élève avec hargne
Quand y mont'nt nos idées.
\endverse

\beginverse
Dix-neuf cent quatre-vingt quatre,
Où donc est notre histoire ?
A-t-elle rejoint Socrate
Dans le fond d'un tiroir ?
Savent-ils bien encore
Tous ceux qui nous entourent
Qui planta le décor(e)
Où ils viv'nt chaque jour ?
\endverse

\endsong
