\beginsong {Le droguiste \footnotemark}[
ititle={Droguiste, Le},
ititle= {Boules de naphtaline, Les}]

\footnotetext {Autre titre : \emph{Les boules de naphtaline}.}

\beginverse
Il était, au fond d'une officine,
Un droguiste avec son calot blanc
Qui vendait des boul's de naphtaline
Et des r'mèd's contre les rag's de dents.
Les p'tits jeun's gens du voisinage
V'naient lui ach'ter des p'tits vêt'ments
Et la cli-entèle de passage
Lui ach'tait des r'mèd's et des onguents.
\endverse

\beginchorus
Contre les petit's bêtes,
Les morpions endurcis,
\bisdeux {Qu'on attrap' sur la quéquette} {Quand on bais' à vil prix.}
\endchorus

\beginverse
Un beau jour entra dans l'officine
Un vieux bonze, un ancien commandant,
Qui voulait des boul's de naphtaline
Et r'nouv'ler sa provision d'onguent.
Dans le mêm' papier d'emballage
On lui env'loppa c' qu'il d'mandait,
Et le soir, notre haut personnage
En chantant, défaisait son paquet
\endverse

\beginchorus
Contre les petit's bêtes
Il mit de l'onguent gris
\bisdeux {Et branlant d' la quéquette} {Fut baiser à vil prix.}
\endchorus

\beginverse
Notre beau, plus heureux qu'Henri IV
Rencontra une horreur du trottoir ;
Pour cent sous, inutil' de rabattre
Elle voulut bien faire son devoir,
Il avait payé la gonzesse,
Il allait lui percer l' vagin
Quand soudain, la môm', serrant les fesses,
S'écria : " Va donc fair' ça plus loin ...
\endverse

\beginchorus
Et là ! Vieux, bas la pine
Et passe ton chemin,
\bisdeux {Tu pues la naphtaline } {Va baiser les mann'quins. "}
\endchorus

\endsong