\beginsong {En revenant de la foire\footnotemark} [
ititle= {En revenant de la foire}]

\footnotetext {Saint Cloud, dans la banlieue de Paris, est connue pour sa fabrique de bougies. Sa foire automnale était un lieu propice aux rassemblements d'étudiants venus du Quartier Latin.}

\beginverse
En revenant de la foire
La a la a la a la a la a la a la
En revenant de la foire
\bis {De la foire de Saint-Cloud.}
\endverse

\beginverse
Je rencontrai une belle
La a la a la a la a la a la a la
Je rencontrai une belle
\bis {Qui me demanda cent sous. }
\endverse

\beginverse
Pour acheter une robe ...
\bis {Une robe de quatr' sous. }
\endverse

\beginverse
Mais la rob' était si courte ....
\bis {Qu'on y voyait par dessous.}
\endverse

\beginverse
On voyait une chapelle ...
\bis {La chapelle de Saint-Cloud. }
\endverse

\beginverse
Pour entrer dans cett' chapelle ...
\bis {Fallait se mettr' à genoux. }
\endverse

\beginverse
Et tenir une chandelle ...
\bis {Qui n'ait pas de mèch' au bout. }
\endverse

\beginverse
Car s'il y avait une mèche ...
\bis {Elle aurait mis l' feu partout. }
\endverse

\beginverse
Et les pompiers du village ...
\bis {N'en pourraient venir à bout. }
\endverse

\endsong