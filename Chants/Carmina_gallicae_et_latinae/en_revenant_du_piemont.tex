\beginsong {En revenant du Piémont\footnotemark} [
ititle= {En revenant du Piémont}]

\footnotetext {Chanson du début XIXème siècle. Dans le recueil "69 Chansons d'étudiants" (1984), le cinquième couplet n'apparaît pas ; serait-il apocryphe ? Il est difficile de le déterminer étant donné les multiples versions que cette chanson a eu, et son apparition tardive dans "Les Fleurs du Mâle" (1972).}

\beginverse
\bis {C'était en rev'nant du Piémont }
\bis {Nous étions six jeunes garçons }
De l'argent nous n'en avions guère,
Sans dessus dessous et sans devant derrière
À nous six nous n'avions qu'un sou.
\bis {Sans devant derriàre et par derrièr' surtout !}
\endverse

\beginverse
\bis {Nous arrivâm's à un logis.}
\bis {" Madam' l'hôtess', qu'avez-vous cuit ? "}
" J'ai du lapin, du civet de lièvre,
Sans dessus dessous et sans devant derrière,
Et de la bonne soup' aux choux.
\bis {Sans devant derrière et par derrièr' surtout ! " }
\endverse

\beginverse
\bis {Et quand nous eûmes bien dîné,}
\bis {" Madam' l'hôtesse où nous loger ? "}
" Vous coucherez sur la litière, ...
Ou bien vous couch'rez avec nous. ... "
\endverse

\beginverse
\bis {Sur les onz' heur's on entendit}
\bis {L'hôtesse pousser de grands cris :}
" Vous m'avez rompu la charnière, ...
Allez-y donc un peu plus doux. ... "
\endverse

\beginverse
\bis {Et la bonn' qui était en bas}
\bis {Dit : " N'y en a-t-il pas pour moi ? "}
" Y'en aura pour la chambrière, ...
Car nous tirons chacun six coups. ... "
\endverse

\beginverse
\bis {Mais quand ce fut sur les minuits,}
\bis {Il se fit un bien plus grand bruit ;}
Le lit du d'ssus se fichait par terre, ...
Avec la bonn' qui baisait d'ssous. ...
\endverse

\beginverse
\bis {" Quand vous repass'rez par ici}
\bis {Souvenez-vous du bon logis}
Souvenez-vous de la bonn' hôtesse,
Qui savait si bien remuer les fesses
Et d' la p'tit' bonne au lit si doux.
\bis {Sans devant derrière et par derrièr' surtout ! "}
\endverse

\endsong