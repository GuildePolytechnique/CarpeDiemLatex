\beginsong {Le grand vicaire\footnotemark} [
ititle= {Grand vicaire (le)},
ititle= {Curé priviligié (le)},
ititle= {Curé et son vicaire (le)}]

\footnotetext {Autre titre : \emph{Le curé privilégié} , \emph{Le curé et son vicaire}}

\beginverse
Chez nous, la musique
Est fort en pratique ;
Moi, je fais d' l'accordéon,
Et ma femm' du vi-olon,
\bis {Et le curé la viole.}
\endverse

\beginchorus
\textbf {Refrain}
Mais le grand vicaire
Toujours par derrière,
N'a jamais pu la violer
\bis {Et c'est ce qui l'emmerde\footnote {À adapter à chaque fois avec le dernier verbe ou "jeu de verbe" du dernier vers de chaque couplet.}}
\endchorus

\beginverse
Chez nous, la rivière
Est fort passagère ;
Moi, j' la passe à l'aviron,
Et ma femme sur le pont,
\bis {Et le curé la saute.}
\endverse

\beginverse
Chez nous, la méd'cine
A fort bonne mine ;
Moi, j' m'occupe de la charpie,
Et ma femm' des bistouris,
\bis {Et le curé des bandes.}
\endverse

\beginverse
Chez nous, les voyages
Sont fort en usage ;
Moi, j'ai visité l'Asie,
Et ma femme la Russie,
\bis {Et le curé la Perse.}
\endverse

\beginverse
Chez nous, la culture
Est fort en usure ;
Moi, j' m'occupe de la moisson,
Et ma femm' d' la fenaison,
\bis {Et le curé laboure.}
\endverse

\beginverse
Chez nous, la pendule
Avance et recule ;
Moi, j' m'occup' du balancier,
Et ma femme du boîtier,
\bis {Et le curé la monte.}
\endverse

\beginverse
Chez nous, les costumes
Sont dans la coutume ;
Moi, j' m'occupe des pantalons,
Et ma femme des vestons,
\bis {Et le curé l'enfile.}
\endverse

\beginverse
Chez nous, la coiffure
Fait bonne figure ;
Moi, je port' des chapeaux m'lons,
Ma femme des chapeaux ronds,
\bis {Le curé des calottes.}
\endverse

\beginverse
Chez nous, la charrette
D'vant chez nous s'arrête ;
Moi, j' dételle les mulets,
Ma femme défait les paquets,
\bis {Et le curé décharge.}
\endverse

\beginverse
Chez nous, les breuvages
Sont fort en usage ;
Moi je prends un diabolo,
Et ma femme du Cointreau,
\bis {Et le curé la Suze.}
\endverse

\beginverse
Chez nous, la vaisselle
Est blanch' et fort belle ;
Moi, j' récure la soupière,
Et ma femme la cuillère,
\bis {Et le curé l'astique.}
\endverse

\beginverse
Chez nous, l' tricotage
Est fort en usage ;
J' tonds la lain' des mérinos,
Ma femm' fait des écheveaux,
\bis {Et le curé la p'lote.}
\endverse

\beginverse
Chez nous, les tentures
S'accroch'nt sur mesure ;
Moi, j' m'occupe des anneaux,
Et ma femme des rideaux,
\bis {Et le curé la tringle.}
\endverse

\beginverse
Chez nous, la lecture
Est fort en usure ;
Moi, je lis Victor Hugo,
Et ma femme Marivaux,
\bis {L' curé La Condamine.}
\endverse

\endsong
