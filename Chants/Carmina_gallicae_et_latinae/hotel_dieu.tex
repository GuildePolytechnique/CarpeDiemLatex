\beginsong {L'Hôtel-dieu\footnotemark}[
ititle= {Bal de l'Hôtel-Dieu, Le},
ititle= {Hôtel-Dieu, l'}]

\footnotetext {Il en existe plusieurs versions : \emph{Le bal de l'Hôtel-Dieu}, \emph{La chanson de l'Hôtel-Dieu}. C'est une chanson de salle de garde empruntée au répertoire des artilleurs.}

\beginverse
\bisdeux {Au bal de l'Hôtel-Dieu, nom de Dieu !} {Y'avait une servante.} 
Elle avait tant d'amants, nom de Dieu !
Qu'elle ne savait l'quel prendre.
\endverse

\beginchorus
\textbf {Refrain \footnote{N'est renseignée ici que la version belge du refrain.}}
Ah, nom de Dieu ! Nom de Dieu ! Nom de Dieu !
Crénom de Dieu ! Nom de Dieu ! Nom de Dieu !
Ah, nom de Dieu ! Nom de Dieu ! Nom de Dieu !
Ah, nom de Dieu, quelle allure !
Ah, nom de Dieu ! Nom de Dieu ! Nom de Dieu !
Ah, quelle allure ! Nom de Dieu !
\endchorus

\beginverse
\bisdeux {Elle avait tant d'amants, nom de Dieu !} {Qu'elle ne savait l'quel prendre.}
Un jour l'intern' de gard', nom de Dieu !
En mariag' la demande.
\endverse

\beginverse
... Le pèr' ne dit pas non, nom de Dieu !
La mèr' est consentante.
\endverse

\beginverse
... Malgré tous les envieux, nom de Dieu !
Ils coucheront ensemble.
\endverse

\beginverse
... Dans un grand lit carré, nom de Dieu !
Tout garni de guirlandes.
\endverse

\beginverse
... Aux quatre coins du lit, nom de Dieu !
Quatr' carabins qui bandent.
\endverse

\beginverse
... La bell' est au milieu, nom de Dieu !
Elle écarte les jambes.
\endverse

\beginverse
... Les règl's lui sort'nt du con, nom de Dieu !
Encor' toutes fumantes.
\endverse

\beginverse
... Vous tous qui m'écoutez, nom de Dieu !
Y passeriez la langue ?
\endverse

\endsong