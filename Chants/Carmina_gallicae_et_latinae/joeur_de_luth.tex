\beginsong {Le joueur de luth\footnotemark} [
ititle= {Joueur de luth, Le},
ititle= {Joueur de turlututu, Le},
ititle= {Épinette, L'},
ititle= {Auberge de l'écu, L'}]

\footnotetext {Autres titres : \emph{Le joueur de turlututu}, \emph{L'épinette}, \emph{L'auberge de l'écu}.}

\beginverse
\bis {Dans notre vill' est venu}
\bis {Un fameux joueur de luth}
Pour attirer la pratique
Il a mis sur sa boutique :
" C'est ici qu', pour un écu,\footnote {Originale : \emph{A l'auberge de l'écu}.}
On apprend à jouer de l'épinette,
C'est ici qu', pour un écu,
On apprend à jouer du ... "
\endverse

\beginchorus
Refrain
Trou la la, trou la la, trou la, trou la, trou la laire
Trou la la, trou la la, trou la, trou la, trou la la.
\endchorus

\beginverse
\bis {Toutes les fill's de Paris}
\bis {De Versaill's, de Saint-Denis}
Ont vendu leur chemisette,
Leurs jarr'tièr's, leurs collerettes\footnote { Originale : \emph{Leurs jarretières et leurs chaussettes, Pour avoir un p'tit écu Apprendr' à jou-er de l'épinette}}
Afin d'avoir un écu
Pour apprendr' à jouer de l'épinette ...
\endverse

\beginverse
\bis {Un' jeun' fill' se présenta}
\bis {Qui, des leçons, demanda}
" Ah ! Que tes leçons sont bonnes
Il faudra que tu m'en r'donnes ;
Tiens voilà mon p'tit écu
Pour apprendr' à jouer de l'épinette ... "
\endverse

\beginverse
\bis {Une vieill' aux cheveux gris}
\bis {Voulut en tâter aussi.}
" Par la porte de derrière,
Fait's-moi passer la première
T'nez, voilà mon vieil écu,
Pour apprendr' à jouer de l'épinette ... "
\endverse

\beginverse
\bis {" Vieille retournez-vous en}
\bis {Et remportez votr' argent }
Car ce n'est pas à votr' âge
Qu'on entr' en apprentissage
Vous avez trop attendu
Pour apprendr' â jouer de l'épinette ... "
\endverse

\beginverse
\bis {La vieill' en se retournant }
\bis {Marmonnait entre ses dents :}
" Ah ! Vous me la baillez belle
De me croire encor' pucelle
Voilà cinquant' ans et plus
Que je sais jouer de l'épinette ... "
\endverse

\beginverse
\bis {La morale de ceci}
\bis {Je vais vous la dire ici :}
C'est quand on est jeune et belle
Il n' faut pas rester pucelle.
Faut donner son p'tit écu
Pour apprendr' à jouer de l'épinette ...
\endverse

\endsong
