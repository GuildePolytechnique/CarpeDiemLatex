\beginsong {Le Trente et un du mois d'août\footnotemark} [
ititle= {Chanson de Surcouf, la},
ititle= {Au trente-et-un du mois d'août},
ititle= {Trente-et-un du mois d'août, Le}]

\footnotetext {Autre titre : \emph{Chanson de Surcouf}. Chanson à virer au canbestan (voir ce mot) du XVIIIème siècle. Dans
l'originale, on bisse les deux premiers vers de chaque couplet ensemble et non pas séparément.}

\beginverse
Au trent' et un du mois d'a-oût (bis)
Nous vîm's venir sous l' vent à nous (bis)
Une frégate d'Angleterre
Qui fendait le mer z-et les flots :
C'était pour bombarder\footnote {Variante: \emph{attaquer}} Bordeaux.
\endverse

\beginchorus
\textbf {Refrain}
Buvons un coup, buvons en deux,
À la santé des amoureux.
À la santé du Roi de France,
Et merd' pour le Roi d'Angleterre
Qui nous a déclaré la guerre !
\endchorus

\beginverse
Le Capitain' du bâtiment (bis)
Fit appeler son lieutenant, (bis)
" Lieutenant, te sens-tu capable :
Dis-moi, te sens-tu assez fort
Pour prendre l'Anglais à son bord ? "
\endverse

\beginverse
Le lieutenant, fier z-et hardi (bis)
Lui répondit : " Capitain' z-oui ! (bis)
Fait's branle-bas à l'équipage :
Je vas hisser not' pavillon
Qui rest'ra haut, nous le jurons ! "
\endverse

\beginverse
Le maître donne un coup d' sifflet, (bis)
Cargue les voiles du perroquet\footnote {Perroquet : (de perroquet ; 1525) 1. sur les grands voiliers, voile haute, carrée, s'établissant au-dessus des huniers (voir ce mot). 2. Mât sur lequel est établi cette voile. (in Larousse, Dictionnaire de la langue
française Lexis 1992) Il faut donc employer l'article "du" en lieu et place de l'article "au" de "Les Fleurs du Mâle" (1983)}. (bis)
File l'écoute et vent arrière
Laisse porter jusqu'à son bord
On verra bien qui s'ra l' plus fort !
\endverse

\beginverse
Vir' lof pour lof\footnote {Lof : (du néerl. loef ; 1138) 1. côté du navire qui se trouve frappé par le vent. 2. Commandement pour mettre la barre sous le vent, de sorte que le navire vienne au vent. Virer lof pour lof : virer vent arrière. (in Larousse, Dictionnaire de la langue française Lexis 1992)}, au même instant (bis)
Nous l'attaquâm's par son avant (bis)
À coups de haches d'abordage,
De sabres, piqu's et mousquetons,
Nous l'eûm's vit' mis à la raison.
\endverse

\beginverse
Que dira-t-on dudit bateau (bis)
En Angleterr' z-et à Bordeaux (bis)
Qu' a laissé prendr' son équipage
Par un corsair' de six canons,
Lui qu' en avait trente et si bons ?
\endverse

\endsong