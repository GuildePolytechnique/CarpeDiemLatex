\beginsong {Avez-vous chanté la lune}[
tu={Que ne suis-je la fougère. (P.: Charles Joseph Prince de Ligne (XVIIIéme siècle))}
ititle={Avez-vous chanté la lune}]

\beginverse
" Avez-vous chanté la lune ? "
Me disait-on l'autre jour.
L'envie en est si commune
Que chacun l'eût à son tour.
" Non, dis-je, pour confidente
Mon amour n'en veut jamais,
Et ma tendresse éclatante
N'aime pas ses doux reflets. "
\endverse

\beginverse
Je veux que celle que j'aime
Soutienne le plus grand jour,
Je veux que le Soleil même
Soit jaloux de mon amour ;
S'il venait à disparaître
Mon coeur je crois suffirait :
On croirait le voir renaître
Tant sa chaleur brûlerait.
\endverse

\beginverse
Cette lune qu'on célèbre
Si souvent en jolis vers
N'a qu'une pâleur funèbre
Éclairant mal l'univers.
Elle n'est jamais la même,
Ses caprices différents
Font qu'on quitte ceux qu'on aime,
C'est l'astre des inconstants.
\endverse

\beginverse
Son croissant n'est que l'image
Du malheur de tant d'époux ;
Et la lune en plein visage
Est un signal pour les fous.
Du soleil ou de mon âme
Je recommande les feux,
Que de mes ardeurs la flamme
Consume ce que je veux.
\endverse

\endsong