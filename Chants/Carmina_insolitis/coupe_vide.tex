\beginsong {La coupe vide\footnotemark} [
ititle= {Coupe vide (La)},
tu= {Mon père était pot.}]

\footnotemark {P. : Maximilien de Robespierre (XVIIIème siècle).}

\beginverse
Oh mes amis, tout buveur d'eau, et vous pouvez m'en croire
Dans tous les temps ne fut qu'un sot, j'en atteste l'histoire :
Ce sage effronté, cynique, vanté, me paraît bien stupide
Oh le beau plaisir d'aller se tapir au fond d'un tonneau vide !
\endverse

\beginverse
Quand l'escadron audacieux des enfants de la terre
Jusque dans le séjour des Cieux osa porter la guerre
Bacchus rassurant Jupiter tremblant décida la victoire :
Tous les dieux à jeun tremblaient en commun, lui seul avait su boire !
\endverse

\beginverse
Il fallait voir dans ses grands jours le puissant dieu des treilles
Tranquille, vidant tour à tour, et lançant des bouteilles
A coups de flacons, renversant les monts sur les fils de la terre
Ces traits dans la main du buveur divin, remplaçaient le tonnerre !
\endverse

\beginverse
Sa main sur les fronts nébuleux et sur leurs faces blêmes
En caractères odi-eux grava cet anathème :
Voyez leur maintien, leur triste entretien, leur démarche timide,
Leur aspect dit bien que comme le mien, leur verre est souvent vide !
\endverse

\endsong