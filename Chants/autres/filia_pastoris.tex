\beginsong {Filia Pastoris\footnotemark} [
ititle= {Filia Pastoris}]

\footnotetext {Le texte initial, strophes 1-2-4 et 5, est austro-allemand. Après être tombé en désuétude chez les Allemands, il réapparut vers la fin du XIXe siècle à Leuven ; ce qui lui valu sa strophe néerlandaise. Le couplet français a été rajouté par l' "Ordo Vagorum" (en italique). Pour plus d'information sur ce sujet, consulter le "Codex Studiosorum Latino-Gallicus" (1986, Ordo Vagorum) à la page 74.}

\beginverse
\bisdeux {Quæ voluptas, quæ voluptas est amare} {Pulchram filiam pastoris !}
\bisdeux {O admiranda, o admiranda, } {O admiranda filia pastoris ! }
\endverse

\beginverse
\bisdeux {Tis hëdonë, tis hëdonë estin agapân} {Kalën paida poimenos }
\bisdeux {Ö thaumasia, Ö thaumasia,} {Ö thaumasia paidos poimenos !}
\endverse

\beginverse
\bisdeux {Welch Vergnûgen, welch vergnügen ist 's zu lieben} {Das Hirten schönstes Töchterlein}
\bisdeux {O wunderbares, o wunderbares, } {O wunderbares Hirtentöchterlein ! }
\endverse

\beginverse
\bisdeux {Sossorasto, sossorasto estie chainie,} {Pierko tchourpu pastora !}
\bisdeux {O navitchainia, o navitchainia, |} {O navitchainia tchourpu pastora !}
\endverse

\beginverse
\bisdeux {Welk genoegen, welk genoegen is 't te minnen} {'t Mooiste meisje van de stad !}
\bisdeux {O wonderbaarste, o wonderbaarste,} {O wonderbaarste meisje van de stad ! }
\endverse

\beginverse
\bisdeux {Qué delicia, qué delicia cuando amas} {A la hija del pastor !}
\bisdeux {O estupenda,  o estupenda} {O estupenda hija del pastor !}
\endverse

\beginverse
\textit {\bisdeux {Quel délice, quel délice quand on aime} {Cette fille du berger !}
\bisdeux {O merveilleuse, o merveilleuse,} {O merveilleuse fille du berger !}}
\endverse

\endsong