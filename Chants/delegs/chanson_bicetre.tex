\beginsong {La chanson de Bicêtre\footnotemark} [
ititle= {Chanson de Bicêtre},
ititle= {Bicêtre}]

\footnotetext {Chanson sans doute écrite entre 1846 et 1851. Le Bicêtre est un hospice de la commune de Kremlin-Bicêtre (Val-de-Marne) construit à l'origine par Louis XIII pour les soldats estropiés.

\emph{Cette chanson est reprise pour le chant de l’Ordre des Vieux Cons}}

\beginverse
Dans ce Bicêtre où l'on s'embête,
Loin de Paris que je regrette,
J'ai bien souvent et longtemps médité
Sur la vieilless' et la caducité.
Or, écoutez ce refrain de Bicêtre,
Cette leçon vous servira peut-être :
\endverse

\beginchorus
\textbf {Refrain}
On n' peut pas bander toujours,
Il faut jou-ir de ses roupettes,
On n' peut pas bander toujours,
Il faut jou-ir de ses amours.
\endchorus

\beginverse
D'un vieux, un jour je tenais la quéquette
La sond' en main, de l'autre la cuvette,
Pendant ce temps, mon esprit méditait,
Ce que tout en bas une voix\footnote {Originale: \emph{le vieillard.}} me disait :
" Prenez bien soin de ces pauvres gogottes,
Vous en viendrez à pisser sur vos bottes. "\footnote {Originale: \emph{Un jour viendra, vous piss'rez sur vos bottes.}}
\endverse

\beginverse
Idi-ot, fou, épileptique
Sont des argu-ments sans réplique.
Tout dépérit, le pauvre genr' humain
N'a plus d'espoir que dans le carabin.
Or, pour créer une race nouvelle,
Jamais, enfants, ne mouchez la chandelle.
\endverse

\beginverse
Quand la vieillesse trist' et caduque
Vous foutra son pied sur la nuque,
Quand votre vit à jamais désossé,
Sur vos roustons, pendra flasqu' et glacé,
Au mêm' instant, crachez au nez du traître,\footnote {Originale : Amis, crachez à la face du traître,}
Répétez-lui ce refrain de Bicêtre :
\endverse

\beginverse
À l'oeuvre donc, jeunes athlètes.
Gaillardement, engrossez les fillettes,
Baisez, foutez, ne craignez nul écueil.
Quand on est jeun' il faut baisez à l'oeil.
Avec le temps, Vénus devient avare.
Aux pauvres vieux, le coup est cher ... et rare.
\endverse

\endsong