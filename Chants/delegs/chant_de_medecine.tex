\beginsong {Chant de Médecine ULB \footnotemark (C.M.)} [
ititle= {Chant de Médecine ULB},
ititle= {Vérolés, Les},
ititle= {Marche des véolés, la},
ititle= {Chant de Lourcine, la},
tu= {Marche des Vérolés (ou Hymne des étudiants carabins)}]

\footnotetext {Autre titre : \emph{Les vérolés}, \emph{La marche des vérolés}, \emph{La chanson de Lourcine} (in 69 Chansons d'Étudiants, 1984). Il est dommage que l'air de cette belle chanson d'amour ait été modifié, la rendant plus pesante à chanter, et il est surtout regrettable que même "Les Fleurs du Mâle" (1983), référence de la chanson estudiantine, s'il en est, à l'ULB, n'ait pu reproduire de manière correcte ces paroles.}

\beginverse
De l'hôpital vieille pratique,
Ma maîtresse est une putain
Dont le vagin syphilitique
Infeste le Quartier Latin.
Mais moi, vieux pilier de l'école,
Je l'aime à cause de son mal, oui, de son mal,
Nous somm's unis par la vérole
\ter {Mieux que par le lien conjugal.}
\endverse

\beginverse
Tous les matins, vidant nos verres,
Nous y pompons avec entrain.
Nous partageons comme des frères
Les pilules de Dupuytren.
Nous vivons et baisons ensemble
Heureux comme des demi-dieux, des demi-dieux.
Et c'est la plus bell' existence
\ter {Pour des amants toujours heureux.}
\endverse

\beginverse
Nous transformons en pharmacie
Le lieu sacré de nos amours ;
La valériane et la charpie\footnote {Originale: \ter {\emph{Les plumasseaux et la charpie S'y confectionnent tour à tour. Tandis qu'avec le bichlorure, Ell' me faisait des frictions, Avec ma seringu' de mercure, Moi je lui fais des injections.}}}
S'y manipulent tour à tour.
Tandis qu'avec de l'iodure,
Ma femm' me fait des injections, des injections,
Avec du bromure de mercure,
\ter {Moi je lui fais des frictions.}
\endverse

\beginverse
Ses cuiss's ont des reflets verdâtres,
Ses seins sont flasques et flétris,
Dans son con,\footnote {Originale : \emph{Au sommet}} des morpions jaunâtres
Sur le fumier ont leur logis.
Pourtant, j'aime mon amante
Et je voudrais jusqu'à demain, jusqu'à demain !
Lécher de ma lèvre brûlante
\ter {Le foutre de son vieux vagin.}
\endverse

\beginverse
Délassement de l'innocence,
Je regarde chaque matin
Si quelque nouvell' excroissance
Ne vient pas orner son vagin
Tandis qu'avec un oeil humide
Elle jett' un timid' regard, timid' regard
Sur mon corps que les syphilides
On taché comm' un léopard. (ter)
\endverse

\beginverse
Et quand viendras l'heure dernière\footnote { Originale : \emph{Quand nous serons las de la terre Nous cesserons tout traitement Et, rongé par un vast' ulcère/ Ad patres nous irons gaiement. Mais nous ferons une supplique Pour être tous les deux portés, tous deux portés ...}}
Quand nous s'rons mangés des morpions
Unis dans un dernier ulcère
Ad patres, gaiement, nous irons.
Nous adress'rons une supplique
Afin qu' nous soyons exposés, oui, exposés
Dans un musée pathologique
À la section des vérolés. (ter)
\endverse

\endsong 
