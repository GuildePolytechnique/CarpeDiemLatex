\beginsong {Chant de Philo ULB (C.P.L.)}[
ititle= {Chant de Philo. ULB},
tu= {Le chant du départ (Étienne Nicolas Méhul, 1794)}]

\beginverse
C'est le chant de PHILO.
Partons à la guindaille
La pine en fleur,
Les roustons en chaleur ;
Comm' de francs saligauds,
Courons à la ripaille,
Bourreaux des coeurs,
Toujours avec ardeur
Les petits et les grands cons
Nous les baisons
Et du soir au matin,
Notre pine guerrière
Fera jou-ir bon nombre de vagins.
\endverse

\beginverse
\textit {Parlé :}
À la PHILO., crénom de nom !
On est peu d' poils, mais on est bon !
\endverse

\endsong