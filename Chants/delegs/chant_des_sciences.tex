\beginsong {Chant des Sciences ULB (C.d.S.}]
ititle= {Chant des Sciences ULB},
tu= {La Marseillaise (Claude Rouget de Lisle, 1792) (P: Paul Hubinon, 1965)}]

\beginverse
Venez, venez, petites filles,
Le jour de rut est arrivé.
Les étudi-ants de chimie
\bis {Ont la pine bien échauffée}
Entendez-vous dans nos campagnes
La gé-ographie en chaleur
Et les matheux si bons baiseurs
Travailler vos mignonnes compagnes ?
\endverse

\beginchorus
Refrain
Aux pines, CdS,
Enl'vons nos pantalons.
Baisons, baisons
Qu'un sperme pur
Abreuve tous ces cons.
\endchorus

\beginverse
Les physiciens aim'nt les béguines
Pour leurs cons molass's mais sacrés
Et les béguin's préfèr'nt leurs pines
\bis {Aux crucifix froids et dorés}
Les botanist's, avec tendresse,
Recueillent les fleurs de tièdes bosquets
Où coulent de gluants pisselets
Entre les monts que l'on nomme fesses.
\endverse

\beginverse
Quand on est en biologie,
On a le sperm' gras et grouillant
C'est qu'à forc' d'él'ver des bactéries,
\bis {On s'y prend mieux pour le rendre consistant}
Les géologu's dans les soutanes,
À grands coups de pics z-et de burins,
Ont cherché d' génitaux organes
Mais n'ont trouvé que d'hybrides machins.
\endverse

\endsong
