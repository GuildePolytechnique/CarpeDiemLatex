\beginsong {Chant du C.E.G. (InRaCi, Bxl)} [
ititle= {Chant du C.E.G.},
ititle= {La Marseillaise (Claude Rouget de Lisle, 1792)}]

\beginverse
Allons enfants de la guindaille
Le CEG est arrivé !
Contre nous de la sobriété
\bis {La chope sacrée est levée}
Entendez-vous dans les tavernes
Chantez ces bons poils et ces plum's
Qu'ils viennent jusque dans vos bras
Dévoyer les bleus et les bleuettes.
\endverse

\beginchorus
Refrain
Aux chopes guindailleurs !
La bière coul' à flots.
Buvons, buvons qu'un' bière pure
Abreuve nos gosiers.
\endchorus

\beginverse
Amour sacré de la guindaille,
Conduis, soutiens nos bas instincts
Paillardise, paillardise chérie
\bis {Jamais tu ne nous abandonnes}
Chaque plume et chaque bleuette
Accourent à nos mâles pennins
Et que la calott' expirante
Voie notr' triomph' et notre gloire.
\endverse

\beginverse
Nos bleus sont dans la guindaille
Car ils sont bien distingués
Ils y trouveront les meilleurs
\bis {Et les traces de nos ripailles}
Bien plus soûlards que la calotte
Que nous enverrons au cercueil
Ils auront au sublime orgueil
D'êtr' CEG et de nous suivre !
\endverse

\endsong