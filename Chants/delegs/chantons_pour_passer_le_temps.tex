\beginsong {Chantons pour passer le temps\footnotemark} [
ititle= {Chantons pour passer le temps},
ititle= {Chanson de la Guilde Horus}]

\footnotetext {Chanson à virer au cabestan (de cabestre, corde de poulie ; 1382). C'est un treuil à axe vertical, actionné au moyen de barres d'anspect enfoncées horizontalement, autour duquel on enroule un câble pour tirer l'ancre, par exemple. (in Larousse, Dictionnaire de la langue française Lexis 1992)}

\beginverse 
(\textit {Solo})	Chantons, pour passer le temps
		Les amours charmants d'une belle fille,
(\textit {Choeur}) 	Chantons, pour passer le temps
		D'une belle fille, les amours charmants.
(\textit {Solo}) 	Aussitôt que son amant l'eût prise,
(\textit {Choeur}) 	Aussitôt elle changea de mise,
(\textit {Solo})	Et prit l'habit de matelot
		Et vint s'embarquer à bord du navire,
(\textit {Choeur}) 	Et prit l'habit de matelot
		Et vint s'embarquer à bord du vaisseau.
\endverse

\beginverse
Le capitain' enchanté
D'avoir à son bord un si beau jeun' homme,
Le capitain' enchanté
Lui dit : " à mon bord, je vais te garder.
Tes beaux yeux, ton joli visage,
Tes cheveux, ton joli corsage
Me font toujours me rappeler
D'anciennes amours avec une belle
Me font toujours me rappeler
Un' beauté d' jadis que j'ai tant aimée ! "
\endverse

\beginverse
" Monsieur, vous vous moquez de moi
Vous me badinez, vous me faites rire.
Je n'ai ni frère ni parents
Et ne suis pas née z-au port de Lorient.
Je suis née z-à la Martinique,
Je suis même z-un' enfant unique
Et c'est un vaisseau hollandais
Qui m'a débarquée en venant des îles,
Et c'est un vaisseau hollandais
Qui m'a débarquée au port de Calais ! "
\endverse

\beginverse
Ils ont bien vécu sept ans
Sur le bâtiment sans se reconnaître
Ils ont bien vécu sept ans
Se sont reconnus au débarquement
" Puisqu'ici l'amour nous rassemble,
Nous allons nous mari-er ensemble
L'argent que nous avons gagnée,
Elle nous servira z-à notre ménage,
L'argent que nous avons gagnée,
Elle nous servira z-à nous mari-er ! "
\endverse

\beginverse
C'ui-là qu' a fait la chanson
C'est le gars Camus, le gabier\footnote {Gabier (1678) : matelot autrefois préposé aux voiles et au gréement. (in Larousse, Dictionnaire de la langue française Lexis 1992)} d' misaine\footnote {Voile de misaine, ou misaine, basse voile du mât de misaine (de mezzo, médian ; 1382. Mât vertical à l'avant d'un navire, situé entre le grand mât et le beaupré). (in Larousse, Dictionnaire de la langue française Lexis 1992)},
C'ui-là qu' a fait la chanson ;
C'est le gars Camus, l' gabier d'artimon\footnote {Voile d'artimon (du gr. artemôn ; 1246) : voile en forme de trapèze, la plus rapprochée de l'arrière. (in Larousse, Dictionnaire de la langue française Lexis 1992)}.
Oh ! mat'lot, larguez la grand-voile
Aux palans\footnote {Palan (du gr. phalanga ; 1573) : appareil de levage vertical sur courte hauteur. (in Larousse, Dictionnaire de la langue française Lexis 1992)}, que tout le mond' y soye,
Et vir', et vire, vire donc,
Sinon t' auras pas d' vin plein ta bedaine,
Et vir', et vire, vire donc,
Ou t' auras pas ta ration dans l' bedon !
\endverse

\endsong