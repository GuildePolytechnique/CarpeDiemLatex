\beginsong {La femme du roulier\footnotemark} [
ititle= {Femme du roulier (la)},
ititle= {Chant de la Guilde de la Robe de Pourpre (la)}]

\footnotetext {Il est minuit, Chanson recueillie dans le Berry par Maurice Rollinat qui la chantait parfois au Chat Noir. La version originale contient plusieurs différences avec celle de "Les Fleurs du Mâle" (1983), et, de plus, elle ne commence pas par Il est minuit. Remarque : elle se chante à minuit dans certains milieux de la guindaille. La Guilde de la Robe de Pourpre de l'ULB l'utilise en tant que chant de délégation}

\beginverse
La femme du roulier
S'en va de porte en porte,
De tavern' en tave-erne,
\bisdeux {Pour chercher son mari, tireli,} {Avec une lante-erne.}
\endverse

\beginverse
" Madam' l'hôtesse,
Où est donc mon mari ? "
" Ton mari est ici,
Il est dans la soupen-ente.
\bisdeux {En train d' prendr' ses ébats, tirela, } {Avec notre servan-ante. " }
\endverse

\beginverse
" Cochon d' mari,
Pilier de cabaret,
Ainsi tu fais la noce,
Ainsi tu fais ripa-aille,
\bisdeux {Pendant que tes enfants, tirelan, } {Sont couchés sur la pa-aille. }
\endverse

\beginverse
Et toi la belle,
Aux yeux de merlan frit,
Tu m'as pris mon mari,
Je vais te prendr' mesu-ure
\bisdeux {D'un' bell' culott' de peau, tirelo, } {Qui ne craint pas l'usu-ure. " }
\endverse

\beginverse
" Tais-toi, ma femme,
Tais-toi, tu m' fais chi-er,
Dans la bonn' société
Est-ce ainsi qu'on s' compo-orte ?
\bisdeux {J' te fous mon pied dans l' cul, tirelu, } {Si tu n' prends pas la po-orte. " }
\endverse

\beginverse
" Pauvres enfants,
Mes chers petits enfants,
Plaignez votre destin
Vous n'avez plus de pè-ère,
\bisdeux {Je l'ai trouvé couché, tirelé, } {Avec une autre mè-ère. " }
\endverse

\beginverse
" Il a raison,
S'écrièr'nt les enfants,
D'aller tirer son coup
Avec la cell' qu'il ai-aime,
\bisdeux {Et quand nous serons grands, tirelan, } {Nous ferons tous de mê-ême. " }
\endverse

\beginverse
" Méchants enfants,
Sacrés cochons d'enfants
S'écrie la mèr' furieuse\footnote {Nd Mimi: Bien que cette chanson connaisse plusieurs variantes quant à son contenu, il n'a jamais été porté à ma connaissance, dans mes recherches, des deux vers suivants : Taisez vos gueules, vos propos m'exaspèrent. que certain(e)s continuent encore à expectorer avec ferveur. Mais si c'est pour rire, moi j'veux bien !}
Et pleine de colè-ère,
\bisdeux {Vous serez tous cocu, tirelu, } {Comm' le fut votre pè-ère. " }
\endverse

\endsong