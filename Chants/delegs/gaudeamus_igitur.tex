\beginsong {Gaudeamus Igitur\footnotemark} [
ititle= {Gaudeamus Igitur}]

\footnotetext {Ce chant est le plus ancien chant estudiantin transmis par la voie traditionnelle. Ses origines premières sont médiévales ; les strophes 2 et 3 remontent à 1267. Le couplage de l'air au texte n'est attesté qu'en 1738, et le texte complet fut édité en 1781 par Christian Wilhelm Kindleben. Pour plus d'information sur ce sujet, consulter le "Codex Studiosorum Latino-Gallicus" (1986, Ordo Vagorum) à la page 84.}

\beginverse
\bis {Gaudeamus, igitur, juvenes dum sumus}
Post jucundam juventutem
Post molestam senectutem
\bis {Nos habebit humus. }
\endverse

\beginverse
\bis {Ubi sunt qui antes nos in mundo fuere ?}
Vadite ad superos,
Transite ad inferos :
\bis {Ubi jam fuere ?}
\endverse

\beginverse
\bis {Vita nostra brevis est, breve finietur,}
Venit mors velociter,
Rapit nos atrociter.
\bis {Nemini parcetur. }
\endverse

\beginverse
\bis {Vivat academia, vivant professores,}
Vivat membrum quodlibet,
Vivant membra quælibet,
\bis {Semper sint in flore !}
\endverse

\beginverse
\bis {Vivant omnes virgines, graciles, formosæ ! }
Vivant et mulieres,
Teneræ, amabiles,
\bis {Bonæ, laboriosæ !}
\endverse

\beginverse
\bis {Vivat et res publica et qui illam regit !}
Vivat nostra civitas,
Mæcenatum caritas,
\bis {Quæ nos hic protegit ! }
\endverse

\beginverse
\bis {Pereat tristitia, pereant osores }
Pereat diabolus,
Quivis antistudius,
\bis {Atque irrisores ! }
\endverse

\endsong
