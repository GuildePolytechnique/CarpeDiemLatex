\beginsong{À Gennevilliers\footnotemark}[
  ititle={À Gennevilliers},
  ititle={Beau grenadier, Le},
  ititle={Fille de Gennevilliers, le},
  tu={Les Filles de Gennevilliers (in Les Fleurs du Mâle, 1972)}]

\footnotetext{Autres titres : \emph{Le beau grenadier}, \emph{La fille de Gennevilliers}.}
  
\beginverse
\bis{A Genn'villiers, y'a d' si tant belles filles}
Mais y'en a z-un' si parfait' en beauté
\bis{Qu'elle a séduit tambours et grenadiers.}
\endverse

\beginchorus
\textbf{Refrain}
\ter{Ah ! Ah !}
\endchorus

\beginverse
\bis{" Beau grenadier, monte dedans ma chambre}
Nous y ferons l'amour en liberté
\bis{Dedans les bras de la volup(e)té ".}
\endverse

\beginverse
\bis{Mais ils n'étaient pas sitôt dans la chambre}
Qu'on entendait que des embrassements
\bis{Dedans les bras de ce nouvel amant.}
\endverse

\beginverse
\bis{Mais l'autr' amant est à la port' qui bisque}
Frappant du pied, levant les bras\footnote{Variante : \emph{yeux}.} aux cieux
\bis{Dit : " Nom de Dieu ! que je suis malheureux !}
\endverse

\beginverse
\bis{D'avoir z-aimé un' si tant belle fille}
Et dépensé mon or et mon argent
\bis{Sans en avoir eu aucun agrément !\footnote{Originale : \emph{Pour n'en avoir que de l'emmerdement !}}}
\endverse

\beginverse
\bis{J'ai bien envie de lui flanquer un' gifle}
Mais elle est femm' et je respecterai
\bis{Son sex' et, seul, à l'homm' je m'en prendrai. "}
\endverse

\beginverse
\bis{Sur le terrain, rencontre son rival(e)}
Et par le corps son sabr' y a passé
\bis{Si bien passé qu'il en est trépassé.}
\endverse

\beginverse
\bis{Oh ! jeunes fill's, ceci doit vous apprendre}
Que quand on veut avoir deux amoureux
\bis{Il faut des deux se méfi-er un peu !}
\endverse

\endsong
