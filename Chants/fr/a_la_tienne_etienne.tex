\beginsong{À la tienne, Étienne}[
  ititle={À la tienne, Étienne}]

\beginverse
Enfants des bords de La Loire,
J' n'ai qu'un tout petit défaut,
C'est d'aimer chanter et boire
Ça n'nous fait ni froid ni chaud.
Saint-Étienne est mon patron
Et chacun dit sans façon :
\endverse

\beginchorus
\textbf{Refrain}
" A la tienne, Étienne,
A la tienne, mon vieux !
Sans ces garc's de femm's
Nous serions tous des frères.
A la tienne, Étienne,
A la tienne, mon vieux !
Sans ces garc's de femm's
Nous serions tous heureux ! "
\endchorus

\beginverse
Ma moitié qui n'est qu'un' buse
Vient toujours, c'est son secret,
A tout's les fois que j' m'amuse,
Me chercher au cabaret.
En riant d'un tel potin
Tous me dis'nt le verre en main :
\endverse

\beginverse
Coiffer ma femm' d'un' calotte
Je n'aurai p't'-êtr' pas raison
Surtout qu'elle port' la culotte,
Comme on dit à la maison ;
Mais j' suis né bon paysan
Et j' vas m' saouler en disant :
\endverse

\beginverse
Elle vient de mettr' au monde
Un moutard solide et beau.
Il a la peau ros' et blonde,
Moi, j' suis noir comme un corbeau ;
Mais quand j'ai vu tant d'émoi,
Je suppos' qu'il est à moi !
\endverse

\beginverse
Pour montrer que j' suis un homme
Parfois je m' fâche, emballé,
Aussitôt la gueus' m'assomme
A grands coups d' manche à balai
Et j' m'en vais clopin-clopant
A l'auberge en répétant :
\endverse

\beginverse
Quand délaissant la colombe,
Au cim'tière, je m'en irai
Point de discours sur ma tombe
Mais pourtant j'exigerai
Qu' mes bons amis d'autrefois
Vienn'nt chanter tous à plein' voix :
\endverse

\endsong
