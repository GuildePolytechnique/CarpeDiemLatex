\beginsong {Conseils d'une putain à sa fille\footnotemark}[
tu= {Tu vas quitter notre montagne},
ititle= {Conseils d'une putain à sa fille},
ititle= {Adieu, fais-toi putain},
ititle= {La grâce de dieu}]

\footnotetext {Autre titre : Adieu, fais-toi putain. \textit{Une première ersion s'intitule} Crème des vertus (\textit {dans} Le Panierau ordure, 1878) \textit {, parodie de} La grâce de Dieu. \textit {Voici reproduite la version française, donc d'origine, qui est celle contenue aussi dans le} "Petit Bitu" (1993)}

\beginverse
Tu vas quitter ta bonne mère
Pour t'en aller dans un boxon;
Je ne te retiens pas ma chère,
Si c'est là ta vocati-on.
Suis bien les conseils de ta mère
Avant toi, je fis le métier:
Tu n'as jamais connu ton père
C'était peut-être tout le quartier.
\endverse

\beginchorus
\textbf {Refrain}
Adieu, fais-toi putain,
Va-t-en gagner ton pain.
Adieu, ma fille adieu!
A la grâce de Dieu!
\endchorus

\beginverse
Evite surtout la vérole,
Chancres, poulain, \textit {et caetera},
Et ne crois jamais sur parole
Le fouteur qui te baisera.
Regarde bien si sa culotte
Cach'un vit bien entret'nu.
Découvre toujours sa calotte
Avant de lui prêter ton cul.
\endverse

\beginverse
Respecte la maquerlle,
N'offense pas le maquereau.
Tâche de te conserver belle
Et surtout n'épargne pas l'eau.
Trois par jour dans la cuvette, 
Lave ton cul bien proprement
Et dans ta table de toilette
Que l'onguent gris soit abondant.
\endverse

\beginverse
Evite bien une grossesse\footnote {Ce couplet n'apparaît pas dans la version original de la chanson. Il est tout de même repris dans la plupart des chansonniers d'étudiants; ce sera la seule raison de sa présence dans ce recueil.},
Ne te laisse pas engrosser,
En resserrant un peu les fesses
Il n'y a guère de danger.
Avec cett' chèr' capot' anglaise,
Reçois ma bénédecti-on
Et maintenant, bais' à ton aise
Et ne crais plus que les morpions.
\endverse

\endsong