\beginsong{Alphonse du gros caillou}[
  ititle={Alphonse du gros caillou}]

\beginverse
J' m'appell' Alphons', j' n'ai pas d' nom de famille,
Parc' que mon pèr' n'en avait pas non plus,
Quant à ma mèr', c'était un' pauvre fille
Qui était née de parents inconnus.
On l'appelait Thérès', pas davantage,
Quoiqu' non mariés, c'étaient d'heureux époux ;
\bisdeux{Et l'on disait : " Quel beau petit ménage,\footnote{L'air des deux derniers vers de chaque couplet est celui de \emph{Le pendu de Saint Germain} (M. : Camille Baron - P. : Maurice Mac Nab) dont l'air est utilisé pour la chanson \emph{La semaine}.}}{Que le ménage Alphons' du Gros Caillou ! "}
\endverse

\beginverse
Après trois ans, ils eur'nt enfin la chance,
Vu leur conduit', leurs bons antécédents,
D' pouvoir ouvrir un' maison d' tolérance
Et surtout cell' d'avoir eu quatr' enfants.
Sur quatr' enfants, Dieu leur donna trois filles
Qui ont servi, dès qu'ell's ont pu, chez nous ;
\bisdeux{C'est que c'était une honnête famille,}{Que la famille Alphons' du Gros Caillou !}
\endverse

\beginverse
Tout prospéra, mes soeurs aidant ma mère
Car elles eur'nt vite fait leur chemin ;
Moi-même aussi, et quelquefois mon père
S'il le fallait, nous y prêtions ... la main.
La clientèle était assez gentille
Car elle avait grande confianc' en nous ;
\bisdeux{Ils s'en allaient disant : " Quelle famille,}{Que la famille Alphons' du Gros Caillou ! "}
\endverse

\beginverse
Moi j' travaillais dans la magistrature,
Le haut clergé, les gros offici-ants,
J'avais pour ça l'appui d' la préfecture
Où je comptais aussi quelques clients
J'étais si beau qu'on m' prenait pour un' fille,
Tant j'étais tendre et caressant et doux
\bisdeux{Aussi j'étais l'orgueil de la famille,}{De la famille Alphons' du Gros Caillou !}
\endverse

\beginverse
Y'avait des jours, fallait être solide
Et le 15 août, fête de l'Empereur,
C'était chez nous tout rempli d'invalides,
De pontonniers, d' cuirassiers, d'artilleurs ;
Car ce jour-là, le militair' godille
Et tous ces gens sortaient contents d' chez nous ;
\bisdeux{Ils se disaient : " Quelle belle famille,}{Que la famille Alphons' du Gros Caillou ! "}
\endverse

\beginverse
Au dehors nous comptions quelques pratiques
Ma mèr' servait les Dam's du Sacré Coeur,
Mes soeurs servaient Madam' de Metternich,
Mon pèr' servait la Maison de l'Emp'reur.
La clientèl' était assez gentille,
Puis on avait grande confianc' en nous
\bisdeux{Et l'on disait : " Quelle sainte famille}{Que la famille Alphons' du Gros Caillou ! "}
\endverse

\beginverse
Maint'nant ma mèr' s'est r'tirée des affaires,
Moi j' continue ... mais c'est en amateur ;
Mes soeurs ont, toutes, épousé des notaires
Mon père est membr' de La Légion d'Honneur,
De notr' vertu la récompense brille
Et si notr' sort a pu fair' des jaloux,
\bisdeux{On dit, tout d' mêm' : " C'est un' belle famille,}{Que la famille Alphons' du Gros Caillou ! "}
\endverse

\endsong
