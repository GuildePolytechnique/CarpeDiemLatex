\beginsong {Auprès de ma blonde\footnotemark}[
ititle={Auprès de ma blonde}]

\footnotetext { \emph {En juillet 1643 (année à vérifier), Anne-Marie, marquise de Noirmoutier et duchesse de la Trémoille,
vit débarquer des Hollandais qui, après avoir saccagé le château de l'île, emportèrent des autochtones
comme garantie de paiement d'une rançon. Le poète local, Joubert, et parent d'un des emmenés écrivit
un ... poème : }( ...Il n'est point dans la danse, Il est bien loin d'ici. Il est dans la Hollande, Les Hollandais
l'ont pris ... ).\emph { Poème sans doute à l'origine de cette chanson.}}

\beginverse
\bis {Dans les jardins d' mon père, les lilas sont fleuris}
Tous les oiseaux du monde viennent y fair' leur nid.
\endverse

\beginchorus
\textbf{Refrain}
Auprès de ma blonde
Qu'il fait bon, fait bon, fait bon.
Auprès de ma blonde
Qu'il fait bon dormir !
\endchorus

\beginverse
\bis {Tous les oiseaux du monde viennent y fair' leur nid. }
La caill', la tourterelle, et la jolie perdrix.
\endverse

\beginverse
... Et ma jolie colombe qui chante jour et nuit.
\endverse

\beginverse
... Qui chante pour les filles qui n'ont pas de mari.
\endverse

\beginverse
... Pour moi ne chante guère car j'en ai un joli.
\endverse

\beginverse
... " Dites-nous donc, la belle, où donc est votr' mari ? "
\endverse

\beginverse
... " Il est dans la Hollande, les Hollandais l'ont pris. "
\endverse

\beginverse
... " Que donneriez-vous, la belle, pour avoir votr' ami ? "
\endverse

\beginverse
... " Je donnerais Versailles, Paris, et Saint-Denis,
\endverse

\beginverse
... Les tours de Notre-Dame, et l' clocher d' mon pays,
\endverse

\beginverse
... Et ma jolie colombe, qui chante jour et nuit ! "
\endverse

\endsong