\beginsong {Le bal au paradis\footnotemark}[
tu= {Barbari, mon ami (1648).},
ititle= {Bal au paradis, le}]

\footnotetext {Cette version, à part un ou deux vers, est celle se trouvant dans la plupart des recueils français. Une version a été publiée dans l' "Anthologie hospitalière et latinesque" (1913).}

\beginverse
Tous les ans pour le carnaval,
Jésus, par politesse,
À tous les saints offr' un grand bal
Et ceux-ci, d'allégresse
Sautent du parvis au plafond,
La faridondaine, la faridondon,
Et du plafond jusqu'au parvis, Biribi,
À la façon de Barbari, mon ami.
\endverse

\beginverse
Jésus Christ dit à Saint Crépin :
" Tu n'es qu'un vil arsouille,
Tu m'as foutu des escarpins,
Avec la peau d' tes couilles,
Cousus avec du poil de con,
La faridondaine, la faridondon,
Fous-moi le camp du paradis, Biribi
À la façon de Barbari, mon ami.
\endverse

\beginverse
Saint' Ursul', entendant cela,
S'en fut trouver Dieu l' Père.
Celui-ci la carambola,
Puis il lui dit : " Ma chère,
Saint Crépin aura son pardon, ...
Et il pourra rester ici, Biribi, ...
\endverse

\beginverse
Saint Nicolas dansait l' chahut
Avec Saint Anasthase
Et, tout en lui grattant le cul,
Disait : " Quoi qu'on en jase,
Moi, je préfèr' à tous les cons, ...
Le petit trou par où l'on chie, Biribi, ... "
\endverse

\beginverse
Saint Augustin pissant sans peur,
Le long d'une fontaine,
Sentit une énorme grosseur
Dans le repli de son aine.
C'était un colossal bubon, ...
Il avait la vérol' aussi, Biribi, ...
\endverse

\beginverse
Le Bon Dieu ayant appris
Cette bonn' aventure
Chassa de suit' du Paradis
Toutes les femm's impures.
Il en chassa trent'-six millions, ...
Qui ont ouvert bordel ici, Biribi, ...
\endverse

\beginverse
Saint Antoine, tout ébloui
Par l'éclat des bougies,
Était là, dans un coin assis,
N'aimant pas les orgies,
Il enculait son p'tit cochon, ...
Son cochon l'enculait aussi, Biribi, ...
\endverse

\beginverse
La Vierg' Marie dit à Jésus :
" Tu mènes trop la vie.
Courir ainsi de cul en cul,
T' auras des maladies,
Chaude-pisse, chancre, morpions, ...
Peut-être la vérol' aussi, Biribi, ... "
\endverse

\beginverse
Mais Jésus Christ lui répondit :
" Ne fais pas la bégueule,
Car pour toutes ces chos's aussi,
Tu peux fermer ta gueule,
Tu prêt's ton cul, tu prêt's ton con, ...
À mon cousin le Saint-Esprit, Biribi ... "
\endverse

\beginverse
Le Bon Dieu, saoul comm' un cochon,
Dormait sous une treille.
Il avait bu cinq cents flacons
Et dix-huit cents bouteilles.
Il dégueulait à gros bouillons, ...
Dans la braguett' du Saint-Esprit, Biribi, ...
\endverse

\beginverse
Saint Marc, Saint Luc, et Saint Mathieu
Sortaient d'une taverne.
Ils rencontrèrent le Bon Dieu
Qui chiait dans sa lanterne.
" Cré nom de Toi, ça n' sent pas bon, ...
Tu as le trou du cul pourri, Biribi, ... "
\endverse

\beginverse
Saint Trophim', étendu au soleil,
Gueulait de tout's ses forces :
" On n'a jamais vu chos' pareille !
La sacrée vieille rosse,
Elle m'a foutu des morpions, ...
Jusqu'aux cheveux j'en suis rempli, Biribi, ... "
\endverse

\beginverse
Le Paradis est un bordel
Où tous les saints s'enculent.
On y voit le grand Saint Michel
Enculer Sainte Ursule.
Et elle lui dit : " Ah ! que c'est bon, ...
Mais fous-y donc les couill's aussi, Biribi, ... "
\endverse

\beginverse
Quand le bal toucha à sa fin,
On éteignit les cierges.
Dans tous les coins du Paradis,
On enculait les vierges.
Le Bon Dieu enculait en rond, ...
Le Père, le Fils, le Saint-Esprit, Biribi, ...
\endverse

\beginverse
Le bal qu' eut lieu au Paradis
Fit de sacrés ravages.
Les cons sont cause que les vits
Bandent encore de rage.
Ils ont foutu chancr's et bubons, ...
Et la vérole aussi, Biribi, ...
\endverse

\beginverse
Puisque c'est Dieu qui nous remit
La Très Sainte Vérole,
Eh bien, eh bien, mes chers amis,
Il faut qu'on s'en console.
Et crions tous à pleins poumons : ...
Je voudrais qu'il l'attrap' aussi, Biribi, ...
\endverse

\beginverse
Vous jugerez avec raison
Ma chanson un peu leste.
Des bals, c'est pourtant la façon
Dans l'empire céleste.
Vous trouverez cela fort bon, ...
Quand vous serez au Paradis, Biribi, ...
\endverse

\endsong