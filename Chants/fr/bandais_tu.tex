\beginsong{Bandais-tu?\footnotemark}[
  ititle={Bandais-tu?},
  ititle={Bel Alcyndor, Le},
  tu={Malheur à celui qui blesse un enfant (Enrico Macias)}]

\footnotetext{Autre titre : \emph{Le bel Alcyndor}. \emph{Alcyndor} fait sans doute référence à Louis XIV, le Roi-Soleil, dont les faveurs étaient partagées en particulier par Marie-Angélique de Fontange. On retrouve d'ailleurs dans le refrain original le prénom \emph{Angèle}, ce qui pourrait confirmer que \emph{Alcyndor} et Louis XIV ne font qu'un, et que l'air daterait du XVIIème siècle.}
  
\beginverse
Si tous les pavés étaient des biroutes
On verrait les femm's s' coucher sur les routes.
\endverse

\beginchorus
\textbf{Refrain}
Bandais-tu, ban- ban- ban-, bandais-tu fort
Quand tu pelotais les nichons d'Adèle ?
Bandais-tu, ban- ban- ban- bandais-tu fort
Quand tu tripotais tous ces divins trésors ?
\endchorus

\beginverse
Si les cons poussaient comm' des pomm's de terre
On verrait les pin's labourer la terre.
\endverse

\beginverse
Si tous les curés n'avaient plus de verges
On verrait les nonn's employer des cierges.
\endverse

\beginverse
Si les cons nageaient comme des grenouilles
On verrait flotter plus d'un' pair' de couilles.
\endverse

\beginverse
Si les cons volaient comme des bécasses
On verrait les pin's partir à la chasse.
\endverse

\beginverse
Si tout's les putains étaient lumineuses
La terr' ne serait qu'une immens' veilleuse.
\endverse

\beginverse
Si tous les cocus avaient des clochettes
On n' s'entendrait plus sur notre planète.
\endverse

\beginverse
Si les cons nichaient comm' des hirondelles
On verrait les vits monter à l'échelle.
\endverse

\beginverse
Si les cons pissaient de l'encre de chine
On verrait s'y tremper toutes les pines.
\endverse

\beginverse
Si les cons savaient l' théorème de Rolle
On verrait les vits leur poser des colles.
\endverse

\beginverse
Si les cons dansaient comm' des ballerines
On verrait les log's se garnir de pines.
\endverse

\endsong
