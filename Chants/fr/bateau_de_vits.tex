\beginsong{Le bateau de vits\footnotemark}[
  ititle={Bateau de vits, Le}]

\footnotetext{Auteur : François Chevigny de la Bretonnière (XVIIème siècle).}
  
\beginverse
Un bateau chargé de vits
Descendait une rivière
Ils étaient si bien raidis
Qu'ils passaient par la portière.
\endverse

\beginchorus
Refrain
Pan, pan, de la Bretonnière
Pan, pan, de la barbe au con.
\endchorus

\beginverse
Ils étaient si bien raidis
Qu'ils passaient par la portière
Une dame de Paris
Envoya sa chambrière
\endverse

\beginverse
... Au bateau chargé de vits
Lui choisir la plus bell' paire
\endverse

\beginverse
... La servante, en femm' d'esprit,
S'en est servi la première
\endverse

\beginverse
... Elle s'en est si bien servie
Qu'elle s'est pété la charnière
\endverse

\beginverse
... Et, du cul jusqu'au nombril,
Ce n'est plus qu'un vaste ornière
\endverse

\beginverse
... Les morpions nagent dedans
Comme poissons en rivière
\endverse

\beginverse
... On croit baiser par-devant
Va t' fair' foutre, c'est par-derrière !
\endverse

\beginverse
... On croit lui faire un enfant
On ne lui donn' qu'un clystère
\endverse

\beginverse
... On croit être son amant
On n'est qu' son apothicaire
\endverse

\beginverse
... On croit l'aimer tendrement
La marchandis' tomb' par terre
\endverse

\beginverse
... " Ah ! Dit-elle en l'écrasant
Ç'ui-là n' battra pas son père.
\endverse

\beginverse
... Et tu n'écorcheras pas\footnote{Couplet apocryphe.}
Le joli con de ta mère. "
\endverse

\endsong
