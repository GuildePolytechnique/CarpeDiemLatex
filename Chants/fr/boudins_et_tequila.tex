\beginsong {Boudins et téquila \footnotemark}[
ititle={ Boudins et téquila},
tu= {Vive la rose (interprétée par Guy Béart}]

\footnotetext {Guilde Polytechnique, ULB ; Festival de la chanson estudiantine CP ULB, 1992.}

\beginverse
Partis entre copains
Pour une noble cause
Direction le Gauguin
Je ne sais pas si j'ose
\bisdeux {Le foie ne tiendra pas } {Viv' la cirrhose, la gueule de bois !}
\endverse

\beginverse
Un' fois sur le terrain
Un p'tit "À-fond" s'impose
Avec un verr' en main
C'est déjà moins morose
\bisdeux {Le foie ne tiendra pas} {Viv' la cirrhose, la gueule de bois !}
\endverse

\beginverse
Le lendemain matin
Aïe ! Aïe ! Ma têt' explose
Je n' me souviens de rien
Ne cherchons pas la cause
\bisdeux {Le lavabo est plein} {J'ai r'tapissé la sall' de bain !}
\endverse

\beginverse
Mais sous mon traversin
Ça ne sent pas la rose
Y a-t-il donc quelqu'un
Infecté de mycoses ?
\bisdeux {Ne cherchons pas plus loin} {J'ai encore ram'né un boudin !}
\endverse

\beginverse
Et si un bon matin
Un' occasion s'arrose
Laissez-là le brassin
Buvez donc autre chose
\bisdeux {Frappez la Tequila!} {Vous courez à votre trépas !}
\endverse

\beginverse
Mêm' si on en revient
De ces orgies grandioses
Avec un intestin
Qui se métamorphose
On les regrettera 
La cirrhose et la Tequila.
On les regrettera
La cirrhose et la gueule de bois.
\endverse

\endsong
