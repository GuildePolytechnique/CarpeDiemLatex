\beginsong {La Brabançonne d'une putain} [
ititle= {Brabançonne d'une putain},
tu= {La Brabançonne (P. : Charles Rogier - M. : Frans Van Campenhout)}]

\beginverse
Je me souviens lorsque j'étais jeune fille,
D'un jeun' garçon qui passait par bonheu-heur.
Il me trouva si jeun' et si gentille
Qu'il me fit voir sa gross' pin' en chaleur,
Et tout à coup, sous mes jupons s'élance,
L'énorme queue qu'il tenait à la main,
Il déchira mon voile d'innocence
\ter {Voilà pourquoi je me suis fait putain !}
\endverse

\beginverse
Je ne sais pas si j'étais déjà coquine,
J'aimais déjà qu'on m' chatouillât l' bouton :
J'avais goûté de ce bon jus de pine,
J'avais reçu du foutre dans le con.
J'avais baisé, je n'étais plus pucelle,
Je chérissais le métier de putain ;
Plus je baisais, plus je devenais belle
\ter {Voilà pourquoi je me suis fait putain !}
\endverse

\beginverse
Quoique je ne sois qu'une fille publique,
J'ai de l'amour et de l'humanité.
Tout citoyen de notr' libre Belgique
Doit baiser et jou-ir en liberté.
Pour de l'argent le riche a ma fente,
Le pauvre, lui, peut en jou-ir pour rien :
Pour soulager l'humanité souffrante,
\ter {Voilà pourquoi je me suis fait putain !}
\endverse

\endsong