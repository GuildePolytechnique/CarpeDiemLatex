\beginsong {La Branleuse de taureaux \footnotemark}[
ititle= {Branleuse de taureaux, La}]

\beginchorus
\textbf {Premier refrain}
C'est la branleuse de taureaux
Qui va, qui vient,
Qui fait son ouvrage ;
C'est la branleuse de taureaux
Qui va, qui vient,
Toujours au boulot.
\endchorus

\beginverse
Dans une ferme modèle,
Depuis qu'elle n'est plus pucelle,
Elle titille avec passion
Pour fair' l'insémination.
C'est elle qui tire la liqueur
À ses bons reproducteurs
Qui ont le gland aussi gros qu'un clocher
Et les claouis comm' des fesses ;
Si en suçant, elle aval' la fumée,
Elle est nourrie pour l'année.
\endverse

\beginchorus
\textbf {Premier refrain
+
Deuxième refrain}
Pomper la s'menc' à ses bestiaux,
C'est pas très sain, qu'elle a du courage...
Faut d' l'expérience et du brio :
\bis {Elle a la main, la branleus' de taureaux.}
\endchorus

\beginverse
Pour arrondir ses fins d' mois,
Elle va tapiner au bois ;
Sa petit' spécialité
Lui assur' des habitués.
On vient la voir de très loin
Avec la pin' à la main,
Mais elle se marre devant les vits bandés
Sous l'effet de ses caresses ;
Quand elle compare avec ses bovidés,
C'est des cur'-dents pour pygmées.
\endverse

\beginchorus
Premier refrain + Deuxième refrain
\endchorus

\endsong