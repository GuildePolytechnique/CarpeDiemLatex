\beginsong {C'était au temps où Bruxelles guindaillait \footnotemark}[
ititle= {C'était au temps où Bruxelles guindaillait},
tu= {Bruxelles bruxellait (P. Jouannest, interprétée par Jacques Brel)}]

\beginchorus
\textbf {Refrain}
C'était au temps où Bruxelles guindaillait
C'était au temps où les students buvaient !
C'était au temps où Bruxelles se marrait
C'était au temps où les students chantaient !
\endchorus

\beginverse
Place de Brouckère on bouffait des marrons
On dégueulait tell'ment on était ronds.
En ce temps-là on avait la vérole
On n'en bouffait pas moins des caricoles.

Et plac' Saint'-Cath'rine
On montrait nos pines
Et aussi nos fesses
Après la grand' messe
Et le vieux vicaire
Ne sachant que faire
Nous engueulait, on s'en foutait
Et on faisait c' qui nous plaisait.
\endverse

\beginverse
Au Grand Sablon démarrait la St V
On y voyait des pennes par milliers.
A la Grand' Place, on était tous bourrés
A l' "Amigo", les flics nous ont emm'nés

Et rue de l'Etuve
Dans sa petit' cuve
Y'avait Manneken pis
Qu' entret'nait sa chaud'-pisse
Souvenir d'une Ibère
Qui s'était laissée faire
Des petits seins, un gros vagin
Il s'en foutait, elle baisait bien.
\endverse

\beginverse
A la Bourse on s'arrêtait pour chanter
"Le Semeur", en choeur était entonné.
Puis tous ensemble on r'gagnait l'ULB
Où la soirée n' faisait que commencer.

A la Mort Subite
On s' foutait un' cuite
En buvant de la Kriek
Et aussi du Lambic,
Et chaussée d' Boondael(e)
On s' rinçait la dalle
Puis au Villon, là chez Simon
On n'arrêtait pas d' fair' les cons.
\endverse

\beginchorus
\textbf {Dernier refrain}
C'était au temps où Bruxelles guindaillait
C'était au temps où les students buvaient !
C'était au temps où Bruxelles se marrait,
C'était au temps ou le folklore vivait !
\endchorus

\endsong