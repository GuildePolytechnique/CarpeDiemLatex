\beginsong {Les Calfats \footnotemark} [
ititle= {Calfats, Les}]

\footnotetext {On y parle des conditions de la corporation des calfats, mal considérée à l'époque par les matelots.
Cette chanson évoque la fin des bateaux en bois, vers 1870-1880, et la naissance de l'ère des bateaux en
fer. Les paroles seraient de Soclet (Source : Chants de marins traditionnels - Sélection de l'Anthologie
des chansons de mer / Volumes I à V - page 6 - SCM 014).}

\beginverse
Quand un bateau entr' en carène\footnote {Carène (génois caréna, latin carina : coquille de noix, 1246) : partie immergée de la coque d'un bateau. Caréner (1642) : nettoyer une carène ou la réparer. (in Larousse, Dictionnaire de la langue
française, Lexis, 1992)}
Comm' c'lui-là qu' vous voyez là-bas
On n' voit pas l' mal et tout' la peine
Que s' donnent ceux qui sont sur les ras\footnote{ Ras (latin ratis : radeau, 1630) : plate-forme flottante, servant aux réparations d'un navire, près de la flottaison. (in Larousse, Dictionnaire de la langue française, Lexis, 1992)}
Dans l'étoupe en plein goudronnage
Vous voyez bien ce tas d' margas
C'est ma bordée, mon équipage
C'est tous calfats, c'est tous calfats !
\endverse

\beginverse
On trouv' partout des ministres
Des sénateurs, des députés
Des charpentiers des ébenistes
Et mêm' des douaniers retraités
On trouve des femmes de ménage
Des nourric's et puis des soldats
Mais c' qu'on trouv' plus, ça c'est dommage
C'est des calfats, c'est des calfats !
\endverse

\beginverse
Je le jure sur la pigouillère
Que j'avions tant d' turbins dans l' temps
Que j'ai vu ma bordée entière
Tous les jours en cracher le sang
Mais à présent, sur ma parole
Adieu maillets et pataras\footnote {Pataras (germ. paita : morceau d'étoffe, 1687) : outil de calfat servant à ouvrir les coutures des
bordages pour y introduire l'étoupe. (in Larousse, Dictionnaire de la langue française, Lexis, 1992)} !
Avec tout's leurs sacrées castroles
Y'a plus d' calfats, y'a plus d' calfats !
\endverse

\beginverse
Maintenant qu' la tôl' fait l' bordage
Y'a plus moyen de faire ses frais
On a supprimé l' calfatage
Ah ! qu' c'est du propr' que leur progrès
Quoi qu' nos fils f'ront de leur carrière
Des ingénieurs ? Des avocats ?
Autant brûler la pigouillère
Faut plus d' calfats, faut plus d' calfats !
\endverse

\endsong