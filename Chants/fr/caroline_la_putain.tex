\beginsong {Caroline, la Putain \footnotemark} [
ititle= {Caroline, la putain},
ititle= {Caroline},
tu= {Ton ton, tontaine, ton ton (M. : Air de cor , P. : Marion de Mersan, 1770).}]

\footnotetext {Autre titre : \emph{Caroline}.}

\beginverse
Amis, amis, versez à boire,
Versez à boir' et du bon vin,
Tintin, tintin, tintain' et tintin.
Je m'en vais vous conter l'histoire
De Caroline, la putain
Tintin, tintain' et tintin.
\endverse

\beginverse
Son pèr' était un machiniste
Au théâtre de l'Odéon ...
Sa mèr' était une fleuriste
Qui vendait sa fleur en bouton ...
\endverse

\beginverse
Elle perdit son pucelage
Le jour d' sa premièr' communion, ...
Avec un garçon de son âge
Derrièr' les fortifications ...
\endverse

\beginverse
À quatorz' ans, suçant les pines,
Elle fit son éducation, ...
À dix-huit ans, dans la débine,
Elle s'engagea dans un boxon ...
\endverse

\beginverse
À vingt-quatr' ans, sur ma parole,
C'était une fière putain, ...
Elle avait foutu la vérole
Au trois quarts du Quartier Latin ...
\endverse

\beginverse
Le marquis de la Couillemolle
Lui fit bâtir une maison, ...
À l'enseign' du "Morpion qui Vole",
Une bell'\footnote {Variante: \emph{Quell' chouett'}} enseign' pour un boxon ...
\endverse

\beginverse
Elle voulut aller à Rome
Pour recevoir l'absolution ...
Le pape était fort bien à Rome,
Mais il était dans un boxon ...
\endverse

\beginverse
Et s'adressant au grand vicaire,
Elle dit : " J'ai trop prêté mon con ... "
" Si tu l'as tant prêté, ma chère,
À moi aussi, prête-le donc ... "
\endverse

\beginverse
En la serrant entre ses cuisses,
Il lui donna l'absolution, ...
Il attrapa la chaude-pisse
Et trent'-six douzain's de morpions ...
\endverse

\beginverse
Elle finit cette tourmente
Entre les bras d'un marmiton ...
Elle mourut la pin' au ventre
Le con fendu jusqu'au menton ...
\endverse

\beginverse
Et quand on la mit dans la bière,
On vit pleurer tous ses morpions, ...
Et quand on la mit dans la terre
Ils entonnèr'nt cette chanson\footnote{Variante : \emph{Ils s'arrachèrent les poils du con ...}} ...
\endverse

\endsong
