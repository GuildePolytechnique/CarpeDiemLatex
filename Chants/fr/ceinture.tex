\beginsong {La ceinture} [
ititle= {Ceinture, la}]

\beginverse
Partant pour la croisade, un Sire fort jaloux
De l'honneur de son nom et de son droit d'époux
Fit fair' une ceintur' à solide fermoir
Qu'il attacha lui-mêm' à sa femm' un beau soir.
\endverse

\beginchorus
\textbf{Refrain}
\bis {Tra la la la la lère, tra la la la la la}
\endchorus

\beginverse
Une fois son honneur solidement bouclé,
Le Sire s'en alla en emportant la clef
Depuis la tendr' Yseult soupire nuit et jour :
" Quand donc t'ouvriras-tu, prison de mes amours ? "
\endverse

\beginverse
Elle fit la rencontre le soir au fond d'un bois,
D'un jeune troubadour, poète montmartrois,
Elle lui demanda gentiment d'essayer
Si d'un poèt' l'amour peut fair' un serrurier.
\endverse

\beginverse
Elle était désirable et belle tant et tant,
Que le fermoir céda et qu'elle en fit autant.
Depuis bientôt deux ans durait leur tendr' amour,
Quand le seigneur revint avec corn's et tambours.
\endverse

\beginverse
La bell' étant enceinte depuis bientôt neuf mois,
S'écria : " Sur ma vie, quel malheur j'entrevois,
En mettant la ceintur' et la serrant un peu
Notre seigneur jaloux n'y verra que du feu. "
\endverse

\beginverse
Le Sir' s'en aperçut et se mit en courroux,
" Seigneur, s'écria-t-elle, cet enfant est de vous !
Depuis votre départ, votre fils enfermé
Attend votre retour pour être délivré. "
\endverse

\beginverse
" Miracle, cria-t-il, femm' au con vertueux,
Ouvrons vite la porte au fils respectueux ! "
De joie, la tendr' Yseult, à ces mots, enfantait
Et depuis, la ceintur', c'est lui qui s' la mettait.\endverse

\endsong