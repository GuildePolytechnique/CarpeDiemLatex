\beginsong{Les cent louis d'or\footnotemark}[
  ititle={Cent louis d'or, Les},
  ititle={Amour en diligence, L'},
  ititle={Louis d'or, Les}]

\footnotetext{Autres titres : \emph{Les louis d'or} (milieu du XIXème), première version dont l'auteur n'est autre que le poète et chansonnier Pierre Dupont, \emph{Parodie des louis d'or de Pierre Dupont}, \emph{L'amour en diligence}}

\beginverse
Un soir, étant en diligence,
Sur une route entre deux bois,
Je branlais avec assurance
Une fillett' au frais minois.
J'avais retroussé sa chemise
Et mis mon doigt sur son bouton.
Et je bandais malgré la bise,
À déchirer mon pantalon.
Pour un quart d'heur' entre ses cuisses.
Un prince eût donné un trésor,
Et moi j'aurais, Dieu me bénisse,
J'aurais donné cent louis d'or !
\endverse

\beginverse
La de branler sans résistance,
La tête en feu, la pine aussi,
Je pris sa main, quell' indécence !
Et la mis en forme d'étui.
Je jou-issais à perdr' haleine,
Je déchargeai, quel embarras !
Sa main, sa rob' en étaient pleines,
Et cela ne suffisait pas.
Sentant rallumer ma fournaise,
Je lui dis : "Tiens, fais plus encore,
Sortons d'ici que je te baise
Je te donne cent louis d'or !"
\endverse

\beginverse
La belle alors, toute confuse,
Me répondit ingénument :
"Pardon, monsieur, si je refuse
Ce que vous m'offrez galamment,
Mais j'ai juré de rester sage
Pour mon fiancé, pour mon mari,
De conserver mon pucelage,
Il ne sera jamais qu'à lui."
"Tu n'auras pas le ridicule,
Dis-je, d'arrêter mon essor,
Permets au moins que je t'encule,
Je te promets cent louis d'or !.
\endverse

\beginverse
Au premier relais sur la route,
Nous descendîmes promptement.
"Au cul, il faut que je te foute,
Ne pouvant te foutre autrement."
Dans une auberge, nous entrâmes,
Tout s'y trouvait : bon feu, bon lit.
Brûlants d'amour, nous nous couchâmes :
Je l'enculai toute la nuit.
Mais pour changer de jou-issance
Je lui dis : "Tiens, fais plus encor',
Livre ton con et tout d'avance,
Je te promets cent louis d'or !"
\endverse

\beginverse
"Je veux bien, sans plus de harangue,
Dit-elle en me suçant le gland,
Livrer mon con à votre langue,
Pour ne pas trahir mon serment."
Aussitôt, placés tête-bêche,
Comme deux amants dans le lit,
Avec ardeur, moi, je la lèche,
Pendant qu'ell' me suce le vit.
Mais la voyant bientôt pâmée,
Je pus lui ravir son trésor,
Et je me dis, la pine entrée :
"Je gagne mes cent louis d'or !"
\endverse

\beginverse
Huit jours après cette aventure,
J'étais de retour à Paris.
Ne prenant plus de nourriture,
Restant tout pensif au logis.
À la gorg', ainsi qu'à la pine,
J'avais, c'était inqui-étant,
Chancre, bubons et, on l'devine,
La chaude-pisse, en même temps,
Prenant le parti le plus sage,
Je me transportai chez Ricord,
Qui me dit : "Un tel pucelage,
Vous coûtera cent louis d'or !"
\endverse

\endsong
