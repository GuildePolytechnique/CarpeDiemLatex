\beginsong {La Chanson du Roi Albert\footnotemark} [
ititle={Chanson du Roi Albert, La},
ititle={Soldat belge, Le},
tu={La Sentinelle du Pont Henri IV}]

\footnotetext {Autres titres: Le Soldat belge. La chanson a paru en 1918 dans le quotidien \emph{Le droit des peuples}. La version présenté dans ce recueil est la version actuelle qui a été peaufinée. L'histoire se serait vraiment passé: Jules Jacob, le milicien, aurait donc été au poste frontière de Zelzate entre la Hollande et la Belgique et aurait reçu deux médailles, dont une pour "\emph{n'avoir laissé passer personne, pas même le roi}". Il est enterré à Jandrain.}

\beginverse
C'était un soir sur les bords de l'Yser(e)
Un soldat belg' qui montait la faction
Vinr'nt à passer trois braves militaires
Parmi lesquels se trouvait le Roi Albert.
" Qui vive-là, cria la sentinelle,
Qui vive-là, vous ne passerez pas ;
Si vous passez, craignez ma baïonnette,
\bis {Retirez-vous, vous ne passerez pas} ... Halte là ! "
\endverse

\beginverse
Le Roi Albert mit la main à la poche :
" Tiens, lui dit-il, et laisse-nous passer "
" Non, répondit la brave sentinelle
L'argent n'est rien pour un vrai soldat belg'.
Dans mon pays, je cultivais la terre,
Dans mon pays, je gardais les moutons ;
Mais maintenant que je suis militaire,
\bis {Retirez-vous, vous ne passerez pas} ... Halte là ! "
\endverse

\beginverse
Le Roi Albert dit à son capitaine :
" Fusillons-le, c'est un mauvais sujet.
Fusillons-le, passons-le par les armes.
Fusillons-le, et puis nous passerons. "
" Fusillez-moi, cria la sentinelle,
Fusillez-moi vous ne passerez pas,
Si vous passez, craignez ma baïonnette,
\bis {Retirez-vous, vous ne passerez pas} ... Halte là ! "
\endverse

\beginverse
Le lendemain, au grand conseil de guerre.
Le Roi Albert l'appela par son nom : " Hé, Julot !
Tiens, lui dit-il, voici la croix de guerre,
La croix de guerre et la décoration. "
" Ah, que dira ma douce et tendre mère,
En me voyant tout couvert de lauriers ;
La croix de guerr' pend à ma boutonnière,
\bis {Pour avoir dit : Vous ne passerez pas,} ... Halte là ! "
\endverse

\endsong