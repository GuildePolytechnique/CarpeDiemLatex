\beginsong {Conseils d'anciens\footnotemark} [
ititle= {Conseils d'anciens},
tu= {Donne du rhum à ton homme (G. Moustaki)}]

\footnotetext {1er prix du XXIIème festival de la chanson estudiantine CP ULB 1996 (Guilde Polytechnique)}

\beginchorus
\textbf {Refrain}
Donne des chopes à ton bleu,
De la clache et des oeufs.
Donne des chopes à ton bleu,
Et tu verras comme il sera joyeux.
\endchorus

\beginverse
Y a des gens dont le sort
Est d'étudier sans cesse,
Communier dans l'effort
Et vivre dans le stress.
Mais ton bleu n'est pas de ceux-là,
Tu le regardes d'un air tendre.
Si tu veux le garder pour toi,
Donne, donne lui sans attendre.
\endverse

\beginverse
Quand aux activités,
Ton bleu hésite et tremble.
Quand il est fatigué,
Qu'il ne veut plus apprendre.
Fais lui faire une dizaines d'à-fonds
Qu'il reprenne du courage.
Puis arrache-lui le caleçon,
Qu'il reparte à l'abordage.
\endverse

\beginverse
Dans les cercles tu voudras
Qu'il entonne à tue-tête,
Son chant qu'il n' retient pas;
Et sans cesse tu répètes
Qu'il va perdre tous ses cheveux,
Tu t'énerves, il devrait faire mieux.
Il doit toujours baisser les yeux,
La bleusaille c'est très très sérieux.
\endverse

\beginverse
Quel baptême que c'lui-là,
On en parle dans la ville,
Même qu'on exagérera
Le sadisme des débiles.
Mais pour l'heure il est baptisé,
Il digère sa renaissance
Dès que tu l'auras réveillé
Si tu veux que ça recommence.
\endverse

\beginchorus
Donne des chopes à ton bleu
Du savon et de l'eau
Donne des chopes à ton bleu
Et tu verras comme il sera beau.
\endchorus

\beginverse
Après le 20 novembre,
Il part sans crier gare
S'enfermer dans sa chambre
Pour refaire son retard.
Au moment de vous séparer,
Pour des mois, des longues semaines,
Rappelle-lui les T.D. ...
\bis {Mais si tu veux qu'il te revienne}
\endverse

\beginchorus
Donne des chopes à ton poil
De la clache et des BLEUS
Donne des chopes à ton poil
Et tu verras comme il sera heureux.
\endchorus

\endsong