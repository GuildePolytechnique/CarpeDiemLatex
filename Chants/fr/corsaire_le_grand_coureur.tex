\beginsong {Le corsaire Le Grand Coureur\footnotemark} [
ititle= {Corsaire Le Grand Coureur (Le)}]

\footnotetext {Chanson à virer popularisée en 1927 par le commandant Hayet. Le thème daterait de l'époque des guerres de l'Empire français contre les Anglais (Source : Chants de marins traditionnels - Sélection de l'Anthologie des chansons de mer / Volumes I à V - page 8 - SCM 014).}

\beginverse
Le corsaire Le Grand Coureur
Est un navire de malheur
Quand il se met en croisière
Pour aller battre l'Anglais,
Le vent, la mer et la guerre
Tournent contre le Français !
\endverse

\beginchorus
\textbf {Refrain}\footnote {Certaines versions de cette chanson bisent le refrain.}
Allons les gars, gai, gai !
Allons les gars, gaiement !
\endchorus

\beginverse
Il est parti de Lorient
Avec bell' mer et bon vent
Il cinglait bâbord amure\footnote {Amure (prov. amura : cordage, 1552) : cordage qui retient le coin inférieur d'une voile du côté d'où vient le vent. Amurer (1540) : raidir l'amure d'une voile. (in Larousse, Dictionnaire de la langue française, Lexis, 1992)}
Naviguant comme un poisson ;
Un grain tomb' sur la mâture,
V'là le corsaire en ponton !
\endverse

\beginverse
Il nous fallut remâter
Et diablement bourlinguer
Tandis que l'ouvrage avance
On aperçut par tribord
Un navire d'apparence
à mantelets\footnote {Mantelet (1138) : volet à rabattement, fermant un sabord [(1402) : ouverture pratiquée dans la muraille d'un navire et servant soit de passage à la souche des canons, soit d'orifice d'aération]. (in Larousse, Dictionnaire de la langue française, Lexis, 1992)} de sabord !
\endverse

\beginverse
C'était un Anglais vraiment
À double rangée de dents
Un marchand de mort subite,
Mais le Français n'a pas peur ;
Au lieu de prendre la fuite
Nous le rangeons à l'honneur !
\endverse

\beginverse
Ses boulets sifflent sur nous;
Nous lui rendons coup pour coup,
Tandis que la barb' en fume
À nos brave matelots
Nous voilà pris dans la brume
Nous échappons aussitôt !
\endverse

\beginverse
Pour nous refair' des combats,
Nous avions à nos repas,
Des gourgan's et du lard rance,
Du vinaigr' au lieu de vin,
Le biscuit pourri d'avance
Et du camphre le matin !
\endverse

\beginverse
Nos pris's au bout de six mois
Ont pu se monter à trois :
Un navir' plein de patates
Plus qu'à moitié chaviré,
Un autre plein de savates,
Un troisième de fumier !
\endverse

\beginverse
Pour finir ce triste sort,
Nous venons périr au port
Dans cett' affreuse misère,
Quand chacun s'est cru perdu,
Chacun, selon sa manière
S'est sauvé comme il a pu !
\endverse

\beginverse
Le cap'tain' et son second
S' sont sauvés sur un canon ;
Le maître sur la grand' ancre ;
Le commis sur son bidon.
Oh ! le trist' et vilain congre,
Le voleur de rati-on !
\endverse

\beginverse
Il eut fallu voir le coq(e)
Avec sa cuiller 't son croc.
Il s'est mis dans sa chaudière
Comme un vilain pot-au-feu.
Il a couru vent arrière,
Il a pris terr' à l'îl'-D'Yeu1\footnote {Île vendéenne où Pétain fut détenu de 1942 jusqu'à sa mort (1951).} !
\endverse

\beginverse
De notr' horrible malheur,
Le calfat\footnote {Calfat (grec kalaphates, 1371) : ouvrier qui calfate les navires. Calfater (1200) : remplir à force avec de l'étoupe (partie la plus grossière de la filasse de chanvre ou de lin) les fentes de la coque d'un navire pour le rendre étanche. (in Larousse, Dictionnaire de la langue française, Lexis, 1992)} seul est l'auteur.
En tombant de la grand' hune
Dessus le gaillard d'avant,
A rebondi dans la pompe,
Défoncé le batiment !
\endverse

\beginverse
Si l'histoire du Grand Coureur
A pu vous toucher le cœur,
Ayez donc bell's manières
Et payez-nous largement,
Du vin, du rack, de la bière
Et nous serons tous contents !
\endverse

\endsong