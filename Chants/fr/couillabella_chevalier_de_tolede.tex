\beginsong {Couillabella, chevalier de Tolède} [
ititle= {Couillabella, chevalier de Tolède},
tu= {Gastibelza (P. : Victor Hugo, 1840 - M. : Georges Brassens)}]

\beginverse
Couillabella, l'homm' à la longue pine
Parlait ainsi :
Qui donc de vous a-t-il connu Sabine ?
Malheur à lui !
Car elle avait, je vous donn' ma parole
Mal au vagin
Et la sal' bêt' m'a foutu la vérole
Dans le bassin.
\endverse

\beginverse
Vénus près d'elle aurait paru bien laide
Lorsqu'un beau soir
Je l'aperçus sous les murs de Tolède
Faisant l' boul'vard.
Elle avait les beaux yeux d'une gazelle
De gros tétons,
Et je bandais en la voyant si belle
Comm' un cochon.
\endverse

\beginverse
Je ne sais pas si j'eus son pucelage
Mais je sais bien
Que mon canal me fit contre l'usage
Un mal de chien.
Et depuis lors je n'ai cessé de prendre
Du copahu\footnote {Copahu (mot tupu-guarani du Brésil ; 1578) : sécrétion oléorésineuse du copayer, autrefois utilisée en médecine. (in Larousse, Dictionnaire de la langue française, Lexis, 1992)}
Et à présent je ne suis plus qu'un chancre
Du ventr' au cul.
\endverse

\endsong