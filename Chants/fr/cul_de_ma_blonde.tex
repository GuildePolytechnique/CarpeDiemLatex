\beginsong{Le cul de ma blonde\footnotemark}[
  ititle={Cul de ma blonde, Le},
  ititle={Blonde, Ma},
  tu={La nature (Gaveaux)}]

\footnotetext{Autre titre : \emph{Ma blonde}. L'auteur est Paul-Émile Debraux, notamment auteur de Fanfan la Tulipe. On en trouve une version en 7 couplets dans les "Gaudrioles du XIXème siècle" où le dervî est remplacé par un rouchis. On trouve le texte original dans "Le Nouveau Parnasse Satyrique du XIXème siècle".}
  
\beginverse
J'ai tâté du vin d'Argenteuil
Et ce vin m'a foutu la foire
J'ai voulu tâter de la gloire
Une balle m'a crevé l'oeil
Des catins du grand monde
J'ai tâté la vertu
Des splendeurs, revenu,
Je veux tâter le cul
\bis{De ma blonde}
\bisdeux{Des splendeurs, revenu,}{\bis{Je veux tâter le cul}}
\bis{De ma blonde}
\endverse

\beginverse
Preux guerriers, vaillants conquérants,
Fi de la gloire qui vous éclope
Votre maîtress' est une salope
Qui vous pince en vous caressant !
Empoignez-moi la ronde,
Et la lanc' et l'écu
De peur d'être cocu
Moi j'empoigne le cul ...
\endverse

\beginverse
Y'a des gens qui font la grimace
Quand ils voient monsieur le curé
Qui promène dans une châsse
Un Bon Dieu en cuivre doré.
Ce bon curé se trompe\footnote{Originale : \emph{Ce système qu'on fronde Serait bien mieux reçu.}}
Il serait mieux venu
Si, foutant là Jésus,
Il promenait le cul ...
\endverse

\beginverse
" Mon fils, me dit un vieux dervî,
Souffrez qu'on vous le dise
A baiser sans permis d'Église
Vous perdez le saint Paradis. "
" Vous foutez-vous du monde ?
Dis-j' à ce noir cocu,
Le Paradis perdu
Vaut-il un poil du cul "...
\endverse

\beginverse
Puisqu'ici bas, l'homme jeté
Doit mourir comm' une victime,
Je me fous d'un trépas sublime,
J'emmerde l'immortalité !
Puissé-j' en passant l'onde
Du fleuve au dieu cornu,
Godiller ferm' et dru,
Et mourir dans le cul ...
\endverse

\endsong
