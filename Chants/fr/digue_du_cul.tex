\beginsong {La digue du cul\footnotemark} [
ititle= {Digue du cul (la)},
ititle= {En revenant de Nantes}]

\footnotetext {Autre titre : \emph{En revenant de Nantes}. Lève la jambe sert souvent de refrain à cette chanson.} 

\beginverse
\bis{La digue du cul, en revenant de Nantes}
De Nantes à Montaigu,
La digue, la digue,
De Nantes à Montaigu,
La digue du cul.
\endverse

\beginverse
\bis {La digue du cul, je rencontre une belle}
Qui dormait le cul nu,
La digue, la digue,
Qui dormait le cul nu,
La digue du cul.
\endverse

\beginverse
\bis {La digue du cul, je band' mon arbalète}
Et la lui fout dans l' cul, ...
\endverse

\beginverse
\bis {La digue du cul, la belle se réveille}
Et dit : " J'ai l' diable au cul ! " ...
\endverse

\beginverse
\bis {La digue du cul, non, ce n'est pas le diable}
Mais un gros dard velu, ...
\endverse

\beginverse
\bis {La digue du cul, qui bande et qui décharge}
Et qui t'en fout plein l' cul ...
\endverse

\beginverse
\bis {La digue du cul, puisqu'il y'est qu'il y reste}
Et qu'il n'en sorte plus, ...
\endverse

\beginverse
\bis {La digue du cul, il fallut bien qu'il sorte}
Il est entré bien raide
La digue, la digue,
Il en sortit menu
La digue du cul.
\endverse

\endsong