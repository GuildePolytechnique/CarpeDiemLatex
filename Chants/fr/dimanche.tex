\beginsong {Un dimanche} [
ititle= {Dimanche (un)},
tu= {La petite Tonkinoise (Joséphine Baker)}]

\beginverse
Un dimanche
Sous les branches
Le soleil était radieux.
Je partis pour la Bohême
Le seul pays où l'on s'aime.
Une Anglaise
Aux yeux d' braise
Se prom'nait flegmatiqu'ment.
Je lui dis en souriant :
Veux-tu que j' sois ton amant ?
\endverse

\beginverse
Je te bais'rai en levrette,
Soit sur le lit, la tabl' de nuit, dans la cuvette,
Soit debout, soit sur un' chaise
Nous nous bais'rons à notr' aise.
Je te ferai ma poulette,
Feuille de ros', 69, ou bien minette,
Je te pelot'rai les seins
Pour me fair' dresser l' marsouin.
\endverse

\beginverse
La gamine
Très caline
Accepta avec passion.
Mais la môm' qu' a pas la trouille,
M'attrap' par la peau des couilles.
Ma quéquette
Dress' la tête
Et nous voilà tous les deux
De plus en plus amoureux
Sur un canapé moelleux.
\endverse

\beginverse
Très émue elle sanglotte :
Fais-moi jou-ir, enfonc'-moi la pin' dans la motte.
Va, je ne suis pas farouche,
Tu m' la foutras dans la bouche
C'est aujourd'hui jour de fête,
Attends un peu, j' m'en vais t' claquer sur les roupettes
Avec mes nichons pointus
J' te chatouill'rai l' trou du cul.
\endverse

\beginverse
On écart' d'abord les cuisses,\footnote {Anciennement (Les Fleurs du Mâle, 1958) ce couplet servait de refrain à la chanson.}
Sans s'occu, -cu, sans s'occuper du trou qui pisse
Pour qu' la jouissanc' soit complète,
On fout l' doigt dans l' trou qui pète.
Puis avec de la vas'line
On y fait gli-, on y fait gli-isser la pine
Si ça n' sent rien en entrant,
Ça pue la merd' en sortant !
\endverse

\beginverse
Cett' vadrouille,
De mes couilles,
Eut un triste lendemain :
Au matin, Bon Dieu d' punaise !
La môm' filait à l'anglaise.
Plus d' galette,
Montr' refaite,
J'en étais comm' deux ronds d' flan.
J'étais entôlé sal'ment
Par la môm' lâché d'un cran.
\endverse

\beginverse
Huit jours après cett' aventure,
Queues de cerises et mixtur' de chapelure,
J' m'aperçois qu' ma pauvre pine
Faisait un' bien triste mine.
Oh ! Bon Dieu d' caricature !
Si je t'attrap', j' te cass' la gueul', je te le jure !
En attendant mon p'tit frère
Vers' des larmes bien amêres.
\endverse

\endsong