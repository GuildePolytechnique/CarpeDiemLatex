\beginsong {La dispute du cul et du con} [
ititle= {Dispute du cul et du con (la)},
tu= {Barbari, mon ami (1648).}]

\beginverse
Chacun de vous sait qu'autrefois
Au Japon comme en France
Le trou du cul avec le con
Vivait d'intelligence.
Voulez-vous savoir la raison
La faridondain', la faridondon,
Qui les a rendus ennemis, Biribi,
À la façon de Barbari, mon ami.
\endverse

\beginverse
Le trou du cul plein de fierté,
Disait dans son langage :
" Foutras-tu toujours sous mon nez
Et dans mon voisinage ?
Comme toi ne suis-je pas bon ? ...
À recevoir aussi le vit, Biribi ... "
\endverse

\beginverse
En entendant ceci, du con
Grande fut la colère
Et il en supprima, dit-on
Les règles ordinaires
" Tais-toi, dit-il, foutu cochon ...
Tu n'es bon qu'à salir le vit, Biribi ... "
\endverse

\beginverse
" C'est bien à toi, reprit le cul,
De parler d'immondices,
Du moins, on ne m'a jamais vu
Foutre la chaude-pisse
Toujours couvert de morpi-ons ...
T'as souvent la vérol' aussi, Biribi ... "
\endverse

\beginverse
À ce moment survint un vit
De superbe encolure
Il était, ma foi, fort bien mis
Et de belle tournure :
" Paix, leur dit-il, taisez-vous donc ...
Vous faites beaucoup trop de bruit, Biribi ... "
\endverse

\beginverse
Tout d'abord, il entra au con
Qu'il trouva bien trop large,
Puis dans l' trou du cul sans façon
Par trois fois, il décharge,
" Hé, hé, dit-il, taisez-vous donc ...
Plus c'est étroit, et plus on jouit, Biribi ... "
\endverse

\beginverse
À cet arrêt, si bien pourtant,
Le con bava de rage,
Et le trou du cul triomphant,
Fit un sacré tapage,
Par trois fois, il pèt' sur le con ...
Lui disant : " Ton règn' est fini, Biribi ... "
\endverse

\beginverse
Le bougre avait ma foi raison,
Je le dis sans mystère
Pour foutre, Il n'est qu'un trou de bon
C'est le trou de derrière
Souple, nerveux et très profond ...
Dieu pour le vit exprès le fit, Biribi ...
\endverse

\endsong