\beginsong {Le duc de Bordeaux\footnotemark} [
ititle= {Duc de Bordeaux (le)}]

\footnotetext {Cette chanson daterait, pour quelques vers du moins, de 1820, date de naissance du duc de Bordeaux qui n'était autre que Henri de Bourbons, fils posthume du duc de Berry.}

\beginverse
Le duc de Bordeaux ressembl' à son frère,
Son frèr' à son pèr' et son pèr' à mon cul ;
De là je conclus qu' le duc de Bordeaux
Ressembl' à mon cul comme deux gouttes d'eau.
\endverse

\beginchorus
\textbf {Refrain}
Taïaut ! Taïaut ! Taïaut !
Ferme ta gueule répondit l'écho.
\endchorus

\beginverse
Le duc de Chevreuse ayant déclaré
Que tous les cocus devaient être noyés,
Madam' de Chevreuse lui a demandé,
S'il était certain de savoir bien nager.
\endverse

\beginverse
Madam' la duchesse de la Trémouille\footnote {La duchesse de la Trémouille dont il est question ici serait probablement l'épouse de Claude, duc de la Trémouille, de la famille des Trémoille marquis de Noirmoutier. OK ? (Vérifiez au Château-musée de Noirmoutier, 85330, au 2ème étage pour la modeste somme de 20 FF). Ndlr : Mimi a vérifié.}
Malgré sa pudeur et sa grande piété,
A patiné plus de paires de couilles
Que la Grand' Armée n'a usé de souliers.
\endverse

\beginverse
Le Roi Dagobert a un' pin' en fer ;
Le bon Saint Éloi lui dit : " Eh bien ! mon roi,
Si vous m'enculez, vous m'écorcherez. "
" C'est vrai, dit le Roi, j'en f'rai fair' un' de bois. "
\endverse

\beginverse
J'emmerde le Roi\footnote {Ce roi serait le Roi Louis XVIII.} et le comt' d'Artois
Le duc de Berry et la duchess' aussi,
Le duc de Nemours, j' l'emmerd' à son tour
Le duc d'Orléans, je l'emmerd' en mêm' temps !
\endverse

\beginverse
Chasseur as-tu vu le trou de mon cul ?
Si tu veux le voir, tu reviendras ce soir.
Moi, j'ai vu le tien, je n'en ai rien dit ;
Si tu vois le mien, tu n'en di-iras rien".
\endverse

\beginverse
La p'tit' Amélie m'avait bien promis
Trois poils de son cul pour en fair' un tapis
Les poils sont tombés, l' tapis est foutu
La p'tit' Amélie n'a plus d' poil à son cul.
\endverse

\beginverse
La bit' à papa qu'on croyait perdue,
C'était la p'tit' bonn' qui l'avait dans les fesses.
La bit' à papa n'était pas perdue,
C'était la p'tit' bonn' qui l'avait dans le cul.
\endverse

\endsong