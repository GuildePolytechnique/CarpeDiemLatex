\beginsong {En descendant la rue Tronchet\footnotemark} [
ititle= {En descendant la rue Tronchet},
tu= {Où allez-vous, monsieur l'abbé ?}]

\footnotetext {Version moderne belge de \emph{En descendant la rue d'Alger} qui fut sans doute inspirée par \emph{L'Abbé raccroché} (in Le Panier aux Ordures, 1878). Les étudiants du P.C.B. (année préparatoire aux études médicales ) remplacent \emph{la rue d'Alger} par \emph{la rue Cuvier}.}

\beginverse
\bis {En descendant la rue Tronchet}
\bis {Par un' putain, j' fus racolé}
Elle me dit d'un air tendre, eh bien ?
Viens coucher dans ma chambre
\bis {Et vous m'entendez bien. }
\endverse

\beginverse
\bis {Moi qui suis poil à l'ULB,}
\bis {J'aim' à savoir où j' mets les pieds}
J'allume ma chandelle, eh bien ?
J'éclaire le bordel,
\bis {Et vous m'entendez bien. }
\endverse

\beginverse
\bis {Quand le bordel fut éclairé}
\bis {J' la prends, j' la fous sur l' canapé}
Et je la carambole si bien
Qu'elle me fout la vérole,
\bis {Et vous m'entendez bien }
\endverse

\beginverse
\bis {Un vieux toubib, quatr' infirmiers}
\bis {Fur'nt désignés pour me soigner}
Mais cette band' d'andouilles, eh bien !
Ils m'ont coupé les couilles,
\bis {Et vous m'entendez bien}
\endverse

\beginverse
\bis {Depuis ce jour, soir et matin}
\bis {Je maudis toutes les putains}
Mais ce que je regrette, eh bien,
C'est ma pair' de roupettes,
\bis {Et vous m'entendez bien.}
\endverse

\endsong