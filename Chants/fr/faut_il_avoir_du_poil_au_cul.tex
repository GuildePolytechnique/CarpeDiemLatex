\beginsong {Les poils du cul\footnotemark} [
ititle= {Poils du cul (les)},
ititle= {Faut-il avoir du poil au cul?},
ititle= {Il faut avoir du poil au cul}]

\footnotetext { Autre titre : \emph{Faut-il avoir du poil au cul?} ou \emph{Il faut avoir du poil au cul} de Auguste Lefranc (in le Parnasse Satyrique du XIXème siècle, 1864).}

\beginverse
Faut-il avoir du poil au cul ?
Comment résoudre cette affaire ?
Les uns dis'nt que c'est nécessaire,
Les autres que c'est superflu.
Dans ce débat contradictoire
Où rien encor n'est résolu,
La Bible, la Fable et l'Histoire,
Vont vous parler des poils du cul.
\endverse

\beginverse
Adam sans doute était velu,
Car cet insecte parasite
Qui sur nos couilles fait son gîte,
Par un froid vif est morfondu ;
Et Dieu qui donna la pâture
A l'oiseau faible et peu vétu,
Aux morpions pour couverture,
Donna les poils de notre cul.
\endverse

\beginverse
" Faut-il avoir du poil au cul,
Disait Hercule aux pieds d'Omphale,
Et que t'importe, ô ma vestale,
Un rouston plus ou moins velu ? "
Dit-il, et découvrant ses couilles,
Des poils lustrés, fins et touffus
Il enroula sur la quenouille
Cent écheveaux de poils du cul.
\endverse

\beginverse
" Faut-il avoir du poil au cul ? "
Disait Thésée aux Amazones,
Quand, à trois cents de ces personnes
Sa pine au cul il eut foutu.
Bandant encore à la dernière
Il dit : " Ma bell' qu'en penses-tu ? "
" Cré nom de Zeus ! dit la guerrière,
Il faut avoir du poil au cul. "
\endverse

\beginverse
Ce fut David au cul tout nu
Qui, armé d'une simple fronde,
Mais d'une main que Dieu seconde
Tua Goliath au cul velu.
Ceci vous prouve bien, je pense,
Que tout Hébreu bien résolu
Doit compter sur la Providence
Plus que sur les poils de son cul.
\endverse

\beginverse
Ce fut par un poil de son cul
D'une longueur phénoménale,
Qu'au bout de la branche fatale,
Absalon resta suspendu !
Depuis ce trépas mémorable,
Tous les Hébreux ont résolu,
Pour éviter un sort semblable,
De se raser les poils du cul !
\endverse

\beginverse
Samson qui, cert's, était velu,
A vu, par une main traîtresse,
Avec le poil noir de sa fesse,
Tomber sa force et sa vertu.
Sous le ciseau qui le dépeuple,
Quand le poil tomb' tout est foutu;
C'est ainsi que le sort des peuples
Tient, dit la Bible, aux poils du cul.
\endverse

\beginverse
Aux temps de nos rois chevelus
Et de l'antique Loi salique,
C'était un titre honorifique
Que de porter du poil au cul.
Mais notre siècle égalitaire\footnote {Les quatre derniers vers se chantent sur l'Internationale (P. : Pottier, 1871 - M. : Degeyter).}
A réformé tous ces abus,
Et maintenant le prolétaire
Peut se payer du poil au cul.
\endverse

\beginverse
Faut-il avoir du poil au cul ?
Vous connaissez tous la Pucelle :
Et bien, certes, ce fut par elle
Que les Anglais furent vaincus.
A la vue de son oriflamme,
Tous les Anglais au cul velu
Ont foutu l' camp devant un' femme
Qui n'avait pas de poils au cul.
\endverse

\beginverse
" Faut-il avoir du poil au cul,
Disait Henri au duc de Guise ? "
Mais celui-ci qui le méprise
N'a pas au sire répondu.
Pour lors le Roi dans sa colère
S'écria : " Je veux qu'on le tue;
Nous pourrons de cette manière
Voir s'il avait du poil au cul. "
\endverse

\beginverse
Avaient-ils donc du poil au cul
Quand, pris d'une valeur antique,
A l'appel de la République
Femm's et veillards sont accourus ?
Remplis d'une ardeur sans pareille,
Jusqu'aux enfants, tous s' sont battus,
Car la valeur, a dit Corneille,
N'a pas besoin de poils au cul.
\endverse

\beginverse
Ce fut par un poil de son cul
Dégraissé pour la circonstance
Que l'hygromètre fut en France
Par de Saussure suspendu.
Ceci prouve avec évidence,
Que tout Français, chauve ou poilu,
Doit réserver pour la science,
Le plus long des poils de son cul.
\endverse

\beginverse
" Faut-il avoir du poil au cul
Disait aux pieds des pyramides,
A ses bataillons intrépides,
Un général fort bien connu ? "
Qu'importe, mais dans la bataille,
Fut-il vainqueur, fut-il vaincu,
Jamais Français sous la mitraille
N'a montré les poils de son cul.
\endverse

\beginverse
" Faut-il avoir du poil au cul ?
Disait au bon Monsieur Fallière,
Un attaché militaire
Qui portait un casque pointu ? "
Alors l'homme à la Lavallière
Lui dit : " Soyez bien convaincu,
Les Français, si survient la guerre,
Vous botteront les poils du cul. "
\endverse

\beginverse
Faut-il avoir du poil au cul ?
Nous avons en cette rencontre
Pesé le pour, pesé le contre
Et rien encore n'est résolu.
Mais un avis que je crois sage
Que nul encor' n'a combattu,
C'est qu'il vaut mieux pour son usage :
Un cul sans poil, qu'un poil sans cul.
\endverse

\endsong