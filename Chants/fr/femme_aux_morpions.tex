\beginsong {La femme aux morpions} [
ititle= {Femme aux morpions},
tu= {La femme aux bijoux (E. Dumont)}]

\beginchorus
\textbf {Refrain}
C'est la femm' aux morpions
Cell' qui tap'\footnote {Synonyme de puer. Dans d'autres versions, on peut trouver "pue" à la place.} du con.
Bien qu'elle soit bell' gosse
Tout ceux qui l'ont baisée
Ne peuvent plus bander.
Pour un' pièc' de vingt ronds
Elle suc' sans façon
Les pin's les plus grosses.
La reine du suçon
C'est la femme aux morpions.
\endchorus

\beginverse
Quand j' l'ai rencontrée, la femme aux morpions,
C'était dans un bal, près des Butt's Chaumont,
Au son d'un piano mécanique
Qui f'sait plus d' bordel que d' musique.
J'invitai la même à faire un tango ;
Son peignoir ouvert laissait voir sa peau
Et d'vant cett' superbe poule
Je faillis perdre la boule.
Quand je la baisai sur son escalier
Le voisin du d'ssous se mit à chanter :
\endverse

\endsong