\beginsong {La femme du vidangeur\footnotemark} [
ititle= {Femme du vidangeur (la)},
ititle= {Pendu (le)},
ititle= {On n'est jamais content (l')}]

\footnotetext {Autres titres : Le pendu , L'on est jamais content. L' "Ordre Studentyssime Vesnesrable et Très Magnyfique de François Villon de Montcorbier" a repris cette chanson comme chant d'ordre. On trouvera la version originale dans "69 Chansons d'étudiants" (1984).}

\beginverse
L'autre jour, l'idée m'est venue,
Cré nom de Dieu, d'enculer un pendu !
Mais l' vent soufflait sur la potence,
Voilà mon pendu qui s' balance.
Je n'ai pu l'enculer qu'en volant !
Cré nom de Dieu ! L'on n'est jamais content !
\endverse

\beginchorus
\textbf {Refrain}\footnote {La dernière partie du refrain original vous a été épargnée afin d'éviter l'étouffement général.}
La femm' du vidangeur
Préfère à toute odeur
L'odeur de son amant
Qu'elle aime éperdument.
Ils étaient deux amants
Qui s'aimaient tendrement
Qui faisaient par-devant
Par-derrière ;
Ils étaient deux amants
Qui s'aimaient tendrement,
Qui faisaient par-derrière
Ce qu'on fait par-devant.
\endchorus

\beginverse
À baiser un con trop petit,
On risque fort de s'écorcher le vit ;
Mais quand le vagin est trop large,
On ne sait plus où l'on décharge.
Se masturber n'est pas très élégant.
Cré nom de Dieu ! L'on ne jouit jamais tant !
\endverse

\beginverse
En arrivant au Paradis,
Je sentis se redresser mon long vit.
J'ai baisé Saint-Michel l'Archange,
La Sainte Vierge et tous les anges.
Si l' Bon Dieu n' s'était pas cavalé
Cré nom de Dieu ! Je l'aurais enculé !
\endverse

\endsong