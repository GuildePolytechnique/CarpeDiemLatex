\beginsong {La fille de gonthier} [
ititle= {Fille de gonthier (la)}]

\beginverse
Au bord de la Moselle,
Y'avait un batelier ;
Sa fille était pucelle,
Et chacun le savait.
Tous les gars du village
Entre eux se demandaient :
" Qui aura le pucelage
De la fill' de Gonthier,
De la fill' de Gonthier, tirelé,
Qui a toujours son pucelage
De la fill' de Gonthier, tirelé,
Qui n' veut pas le donner ? "
\endverse

\beginverse
Elle fit la rencontre
D'un galant de chez nous
Qui lui offrit sa montre
Et la prit sur les g'noux ;
Un oiseau dans les vignes,
Éperdument chantait ;
Elle n'eut qu'à faire un signe
Et l'oiseau s'envolait ;
Et la fill' de Gonthier, tirelé,
Perdait son pucelage
Et la fill' de Gonthier, tirelé,
N'eut plus rien à donner.
\endverse

\beginverse
Malgré bien des promesses
L'amant ne revint pas,
Pour cacher sa grossesse
La pauvrett' se noya.
Aux jeun's fill's plein's de crainte,
L'hiver, à la veillée,
On chante la complainte
De la fill' de Gonthier
De la fill' de Gonthier, tirelé,
Qui a perdu son puc'lage
Et qui s'est suicidée, tirelé,
De n' pouvoir le r'trouver.
\endverse

\endsong