\beginsong {La jeune fille du métro\footnotemark} [
ititle= {Jeune fille du métro (la)},
ititle= {Fille du métro (la)}]

\footnotetext {Autre titre : la fille du métro. Chanson "Belle époque" attribuée à Aristide Bruant et basée sur le jeu de la fausse-rime, très prisé à l'époque.}

\beginverse
C'était un' jeun' fill' chaste et bonne
Qui n' refusait rien a personne.
Un jour dans l' métro y'avait presse,
Un jeun' homme osa, je l' confesse,
Lui passer la main dans les ... ch'veux
Comme elle avait bon coeur
Elle s'approcha un peu.
\endverse

\beginverse
L' jeune homm' vit l' mouvement d' la demoiselle
Il se rapprocha de plus belle ;
Mais comm' en chaque homme tout de suite
S'éveill' le cochon qui l'habite,
Sans tarder il sortit sa ... carte,
Lui dit qu'il s'app'lait Jules
Et d'meurait rue Descartes.
\endverse

\beginverse
L' métro continuait son voyage.
Elle dit : " Ce jeun' homme n'est pas sage,
Je sens quelque chos' de pointu
Qui, d'un air ferme et convaincu,
Cherch' à pénétrer dans mon ... coeur
Ah, qu'il est doux d'aimer,
Doux frisson du bonheur ! "
\endverse

\beginverse
Comm' elle avait peur pour sa robe,
A cett' attaque elle se dérobe ;
Voulant savoir c' qui la chatouille,
Derrièr' son dos elle tripatouille,
Et tomb' sur un' bell' pair' de ... gants,
Que l' jeun' homme, à la main,
Tenait négligemment.
\endverse

\beginverse
Ainsi à Paris quand on s'aime,
On peut s' le dire en public même.
Les amoureux ne s' font pas d' bile,
Entre tout l' mond', ils se faufilent,
Je crois même bien qu'ils s'en ... fichent
L'amour ouvrant les yeux
Même aux gens très godiches.
\endverse

\endsong