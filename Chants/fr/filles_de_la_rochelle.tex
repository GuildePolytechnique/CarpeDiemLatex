\beginsong {Les filles de la Rochelle} [
ititle= {Filles de la Rochelle (la)},
tu= {Les filles de La Rochelle (Traditionnel)}]

\beginverse
Sont les filles de La Rochelle
\bis {Qu' ont armé un bâtiment.}
Ell's ont la cuisse légère
Et la fess' à l'avenant.
\endverse

\beginchorus
Refrain
Ah ! la feuille s'envole, s'envole
Ah ! la feuille s'envol' au vent.
\endchorus

\beginverse
Sont parties aux Amériques
\bis {Un matin, la voil' au vent ;}
Ont choisi pour capitaine
Une fille de quinz' ans.
\endverse

\beginverse
Nous n'avons pas besoin d'hommes,
\bis {Disaient-ell's à tout venant ;}
Mais au bout de six semaines
Ell's avaient le cul brûlant.
\endverse

\beginverse
Un beau soir, une frégate
\bis {Apparut sur l'Océan,}
Pleine de jolis pirates,
De beaux gars appétissants.
\endverse

\beginverse
Ell's allèr'nt à l'abordage
\bis {À coups d' sabre et à coups d' dents}
Ell's y prirent l'avantage
Et se ram'nèr'nt des galants.
\endverse

\beginverse
Et sous la Lune jolie,
\bis {étendues sans vêtements,}
Ell's ont écarté les cuisses
Tout's sur le gaillard d'avant.\footnote {Gaillard : (de château gaillard, château fort ; 1573) chacune des superstructures placées à l'avant et à l'arrière, sur le pont supérieur, et servant de logement. Actuellement, seul le gaillard d'avant a gardé son nom, et le gaillard d'arrière s'appelle "dunette". (in Larousse, Dictionnaire de la langue française Lexi 1992)}
\endverse

\beginverse
Ont baisé à perdre haleine
\bis {Jusqu'au clair soleil levant}
Et c'était la capitaine
Qui menait le mouvement.
\endverse

\beginverse
Le lend'main le beau navire
\bis {Repartit vers le couchant,}
Et les fill's de La Rochelle
Le cul frais allaient chantant :
" J'ai perdu mon pucelage
\bis {Au milieu de l'Océan.}
Il est parti vent arrière
Reviendra z'en louvoyant. "\footnote {Louvoyer : (de lof ; 1524) naviguer contre le vent, tantôt à droite, tantôt à gauche de la route à suivre. (in Larousse, Dictionnaire de la langue française Lexis 1992)}
\endverse

\endsong