\beginsong {Le fils-père} [
ititle= {Fils-père (le)}]

\beginverse
Il était beau, il s'appelait Jules
Et il n'avait jamais fauté,
Quand un beau soir au crépuscule
Par le désir, il fut hanté
Sous la forme d'une brunette
Qui descendait de l'autobus.
Elle lui dit : " Viens dans ma chambrette,
J'habit' là-haut, quartier Picpus. "
\endverse

\beginverse
\textit {Amour, amour, tu fais fair' des folies
Amour, amour, tu nous fais bien du mal.
Il soupira : " Si je faute ma mie,
M'épous'ras-tu ? " " Oui, dit-elle, c'est fatal "
Mais quand il s' fut donné bêt'ment
Elle lui dit : " Maintenant, fous l' camp ! "
Elle le chassa de sa maison
Sans mêm' lui rendr' son pantalon.
C'est alors qu'il comprit
Sa honte et sa misère,
Un malaise le prit
Jules allait être père.}
\endverse

\beginverse
Afin d' dissimuler sa faute
Il prit d'horribles précautions :
Il se serra les entrecôtes
Et fit élargir ses cal'çons.
Mais un jour, il perdit sa place,
Le patron l'ayant fait app'ler
Lui dit : " Jules, t' as fauté, j' te chasse ;
Faut pas d' fils pèr' à l'atelier. " 
(Parlé : Mon Dieu !)
\endverse

\beginverse
\textit {Amour, amour, tu fais fair' des folies
Amour, amour, tu nous fais bien du mal.
Pour oublier, il sombra dans l'orgie
Il but du cidr' et de l'Urodonal.
Alors à Montmartre là-haut,
On l' vit rouler dans le ruisseau
Tandis que d' joyeux noctambul's
Venaient tirer l'oreille à Jules.
Et de son pauvre corps
Les filles abusèrent ;
On n'est pas respecté
Quand on est un fils père.}
\endverse

\beginverse
Un soir, dans un' louch' officine,
Il entra décidé à tout.
Il vit un' femm', une gourgandine
Qui s'app'lait "la mèr' Guett'-au-trou"
Pour fair' disparaître les traces
De la faute du pauvre gueux,
Elle lui charcuta la carcasse
En se servant d'un' pell' à feu. 
(Parlé : Oh, quelle horreur !)
\endverse

\beginverse
\textit {Amour, amour, tu fais fair' des folies
Amour, amour, tu nous fais bien du mal.
Le pauvre gars faillit perdre la vie
Il vient d' sortir de l'hôpital
Et maintenant pâl' et flétri,
La peau d' son ventr' faisant des plis,
Sur l' Sébasto, on peut le voir
Jul's est dev'nu fils du trottoir.}
Parlé : \textit{moralité
Mariez-vous, jeunes gens
Ne vous laissez pas faire.
Ne faites pas comm' Jul's
Le malheureux fils père.}
\endverse

\endsong