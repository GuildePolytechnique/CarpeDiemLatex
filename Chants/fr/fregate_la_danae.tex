\beginsong {La frégate la Danaé\footnotemark} [
ititle= {Frégate la Danaé (La)}]

\footnotetext {Chanson à virer du répertoire du gaillard d'avant, inspirée de la ballade Le plongeur de Friedrich von Schiller.}

\beginverse
\bisdeux {C'était une frégate,}{Larguez les ris,\footnote {Ris (anc. scand. rifs ; 1155) : partie d'une voile où passent les garcettes (1643 : petit cordage tressé, servants à faire différents amarrages), qui permettent de la serrer sur la vergue (forme norm. ou picarde de verge (1240) : espar cylindrique, effilé à ses extrémités, placé en travers d'un mât pour soutenir et orienter la voile) pour diminuer sa surface. (in Larousse,Dictionnaire de la langue française Lexis 1992)}}
Qui s'app'lait La Danaé,
Larguez les ris dans les bass's voiles.
Qui s'app'lait La Danaé,
Larguez les ris dans les huniers.\footnote {Hunier (1615) : voile carrée enverguée sur la vergue de hune et hissée sur le mât de hune (anc. scand. hûnn ; 1138 : plate-forme fixée sur les bas-mâts). (in Larousse, Dictionnaire de la langue française Lexis 1992)}
\endverse

\beginverse
à son premier voyage, ...
La frégat' a bien marché ...
\endverse

\beginverse
À son deuxièm' voyage, ...
La frégat' heurt' un rocher ...
\endverse

\beginverse
À son troisièm' voyage, ...
La frégat' a chaviré ...
\endverse

\beginverse
De tout son équipage, ...
Un seul homme fut sauvé ...
\endverse

\beginverse
C'était un quartier-maître, ...
Qui savait fort bien nager ...
\endverse

\beginverse
Arrivant au rivage, ...
Il vit un' femm' éplorée ...
\endverse

\beginverse
Bell' comm' un' frégate, ...
Française et pavoisée ...
\endverse

\beginverse
Il lui dit : " Oh ! La belle, ...
Qu'avez-vous donc à pleurer ? " ...
\endverse

\beginverse
" J'ai perdu mon puc'lage, ...
Et ne puis le retrouver ! " ...
\endverse

\beginverse
" Ne pleurez pas, la belle ...
On va vous le rechercher ...
\endverse

\beginverse
Et qu'aurait donc, la belle, ...
Celui qui vous le rendrait ? " ...
\endverse

\beginverse
" Lui en ferait offrande ...
Avecque mon amitié. " ...
\endverse

\beginverse
À son premier coup d' sonde ...
L' quartier-maîtr' n'a rien trouvé ...
\endverse

\beginverse
Car jamais pucelage ...
Pe-erdu n'est retrouvé ...
\endverse

\beginverse
À son second coup d' sonde ...
L' quartier-maîtr' est vérolé ...
\endverse

\beginverse
Moral' de cett' histoire : ...
Il ne faut jamais baiser ...
\endverse

\beginverse
Moral' de la morale : ...
Gardez-vous de l'observer ...
\endverse

\endsong
