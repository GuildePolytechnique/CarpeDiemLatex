\beginsong {Le Fusil} [
ititle= {Fusil (le)},
tu= {Lolotte (Jacques Bertrand, 1865)}]

\beginverse
J'avais quinz' ans et la passion des armes,
Un beau fusil tout neuf et tout luisant.
J'aurais voulu connaître les alarmes
Et les combats de tout soldat vaillant.
Mon père était de la garde civique,
Pour son adresse, on l'admirait beaucoup :
\bisdeux {Ah, mes amis ! Ah, quel plaisir unique} {Quand je voyais papa tirer son coup.}
\endverse

\beginverse
Un beau matin, je lui dis : " Petit père,
J'ai mes quinz' ans et j' voudrais essayer
Le beau fusil que seul avec ma mère
Tu mis neuf mois à pouvoir m' fabriquer. "
Il m' répondit d'une voix marti-ale :
" Ta noble ardeur me réjou-it beaucoup.
\bisdeux {Tiens, mon enfant, voilà toujours cinq balles,} {Va-t-en mon fils, va-t-en tirer ton coup ! "}
\endverse

\beginverse
En ce temps là, vint un tir à la mode
Qui s'établit, je crois, rue du Persil.
Vit' je courus vers cet endroit commode
Pour essayer mon excellent fusil.
Les cibl's étaient toutes blanches et roses,
Mon beau fusil se leva tout à coup,
\bisdeux {Je déchargeai et je fis une rose.} {Ah ! Mes amis, que c'est bon l' premier coup.}
\endverse

\beginverse
En peu de temps, ma renommée fut grande.
De nobles dam's se disputaient l'honneur
De chatouiller avec leurs mains fringantes
Le beau fusil d'un si parfait chasseur ;
Toutes les nuits, j'étais à l'exercice,
Ma cartouchièr' n'était jamais à bout.
\bisdeux {Mais maintenant, j'use d'un artifice} {Je ne peux plus par nuit tirer qu'un coup.}
\endverse

\beginverse
Et maintenant l' beau fusil, qui, naguère,
A, d' si hauts faits, si souvent abusé,
Repos' en paix au musée de la guerre
Où il surmont' deux vieux boulets usés.
Il a connu tant de chaudes alarmes
Et tant de com-bats livrés coup sur coup.
\bisdeux {Quand, par hasard, il laiss' couler un' larme,} {C'est par regret de n' plus tirer son coup.}
\endverse

\endsong
