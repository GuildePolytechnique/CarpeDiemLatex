\beginsong {Le grand métingue du métropolitain\footnotemark} [
ititle= {Grand métingue du métropolitain (Le)}]

\footnotetext { La mélodie fredonnée actuellement est tout à fait différente de l'originale qui n'est autre que celle d'Alphonse du Gros Caillou. Métingue, francisation de \textit{meeting}, est à prononcer mètingue.}

\beginverse
C'était hier, samedi, jour de paye,
Et le soleil se levait sur nos fronts.
J'avais déjà vidé plus d'un' bouteille,
Si bien qu' j' m'avais jamais trouvé si rond.
V'là la bourgeois' qui rappliqu' devant l' zingue :
" Feignant, qu'elle dit, t'as donc lâché l' turbin ? "
\bisdeux {" Oui, que j' réponds, car je vais au métingue, } {Au grand métingue du métropolitain ! " }
\endverse

\beginverse
Les citoyens, dans un élan sublime,
Etaient venus guidés par la raison.
A la port', on donnait vingt-cinq centimes
Pour soutenir les grèves de Vierzon.
Bref, à part quatr' municipaux qui schlinguent
Et trois sergeots déguisés en pékins,
\bisdeux {J'ai jamais vu de plus chouette métingue} {Que le métingue du métropolitain !}
\endverse

\beginverse
Y'avait Basly, le mineur indomptable,
Camélinat, l'orgueille du pays ...
Ils sont grimpés tous deux sur une table,
Pour mettre la question sur le tapis.
Mais, tout à coup, on entend du bastringue ;
C'est un mouchard qui veut fair' le malin !
\bisdeux {Il est venu pour troubler le métingue } {Le grand métingue du métropolitain !}
\endverse

\beginverse
Moi j' tomb' dessus, et pendant qu'il proteste,
D'un grand coup d' poing, j'y renfonc' son chapeau.
Il déguerpit sans demander son reste,
En faisant sign' au quatr' municipaux.
A la faveur de c' que j'étais brind'zingue
On m'a conduit jusqu'au poste voisin ...
\bisdeux {Et c'est comm' ça qu'a fini le métingue,} {Le grand métingue du métropolitain.}
\endverse

\beginverse
Moralité :
Peuple français, La Bastill' est détruite
Mais y'a z-encor' des cachots pour tes fils !
Souviens-toi des géants de quarante-huit(e)
Qu'étaient plus grands qu' ceuss' d'au jour d'aujourd'hui.
Car c'est toujours l' pauvr' ouverrier qui trinque
Mêm' qu'on le fourr' au violon pour un rien ...
\bisdeux {C'était tout d' mêm' un bien chouette métingue,} {Que le métingue du métropolitain !}
\endverse

\endsong
