\beginsong {Les marteaux \footnotemark} [
ititle= {Marteaux, les},
ititle= {C'est à boire qu'il nous faut},
ititle= {Nous étions cinq, six bons bougres}]

\footnotetext { Autres titres : \emph{C'est à boire qu'il nous faut}, \emph{Nous étions cinq, six bons bougres}.}

\beginverse
Nous étions six fameux bougres
Revenant de Longjumeau,
Nous entrâm's dans une auberge
Pour y boir' du vin nouveau. Oh !
\endverse

\beginchorus
\textbf {Refrain}
C'est à boire, à boire, à boire,
C'est à boire qu'il nous faut !
Oh ! Oh ! Oh ! Oh !
\endchorus

\beginverse
Nous entrâm's dans une auberge
Pour y boir' du vin nouveau.
Nous vidâm's plus d'un' fiole
Nous y bûmes plus d'un pot. Oh !
\endverse

\beginverse
Chacun fouilla dans sa poche\footnote {Les deux premières strophes se chantent sur un mode qui n'a absolument aucun rapport avec la manière dont le reste de la chanson est interprété; sans doute qu'à l'origine, on le chantait comme ça.}
Quand il fallut payer l' pot,
Dans la poche du plus riche
On n' trouva qu'un écu faux. Oh !
\endverse

\beginverse
" Sacrebleu ! dit la patronne,
Qu'on leur prenne leur shako ! "
" Nom de Dieu ! dit la servante,
Leur falzar, leurs godillots. " Oh !
\endverse

\beginverse
Quand nous fûmes en liquette,
Nous montâm's sur des tonneaux,
Nos liquett's étaient si courtes
Que l'on voyait nos marteaux. Oh !
\endverse

\beginverse
" Sacrebleu ! dit la patronne,
Qu'ils sont noirs et qu'ils sont beaux ! "
" Nom de Dieu ! dit la servante,
J'en voudrais bien un morceau. " Oh !
\endverse

\beginverse
" Sacrebleu ! dit la patronne,
Tous les six, il me les faut ! "
Et tous les six y passèrent,
Du plus p'tit jusqu'au plus gros. Oh !
\endverse

\beginverse
" Sacrebleu! dit la patronne,
Qu'on leur rende leur shako ! "
" Nom de Dieu ! dit la servante,
Leur falzar, leurs godillots. " Oh !
\endverse

\beginverse
Et en sortant nous plaçâmes
Sur la porte un écriteau :
C'est ici qu'on boit, qu'on mange
Et qu'on paye à coups d' marteaux. Oh !
\endverse

\endsong