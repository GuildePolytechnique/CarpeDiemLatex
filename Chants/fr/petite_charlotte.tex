\beginsong {La petite Charlotte\footnotemark} [
ititle= {Petite Charlotte, la},
ititle= {Charlotte},
ititle= {Carotte, la}]

\footnotetext {Autre titre : \emph{La carotte}, \emph{Charlotte}.}

\beginverse
Dans son boudoir la petite Charlotte
Chaude du con faute d'avoir un vit
Se masturbait avec une carotte
Et jou-issait étendue sur son lit.
\endverse

\beginchorus
\textbf {Refrain}
Branle, branle, branle Charlotte
Branle, branle, ça fait du bien.
Branle, branle, branle ma chère
Branle, branle jusqu'à demain.
\endchorus

\beginverse
" Ah !, disait-elle, en ce siècle où nous sommes,
Il faut savoir se passer des garçons,
Moi, pour ma part, je me fous bien des hommes,
Avec ardeur, je me branle le con ! "
\endverse

\beginverse
Alors sa main n'étant plus paresseuse,
Allait, venait, comme un petit ressort
Et faisait jouir la petite farceuse ;
Aussi ce jeu lui plaisait-il bien fort !
\endverse

\beginverse
Mais, ô malheur ! Ô fatal disgrâce !
Dans son bonheur, elle fait un brusque saut,
Du contrecoup, la carotte se casse,
Et dans le con, il en reste un morceau !
\endverse

\beginverse
Un médecin, praticien fort habile,
Fut appelé, qui lui fit bien du mal ;
Mais, par malheur, la carotte indocile
Ne put sortir du conduit vaginal.
\endverse

\beginverse
Mesdemoisell's que le sort de Charlotte
Puisse longtemps vous servir de leçon ;
Ah ! Croyez-moi, laissez là la carotte,
Préférez-lui le vit d'un beau garçon !
\endverse

\beginchorus
\textbf {Dernier refrain\footnote{Ce refrain est celui chanté par la Chorale de l'ULB }}
Baise, baise, baise Charlotte
Baise, baise, ça fait du bien.
Baise, baise, baise ma chère
Baise, baise jusqu'à demain.
\endchorus

\endsong