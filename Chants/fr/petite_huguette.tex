\beginsong {La petite Huguette\footnotemark} [
ititle= {Petite Huguette (La)},
ititle= {Gnouf, gnouf, gnouf},
ititle= {Petite Lisette (La)}]

\footnotetext {Autres titres : Gnouf gnouf gnouf , La petite Lisette.}

\beginverse
Un jour la p'tit' Huguette,
Tripote-moi la bit' avec les doigts.\footnote {Variante : Gnouf gnouf gnouf comm' on attrap' ça !}
Un jour la p'tit' Huguette
\bis {S'en revenait du bois.}
\endverse

\beginverse
En chemin elle rencontre
Tripote-moi la bit' avec les doigts.
En chemin elle rencontre
\bis {Le fils d'un avocat.}
\endverse

\beginverse
Il la prend, il la baise ...
\bis {Sur du foin qui était là.}
\endverse

\beginverse
Le foin était si sec(e) ...
\bis {Qu'il en faisait fla fla.}
\endverse

\beginverse
Vint à passer la mère ...
\bis {Qui revenait par là.}
\endverse

\beginverse
Ma fill', ma chère fille ...
\bis {Qu'est c' que cett' pose-là ?}
\endverse

\beginverse
Ma mèr', tu vois je baise ...
\bis {Avec ce garçon-là.}
\endverse

\beginverse
Baise, baise ma fille ...
\bis {Car on ne meurt pas d' ça.}
\endverse

\beginverse
Car si j'en étais morte, ...
\bis {Tu ne serais pas là.}
\endverse

\beginverse
Ni bien d'autres encore ...
\bis {Que papa n' connaît pas. }
\endverse

\beginverse
Et si t'en meurs, ma fille ...
\bis {Sur ta tomb', on mettra :}
\endverse

\beginverse
Ci-gît la p'tit' Huguette ...
\bis {Qu' est mort' en faisant ça.}
\endverse

\beginverse
En faisant sa prière ...
\bis {Au grand Saint Nicolas.}
\endverse

\beginverse
Ce grand saint que les hommes ...
\bis {Portent la tête en bas.}
\endverse

\beginverse
Quand ils la port'nt en l'air(e) ...
\bis {Ils inondent les draps.}
\endverse

\beginverse
Et quand ils la relèvent ...
\bis {Les femm's ne pens'nt qu'à ça. }
\endverse

\endsong 