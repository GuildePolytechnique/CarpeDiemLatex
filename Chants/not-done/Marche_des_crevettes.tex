\beginsong {La Marche des Crevettes}\footnotemark [ititle={Marche des Crevettes, La}]
\footnotetext{Robert Lanson}

\beginverse
C'est la marche des crevettes enrhumées
C'est la marche des crevettes épluchées
N'ayant pas mis de manteau
De godasses ni de chapeau
Elles ont maintenant un rhume de cerveau
\endverse

\beginverse
C'est la marche des crevettes enrhumées
Oje les pauvres, elles ne font plus qu'éternuer
Malgré les antigrippines
Les zirolles, les asprirines
Oje mon dieu, elles ont quand même mauvaise mine
\endverse

\beginchorus
Les crevettes! Les crevettes!
Sont sensibles à l'hiver et allergiques à la mer!
Plom plom plom!
\endchorus

\beginverse
C'est la marche des crevettes enrhumées
C'est la marche des crevettes enmistouflées
Dans leur gilei de mowèr
Elles évitent les courants d'air
En chantant l'Ave Maria de Shubert
\endverse

\beginverse
C'est la marche des crevettes enrhumées
C'est la marche des crevettes fatiguées
Elles prennent un peu de repos
Dans une station de métro
Attendant de faire leur numéro
\endverse

\beginchorus
Les crevettes! Les crevettes!
Sont sensibles à l'hiver et allergiques à la mer!
Plom plom plom!
\endchorus

\beginverse
C'est la marche des crevettes enrhumées
C'est la marche des crevettes épluchées
Ce soir c'est l'anniversaire
D'un cousin intermédiaire
Entre le homard et le ver de terre
\endverse

\beginverse
Comme elles aiment bien une bonn' geuze grenadine
Elle buront quelques bouteilt, ces stoem' gamines
Elles seront kroemeneilzat
Et cachées dans une tomate
Elles s'en vont chez elles à Berchem-Sainte-Agathe
\endverse

\beginchorus
Les crevettes! Les crevettes!
Sont sensibles à l'hiver et allergiques à la mer!
Plom plom plom!
\endchorus

\beginchorus
Les crevettes! Les crevettes!
Je ne veux plus les manger, elles sont scheilzat épluchées
Plom plom plom!
\endchorus
\endsong