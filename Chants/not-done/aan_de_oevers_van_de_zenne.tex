\beginsong{AAN DE OEVERS AAN DE ZENNE}\footnotemark [
ititle={Aan de oevers aan de Zenne}]

\footnotetext{Autres titres : Aan de oevers van de Dijle (Studenten Codex, Gent, 1994)}

\beginverse
Aan de oevers van de Zenne,
Daar verscholen in het riet,
Zat een kleine jonge kikker,
Bij zijn moeder op de knie!
\endverse

\beginverse
"Ziet gij daar, zo sprak die moeder,
Ziet ge daar, dien ooievaar?
'T is de moord'naar van uw vader,
Hij vrat hem op met huid en haar."
\endverse

\beginverse
"Godverdomme, sprak die kleine,
Heeft die rekel dat gedaan?
Als ik groot en sterk zal wezen,
Zal 'k hem op 't onderst' boven slaan!"
\endverse

\beginverse
Vele jaren zijn verstreken,
En de kikker is niet meer,
Maar de ooienvaar zijn kloten,
Doen verdomme nog altijd zeer!
\endverse

\beginverse
" 'k Heb zoveel u nog te zeggen,
Maar ge zoudt het niet verstaan
'k Zal u in uw bedje leggen ... "
En daarmee is 't lied gedaan!
\endverse
\endsong