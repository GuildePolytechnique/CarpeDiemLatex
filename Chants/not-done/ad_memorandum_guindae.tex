\beginsong {Ad Memorandum Guindae}\footnotemark[
ititle={Ad Memorandum Guindae},
tu={Les Acadiens (P. : Maurice Vidalin - M. : Michel Fugain)}]
\footnotetext{ISIB ; Festival de la chanson estudiantine CP ULB, 1993.}
\beginverse
Y'a dans les rues de chaque ville
Et dans des coins sombr's et pouilleux
Des tas de jeun's qui sembl'nt débiles
Mais tu n' vois pas que t' es comm' eux.
Et quand lent'ment le soir arrive
Ils s' mett'nt un tablar sur le dos
Un' penn' qui ignor' la lessive.
Qu'il pleuve, qu'il gèle, ou qu'il fass' froid
Ils gueul'ront toujours sous tes toits.
\endverse
\beginchorus
\textbf{Refrain}
Tous les guindailleurs, toutes les guindailleuses
Vont clacher, vont gueuler sur le p'tit bleu
Les "À-fonds" bien cras et les coups de tondeuses
Vous trouvez ça con, mais ça c'est pas dang'reux.
\endchorus

\beginverse
La guins', c'est cra, c'est laid, c'est nul(e)
C'est ce que pensent les vieux cons
Y'en a qui couch'nt sur pellicule
Un' parodie de c' que nous f'sont.
Mais toi comm' un vieil imbécile
Qui croit toujours avoir raison
Pour toi je ferai pas la file
Au lieu de nous montrer du doigt
Viens voir chez nous que c'est pas ça.
\endverse
\beginverse
Après l'épreuve fatidique
Il faut les voir nos petits bleus,
Ils rient, ils chantent, c'est fantastique,
Il ne se prenn'nt plus au sérieux.
Et quand ils ont leurs 50 ans
Et tous les jours le mêm' boulot
Ils rêv'nt encore du bon vieux temps ;
Boir' et chanter, c'était sympa
Avec tous ceux qui aimaient ça.
\endverse
\endsong


