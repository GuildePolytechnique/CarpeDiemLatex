\beginsong{ALLEN DIE WILLEN NAER ISLAND GAEN}\footnotemark [
ititle={Allen die willen naer Island gaen}]

\footnotetext{Répertoriée dans l'ouvrage d'Edmond de Coussemaker \emph{Chants populaires des Flamands de France} (Gand, 1856)}

\beginverse
Allen die willen naer Island gaen
Om kabeljauw te vangen
En te visschen met verlange
Naer Iseland, naer Iseland, naer Island toe
Tot drie-en-dertig reyzen zyn zy nog niet moe.
\endverse

\beginverse
Komt on de tyd van de fooie aen,
Wy danse met behaegen
En wy weten van geen klaegen :
Maer komt de tyd, maer komt de tyd naer zee te gaan
Dan is er wel ons hoofd van zorgen zwaer belaen!
\endverse

\beginverse
Als er de wind van het Noorden waeyt,
Wy gaen naer de herberge
En wy drinken zonder erge,
Wy drinken daer, wy drinken daer op ons gemak,
Totdat den lesten stuyver is uyt onzen zak.
\endverse

\beginverse
Als er de wind van het Oosten waeyt,
De schipper, bly van herte,
Zegt : "Die wind die speelt ons perten",
't Zal beter zyn, 't zal beter zyn, 't zal beter zyn,
Te lopen voor de wind recht het Kanaal maar in.
\endverse

\beginverse
Langs de Lezaars, de Schorrels voorbij,
Vandaer al naer Kaap Claire,
Die niet weet, hy zal wel leren.
Toen komt er by, toen komt er by ons sture man,
En hy geeft ons de koerse recht naer Iseland.
\endverse

\beginverse
Wy lopen 't eiland Rockol voorby,
Al naer de Vogelscharen,
Dat kan ieder openbaren;
En dan vandaar, en dan vandaar naer Bredefjord,
En daer dan smyten wy de kollen buiten boord.
\endverse

\beginverse
Eind'lijk dan komen w' op Island aen
Om kabeljauw te vangen
En te visschen met verlange
Naer Iseland, naer Iseland, naer Island toe :
Tot drie-en-dertig reyzen zyn wy nog niet moe.
\endverse
\endsong