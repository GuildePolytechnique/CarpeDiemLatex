\beginsong{ALS DE KERELS TE GARE ZIJN}\footnotemark [
ititle={Als de kerels te gare zijn}]

\footnotetext{Auteur : Berten Rodenbach. Tirée d'une mélodie anglaise.}

\beginverse
Als de kerels te gare zijn,
Doedle bomle rom dom dom,
Wat liedje moet er gezongen zijn?
Doedle rom dom dom.
'T kerelslied, 't kerelslied,
Doedle bomle rom dom dom,
'T kerelslied, 't kerelslied,
Doedle rom dom dom.
\endverse

\beginverse
Zij renden met zessen langs de baan,
Doedle bomle rom dom dom,
Zij hadden stalen kleren aan,
Doedle rom dom dom!
Isegrims, Isemgrims,
Doedle bomle rom dom dom,
Isegrims, Isegrims,
Doedle rom dom dom!
\endverse

\beginverse
Zij hadden waaiend' helmen aan, ...
Zij renden zingend langs de baan, ...
Wat zongen zij? ...
\endverse

\beginverse
Van edele ridders en heren groot, ...
Van nijdige kerels en galgedood, ...
Isegrims, Isegrims, ...
\endverse

\beginverse
De kerels kennen een schon'ren zang, ...
De zang der kerels is niet lang, ...
Maar zegt veel, ...
\endverse

\beginverse
En als de kerel aan 't zingen valt, ...
Zijn liedje vromer als d' and're schalt, ...
Storm op zee! ...
\endverse

\beginverse
Vliegt de Blauwvoet? Storm op zee! ...
Vliegt de Blauwvoet? Storm op zee! ...
Storm op zee! ...
\endverse
\endsong