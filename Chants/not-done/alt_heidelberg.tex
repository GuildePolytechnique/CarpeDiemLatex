\beginsong{ALT HEIDELBERG}\footnotemark [
ititle={Alt Heidelberg}]

\footnotetext{Auteurs : Joseph Viktor von Scheffel, 1853 - Anton Zimmermann, 1861.}

\beginverse
Alt Heidelberg, du feine,
Du Stadt an Ehren reich,
Am Nekkar und am Rheine,
Kein' and're kommt dir gleich.
Stadt fröhlicher Gesellen,
An Weisheit schwer und Wein,
Klar zieh'n des Stromes Wellen,
\bis{Blau-äuglein blitzen drein.}
\endverse

\beginverse
Und kommt aus Winden Südens,
Der frühling über 's Land,
So webt er dir aus Blüten,
Ein schimmernd' laut' Gewand.
Auch mir stehst du geschrieben,
Ins Herzgleich einer Braut.
Es klingt wie junges Lieben,
\bis{Dein Name mir, so traut!}
\endverse
\endsong