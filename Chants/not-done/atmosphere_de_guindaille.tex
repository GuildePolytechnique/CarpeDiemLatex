\beginsong{ATMOSPHERE DE GUINDAILLE}\footnotemark [
ititle={Atmosphere de guindaille},
tu={Le printemps (Michel Fugain)}]

\footnotetext{Corporation Mali Filiae. Festival de la Chanson Estudiantine ULB-CP, 1996.}

\beginverse
La guindaille est arrivée
En cette saison
Le bon air de la rentrée
Sentant le houblon
La fin de l'été,
Retour à l'ULB.
\endverse

\beginverse
Verse la bière, résonne le chant,
Vive la vie en tablier blanc.
Folle atmosphère, c'est délirant,
Et vivent les étudiants.
\endverse

\beginverse
Dépêche-toi, dépêche-toi,
Rejoins les fous chantants
Reprends ta penne et ta bonne humeur
Soûles-toi de bonheur
Du comitard à la bleuette,
Chacun apporte son grain d'folie.
Du p'tit bleu à la vieille bête,
Place à la magie.
\endverse

\beginchorus
\bisquatre{{Y'a la guindaille qui te rappelle},
	{Ecoute sa voix si tentante},
	{Y'a la guindaille qui t'ensorcelle},
	{Aussi belle qu'une amante.}}
\endchorus

\beginverse
Une ambiance survoltée
Ivre de chanson
Te sens-tu emporté
Par le tourbillon?
La bière fait briller
Ces regards étoilés.
\endverse

\beginverse
Verse la bière, résonne le chant,
Vive la vie en tablier blanc.
Folle atmosphère, c'est délirant,
Et vivent les étudiants.
\endverse

\beginverse
Réjouis-toi, défoules-toi
Ne perds pas de temps.
Hurles ta joie ou chant' Le Semeur
Sans souci de l'heure.
Le folklore jamais n'arrête
C'est la folie même à minuit.
Toute l'année, oui tu collectes
Des images pour la vie.
\endverse

\beginchorus
\bisquatre{{Y'a la guindaille qui te rappelle},
	{Ecoute sa voix si tentante},
	{Y'a la guindaille qui t'ensorcelle},
	{Aussi belle qu'une amante.}}
\endchorus
\endsong