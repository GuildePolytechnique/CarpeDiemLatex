\beginsong{La ballade des cocus}
[ititle={ballade des cocus, La}]

\beginverse
\bis{C'est pour la somme de dix francs}
\bis{Qu'on fait cocu un étudiant.}
Les étudiants eux-autres
En font cocu bien d'autres.
Et tout au long d' la s'maine
Les cocus se promènent.
\endverse
\beginchorus
\textbf{Refrain}
Cocu, cocu, cocu, cocu, cocu !
Mon dieu qu' les cocus sont heureux
Quand on leur tient la chandelle.
Mon dieu qu' les cocus sont heureux
Quand donc le serais-j' comm' eux.
\endchorus
\beginverse
\bis{C'est pour la somme d'un florin}
\bis{Qu'on fait cocu un pharmacien.}
Les pharmaciens ...
\endverse
\beginverse
\bis{C'est pour la somme d'un ducat}
\bis{Qu'on fait cocu un avocat.}
Les avocats ...
\endverse
\beginverse
\bis{C'est pour la somme d'un douro}
\bis{Qu'on fait cocu tout' la PHILO.}
Les philosoph's ...
\endverse
\beginverse
\bis{C'est pour la somme d'un kopeck}
\bis{Qu'on fait cocu la POLYTECH.}
Les polytechs ...
\endverse
\beginverse
\bis{C'est pour la somm' d'un fifrelin}
\bis{Qu'on fait cocu un carabin.}
Les carabins ...
\endverse
\beginverse
\bis{C'est pour la somm' de presque rien}
\bis{Qu'on fait cocu les trois doyens.}
Les trois doyens eux-autres
En font cocu peu d'autres ...
\endverse
\beginverse
\bis{C'est pour la somm' d'un' pièc' de bois}
\bis{Qu'on fait cocu tous les bourgeois.}
Tous les bourgeois eux-autres
N'en font cocus point d'autres ...
\endverse
\beginverse
\bis{Et moi j' m'en fous si j' suis cocu,}
\bis{Pourvu qu' ça m' rapport' un écu.}
Avec l'écu des autres
J'en f'rai cocu bien d'autres ...
\endverse
\endsong
