\beginsong{BOERENKERMIS}\footnotemark [
ititle={Boerenkermis}]

\footnotetext{Daterait du XVIIème siècle.}

\beginverse
De boerkens smelten van vreugd en plezier
Als d' oogst is binnen gereden.
Ze gaan met hunne boerinnen te bier
En zij maken zeer goede sier.
De bezem steekt ten venster uit :
\endverse

\beginchorus
\textbf{Refrain}
Men danst er, men speelt er al op de fluit,
Op potten en pannen
Op glazen en kannen,
Op allerhande geluid;
Op messen, op schup, en op zoutevat,
Op hangel, op tangel, op dit en op dat,
Op trommeltje, rom dom domme dom dom;
Op keteltjes, lepeltjes, tikke tik tang,
En dat gaat zo de helen dag lang.
\endchorus

\beginverse
De boerkes hebben het aards paradijs
Door Adam verloren, hervonden.
Zij roeren de lepel als was het om prijs,
In de rijstpap die hemelse spijs.
De jonkheid kiest een liefje uit :
\endverse
\endsong