% !TEX encoding = UTF-8 Unicode
\beginsong (Le Boulevard Anspach)\footnotemark [
ititle = {Boulevard Anspach (Le)}
tu = {Les Champs-Elysées (Joe Dassin)}]

\footnotetext {IIème Festival de la chanson estudiantine CP ULB, 1976. (Paul Hanson, ULB)}

\beginchorus
\textbf {Refrain}
Au boul'vard Anspach
On était tous zat !
On est entré dans un café,
On a poté tout' la soirée,
Y a que les poils de l'ULB
Pour rigoler !
\endchorus

\beginverse
C'est le lendemain d'un' guindaille
Et j'ai un' fameus' gueul' de bois
Je sens bien que ma voix déraille
Et je n' march' plus droit.
J'avais vu la veill' un' bleuette
Elle avait sa penn' sur la tête
Son étoil' brillait dans le noir
Au bout du comptoir.
\endverse

\beginverse
Je m' dirigeai alors vers elle
Et lui dis d'un ton pas très net :
" Veux-tu que je t' montre, ma belle
Mes grosses roupettes. "
Elle me répond d'un air câlin :
" T' es un coquin mais moi j' veux bien. "
Pour rejoindre la rue aux Laines,
On a eu d' la peine.
\endverse

\beginverse
C'est alors que commenc' le drame
Je me sentis mal dans le tram,
Si bien que, chez ma dulcinée,
J'ai pas pu bander.
Comme j'ai plus pu, elle a trop bu
Elle est descendue dans la rue
Et elle a montré son gros pet
A tous les kets.
\endverse

\beginverse
Malheureus'ment, y avait un flic
Qui lui a proposé l' pic-nic
Avec des autres rigolos
A l' "Amigo".
J'ai été la rejoindr' là-bas
En pensant la récupérer
C'est là que j' sus qu'ils n'aimaient pas
Les poils d' l' ULB.
\endverse

\beginverse
Ils m'ont demandé de souffler
Dans un ballon très coloré
Et se sont aperçus soudain
Qu' j'étais pas à jeûn
Ils ont dit on va mettr' ce klûte
Dans un cachot avec les putes
Je me souviendrai d' cett' guindaille
Aïe, aïe, aïe, aïe.
\endverse

\endsong