\beginsong {Ca, ça, qu'on boive}\footnotemark 
[ititle={Ca, ça, qu'on boive},tu={Edite, bibite, collegiales !}]
\footnotetext{Version française de Edite, bibite, collegiales ! (XVIe siècle), chant d'étudiants germano-latins. Pour plus d'information sur ce sujet, consulter le "Codex Studiosorum Latino-Gallicus" (1986, Ordo Vagorum) à la page 56.}

\beginverse
Ça, ça, qu'on boive,
Fi des gosiers restés poltrons !
Qu'on ne reçoive
Que biberons.
\endverse
\beginchorus
\textbf{Refrain}
Edite, bibite, collegiales !
\ter{Post multa sæcula, pocula nulla !}
\endverse
\beginverse
Ça, ça, qu'on verse
Bon professeur qui n'a pas cours,
Vin mis en perce
Vaut ton discours !
\endverse
\beginverse
Ça, ça, sans trêve,
Gai compagnon, boire ton droit,
Toujours est brève
L'heure où l'on boit.
\endverse
\beginverse
Ça, ça, qu'un trône
Soit, à Bacchus, dressé par nous,
Vider la tonne
Nous emplit tous.
\endverse
\beginverse
Ça, ça, la vie
Ne devient dure qu'à l'instant
Où l'on renie
D'êtr' étudiant.
\endverse
\endsong
