\beginsong{CHANSON A BOIRE}\footnotemark [
ititle={Chanson à boire}]

\footnotetext{P. : Gabriel Bataille (1615).}

\beginverse
Qui eut chasser une migraine
N'a qu'à boire toujours du bon
Et maintenir sa table pleine
De cervelas et de jambons.
\endverse

\beginchorus
\textbf{Refrain}
\bistrois{{L'eau ne fait rien que pourrir le poumon,},
	{Boute, boute, boute, boute compagnon :},
	{Vide-nous ce verre et nous le remplirons.}}
\endchorus

\beginverse
Le vin gousté par ce bon père
Qui s'en rendit si bon garçon
Nous fait discourir sans grammaire
Et nous rend savants sans leçon.
\endverse

\beginverse
Loth buvant dans une caverne.
De ses deux filles enfla le sein.
Montrant que sirop de taverne,
Passe celui d'un médecin.
\endverse

\beginverse
Buvons donc tous à la bonne heure 
Pour nous émouvoir le rognon
Et que celui d'entre nous meure
Qui dédira son compagnon.
\endverse
\endsong