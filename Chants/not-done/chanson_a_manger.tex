\beginsong {Chanson à manger}\footnotemark 
[ititle={Chanson à manger}]

\footnotetext{Auteur : Scarron (XVIIème siècle) qui se singularisa par ces couplets qui se trouvent dans l'Anthologie Française (1765) (Source : Chansons insolites - Guy Breton - Disques Omega - 143.022 - 1978).}

\beginverse
Quand j'ai bien faim, et que je mange,
Et que j'ai bien de quoi choisir,
Je ressens autant de plaisir
Qu'à gratter ce qui me démange.
Cher ami, tu m'y fais songer :
Chacun fait des chansons à boire ;
Et moi, qui n'ai plus rien de bon que la mâchoire,
Je n'en veux faire qu'à manger.
\endverse
\beginverse
Quand on se gorge d'un potage
succulent comme un consommé,
Si notre corps en est charmé,
Notre âme l'est bien davantage.
Aussi Satan, le faux glouton,
Pour tenter la femme première
N'alla pas lui montrer du vin ou de la bière
Mais de quoi branler le menton.
\endverse
\beginverse
Quatre fois, l'homme de courage
En un jour peut manger son saoul
Le trop boire peut faire un fou
De la personne la plus sage.
A-t-on vidé mille tonneaux
On n'a bu que la même chose
Au lieu qu'en un repas, on peut doubler la dose
De mille et différentes façons.
\endverse
\endsong

