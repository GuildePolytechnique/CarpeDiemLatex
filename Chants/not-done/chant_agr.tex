% !TEX encoding = UTF-8 Unicode
\beginsong {Chant de Ad Gildam Romanum\footnotemark} [
ititle= {Chant de l'AGR},
tu= {Lolotte}]

\footnotetext {L'Ad Gildam Romanum (2015) est la guilde du cercle ISTI.}

\beginverse
J'étais istien et passionné de Rome,
Une bonne bière et un carpe en main 
Ô citadelle connue de tous les hommes 
Fais se lever notre pavois romain 
Jeunes héritiers de la Louve et de Bacchus
Nous offrent chaque jour une belle façon de jouir 
\bisdeux {Gloria in Roma et cara Bruxellis nostra}{Nos benedicunt in nomine virtutis}
\endverse

\beginverse
La galea, c'est ainsi qu'on la nomme,
Symbole de l'If qui a soif de gaieté 
Ou soldat ivre, la bonhomie se révolte,
C'est la guindaille qui nous a dépravés 
Elle nous unit tous contre l'ignorance
Revêtue d'or, de carmin et de roy 
\bisdeux {Gloria in Roma et cara Bruxellis nostra}{nos benedicunt in nomine virtutis}
\endverse

\beginverse
Triste zoïle de la voir invaincue 
La guilde Romaine qui fait notre fierté 
Chantons ensemble amis de la patrie 
Nos voix résonnent pour la fraternité 
Imperator, et membres de la guilde
Faites que l'on clame \emph {(crié)} « Ad Gildam Romanum » 
\bisdeux {Gloria in Roma et cara Bruxellis nostra }{nos benedicunt in nomine virtutis}
\endverse

\endsong