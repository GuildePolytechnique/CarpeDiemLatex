% !TEX encoding = UTF-8 Unicode
\beginsong {Chant de Belles-Lettres Bruxelles (BLB!)\footnotemark} [
ititle= {Chant de Belles-Lettres Bruxelles (BLB!)},
tu= {Roulez, tambours!}]

\footnotetext { BLB (2014) est la branche belge d'une société estudiantine suisse (1806) qui réunit les amoureux des arts, des lettres et de la gaité.}

\beginverse
Bellétriens, plus de deux cent années
Ont vu grandir l'arbre que nous aimons ; 
Sous ses rameaux que d'heure fortunées. 
Que de beaux soirs, que de folles chansons !

Il a nargué mainte tourmente,
Il a bravé plus d'un hiver !
\bisdeux {Célébrez l'arbre que je chante :} {\bis {Le sapin vert !}}
\endverse

\beginverse
Bellétriens, à Lausane, à Genève,
A Neuchâtel, nous l'avons vu grandir ; 
Sous le ciel bleu son beau front qui s'élève 
Par nos bons soins n'a cessé de verdir

Il est à nous, l'arbre vivace,
Et nous n'avons jamais souffert
\bisdeux {Qu'un autre arbre en splendeur efface} {\bis {Le sapin vert !}}
\endverse

\beginverse
Bellétriens, voyez nos nouveaux frères
De Bruxelles qui porte haut nos couleurs !
Le vin, la bière offrent un terreau prospère 
Accueillant l'arbre que chérissent nos coeurs !

La gaité réunit les masses
Venez nombreux autour d'un verre
\bisdeux {Voir chaque année sur la Grand' Place} {\bis {Le sapin vert !}}
\endverse

\beginverse
Bellétriens, il a ses adversaires :
A leurs assauts répondons fièrement ; 
Défendons-le de ces mains téméraires, 
Veillons toujours sur notre arbre romand !

Devant ce noble et doux emblème,
Buvons à notre arbre si cher !
\bisdeux {Pour qu'il prospère, 'faut pisser contre} {\bis {Le sapin vert !}}
\endverse

\endsong