% !TEX encoding = UTF-8 Unicode
\beginsong {Le chant des étudiants luxembourgeois} [
ititle= {Chant des étudiants luxembourgeois}]

\beginverse
O Luxembourgeois! O terre maternelle!
nous, les enfants, au seuil de l'avenir,
Nous te jurons un amour éternel,
Dans notre coeur et notre souvenir.
C'est un serment qu'au nom de la jeuness
Nous te jurons d'une commune voix,
Accepte-le, crois en notre promesse,
\bis {Il n'est ici que des Luxembourgeois.}
\endverse

\beginverse
Unissons-nous pour chanter la patrie,
A ce banquet de la fraternité,
Scellons gaiement notre union chérie,
Trinquons, amis, à sa prospérité.
Autour de nous, lorsque le vin pétille
Humiliant nos raisons sous ses lois,
Rions, chantons et buvons en famille,
\bis {Il n'est ici que des Luxembourgeois.}
\endverse

\beginverse
Des souverains les volontés altières,
Jusqu'aujourd'hui nous séparent en vain
Et par dessus d'impuissantes frontières
En souriant, nous nous tendons la main;
De nos aïeux, morcelez l'héritage,
Divisez-nous en deux peuples,ô rois,
Notre amitié se rit de vos partages,
\bis {Il n'est ici que des Luxembourgeois.}
\endverse

\beginverse
De l'union renouvelons les gages,
Serrons les rangs, Arlonnais et Wallons;
Bien que parlant de différents langages,
C'est par le coeur que nous nous rassemblons.
Toujours amis, nous saurons nous comprendre,
Nos coeurs du moins ne parlent qu'un patois;
Toujours amis, pour s'aimer et s'entendre,
\bis {Il n'est ici que des Luxembourgeois.}
\endverse

\beginverse
Versez à flot ce généreux Moselle,
Jeunes buveurs plus généreux encore;
Qu'à larges flots, il pétille et ruiselle,
Ce jus divin qui brille comme l'or;
Autour de lui nous célébrons nos veilles,
Ces gais festins qui font peur aux bourgeois,
On peut le voir au nombre de bouteilles,
\bis {Il n'est ici que des Luxembourgeois.}
\endverse

\endsong

