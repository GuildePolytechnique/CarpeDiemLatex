% !TEX encoding = UTF-8 Unicode
\beginsong {Le chant des partisans\footnotemark}[
ititle= {Chant des partisans (Le)}]

\footnotetext {Hymne de la Résistance française durant l?occupation par l'Allemagne nazie, pendant la Seconde Guerre mondiale. La musique, initialement composée en 1941 sur un texte russe, est due à la Française Anna Marly ancienne émigrée russe. Les paroles originales en français ont ensuite été écrites en 1943 par Joseph Kessel, également d'origine russe, et son neveu Maurice Druon qui venaient tous deux de rejoindre les Forces françaises libres.)

\beginverse 
Ami, entends-tu le vol noir des corbeaux sur nos plaines?
Ami, entends-tu les cris sourds du pays qu'on enchaîne?
Ohé, partisans, ouvriers et paysans, c?est l?alarme.
Ce soir l?ennemi connaîtra le prix du sang et les larmes.
\endverse

\beginverse
Montez de la mine, descendez des collines, camarades!
Sortez de la paille les fusils, la mitraille, les grenades.
Ohé, les tueurs à la balle et au couteau, tuez vite!
Ohé, saboteur, attention à ton fardeau : dynamite...
\endverse

\beginverse
C'est nous qui brisons les barreaux des prisons pour nos frères.
La haine à nos trousses et la faim qui nous pousse, la misère.
Il y a des pays où les gens au creux des lits font des rêves.
Ici, nous, vois-tu, nous on marche et nous on tue, nous on crève...
\endverse

\beginverse
Ici chacun sait ce qu'il veut, ce qu'il fait quand il passe.
Ami, si tu tombes un ami sort de l'ombre à ta place.
Demain du sang noir sèchera au grand soleil sur les routes.
Sifflez, compagnons, dans la nuit la liberté nous écoute...
\endverse

\beginverse
Ami, entends-tu le vol noir des corbeaux sur nos plaines?
Ami, entends-tu les cris sourds du pays qu'on enchaîne?
Oh oh oh oh oh oh oh oh oh oh oh oh oh oh oh oh...
\endverse

\endsong