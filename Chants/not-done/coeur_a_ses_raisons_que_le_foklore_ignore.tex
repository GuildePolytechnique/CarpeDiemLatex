% !TEX encoding = UTF-8 Unicode
\beginsong {Le coeur a ses raisons que le folklore ignore\footnotemark} [
ititle= {Coeur a ses raisons que le folklore ignore (Le)},
tu = {Education sentimentale (Maxime Le Forestier}]

\footnotetext {XLème Festival de la chanson estudiantine CP ULB 2014 (Sarah Cogels et le Cercle des Sciences}

\beginverse
Il est dix-huit heures 
Je vais tout rêveur
Pour te retrouver
J'irai il le faut bien 
Dans les préfabs du coin 
L'acti va durer.
\endverse

\beginverse
Pour toi, ma bleuette,
J'affonerais sans cesse
Et dans ton carnet
Sur toutes les pages
J'écrirais ma rage
Si je le pouvais
\endverse

\beginverse
Fin de la bleusaille
T'es pennée, canaille,
Félicitations
Pourtant toi sans crainte
Tu fouis mon étreinte
Je me sens bien con
\endverse

\beginverse
A la Jefke plus tard,
Auprès d'un comitard,
Tu t'réfugieras.
Les cheveux emmêlés
L'haleine imbibée
Tu te donneras.
\endverse

\beginverse
Et c'est sans rancune,
Que ma belle plume,
Je vais t'oublier.
Et pour ça, mon coeur,
Je vais sans rancoeur
Aller m'ennivrer
\endverse

\beginverse
C'est bien mon destin
De t'approcher en vain
Et toi t'éloigner
Puis sans le prevoir
D'un fait péremptoire
Te voilà togée
\endverse

\beginverse
Il est dix-huit heures
Je vais tout rêveur
Pour te retrouver
J'irai il le faut bien
Dans les préfabs du coin
L'acti va durer
\endverse

\beginverse
Ô, ma comitarde,
Je me suis fait barde
Pour leur déclamer
Que malgré mon âge,
et mon air si sage,
Je n'peux t'oublier.
\endverse

\endsong