% !TEX encoding = UTF-8 Unicode
\beginsong {La complainte du croisé\footnotemark}[
ititle= {Complainte du croisé},
tu= {Sur l'eau, sur la rivière (Mes souliers sont rouges)}]

\footnotetext {XXVIIIème Festival de la chanson estudiantine CP ULB 2002 (Gens Fraternae Libidinis)}

\beginchorus
Au nom de notre Père, il va vers la lumière.
Il faut que sur la terre il répande son credo.
\endchorus

\beginverse
Ogier fourbit ses armes là-haut dans son château
Laisse-là un cœur en larmes, mets la paix au cachot.
Preux chevalier du Christ s'embarque dans un vaisseau.
\endverse

\beginchorus
\bisdeux {Au nom de notre Père, je vais vers la lumière. } {Il faut que sur la terre je répande mon credo }
\endchorus

\beginverse
Ogier rejoint l'armée formée par les croisés
Sa foi inébranlable les autres s'en sont gaussés
Sont venus chercher pillage, est venu chercher pureté.
\endverse

\beginverse
Dans le port de Venise, entament leur descente.
Chacun pratique son vice : marins, adolescentes.
Dans les lagunes s'enlisent la foi, les idéaux.
\endverse

\beginverse
Sur les cendres de Byzance, les croisés font fortune
Ogier, sur ses croyances, larmoyait à la lune.
Les prêcheurs d'ignorance fossoient ses espérances.
\endverse

\beginverse
La pieuse armée en hardes déferle sur la ville sainte,
Dans une vague de haine, massacre les sarrasins.
Laissant sur le rivage Ogier et ses mirages.
\endverse

\beginverse
Au soir de la croisade, le voile se déchire
Son cœur dénudé cherche habit à vêtir.
Jamais credo ne s'ra chemin de vérité.
\endverse

\beginchorus
\quater {Pour toi, pour ton bonheur, suis ton propre destin. } {Pour toi, pour ton bonheur, ne suis pas les ovins. }
\endchorus
\endsong