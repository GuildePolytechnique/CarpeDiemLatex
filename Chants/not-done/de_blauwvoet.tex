\beginsong{DE BLAUWVOET}\footnotemark [
ititle={Blauwvoet, De}]

\footnotetext{Auteurs : Berten Rodenbach - Emiel Hullebroeck.}

\beginverse
Nu het lied der Vlaamse zonen,
Nu een dreunend kerelslied,
Dat in wilde noordertonen
Uit het diepst ons herten schiet.
\endverse

\beginchorus
\textbf{Refrain}
Ei! Het lied der Vlaamse zonen,
Met zijn wilde noordertonen,
Met het oude Vlaams, Hoezee!
Vliegt de blauwvoet? Storm op zee!
\endchorus

\beginverse
't Wierd gezeid da Vlaand'ren groot was,
Groot scheen in der tijden wolk,
Maar dat Vlaanderland nu dood was,
En het vrije kerelsvolk.
\endverse

\beginverse
Maar dan klonk een stemme krachtig,
Over 't oude Noordzeestrand,
En het stormde groots en machtig,
In dat dode Vlaanderland!
\endverse

\beginverse
En hier staan wij, 't hoofd omhoge,
Vuisten siddrend, kokend bloed;
Vlam in 't herte, vlam in d' oge,
En ons naam ons trillen doet!
\endverse

\beginverse
Van de blonde noordse stranden,
Dwang en buigen ongewend,
Onze vaders herwaarts landden,
Leden, streden, ongetemd.
\endverse

\beginverse
Ja wij zijn der Vlamen zonen,
Sterk van lijve, sterk van ziel,
En wij zoûn nog kunnen tonen,
Hoe de klauw der Klauwaards viel.
\endverse

\beginverse
Weg de bastaards, weg de lauwaards,
Ons behoort het noorderstrand,
Ons de kerels, ons de Klauwaards,
Leve Geus en Vlaanderland!
\endverse
\endsong