\beginsong{DE GILDE VIERT}\footnotemark [
ititle={Gilde viert, de}]

\footnotetext{Auteurs : René De Clercq - Emiel Hullebroeck (1910)}

\beginverse
De gilde viert, de gilde juicht,
Wat zit gij daar en blokt en buigt
Nog over uwe boeken?
De wijsheid ligt maar in de kan;
Die ze elders zoeken wil, die kan,
Doch laat hem, laat hem zoeken.
\endverse

\beginchorus
\textbf{Refrain}
Het beste biertje lust hij niet,
Het liefste liedje sust hem niet,
Het mooiste meisje kust hem niet.
Hoog het glas! Hoog het hart! Hoog het lied!
\endchorus

\beginverse
De beker ruist, de beker schuimt!
Sa makkers, fris en opgeruimd
Het glas aan uwe lippen!
Die op zijn kamer koekeloert,
En, geestversnipprend, dwaashe'en snoert,
Drinkt water als de kippen!
\endverse

\beginverse
Het pijpke dampt in monkelmond,
En spreidt wellustig in het rond
Studentikoze geuren!
Die steeds aan perkamenten kluift,
En perkamenten reuken snuift,
Krijgt perkamenten kleuren!
\endverse

\beginverse
De gilde juicht, de gilde viert!
Hoera! De pet omhoog gezwierd,
En nog eens hard gklonken!
De blokker ligt reeds log en loom,
Gekweld door nare blokkersdroom,
Met droge kelen ronken.
\endverse
\endsong