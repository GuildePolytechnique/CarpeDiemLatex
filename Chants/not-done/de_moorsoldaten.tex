\beginsong{DE MOORSOLDATEN}\footnotemark [
ititle={Moorsoldaten, de}]

\footnotetext{Origine: \emph{Moorsoldatenlied},chant écrit en 1933 par des prisonniers du camp de concentration allemand de Börgemoor. S'étant rapidement propagé hors des murs de celui-ci, il fut traduit dans diverses langues, dont ici la version néérlandophone par D. Van Esbroeck.}

\beginverse
Waarheen wij ook mogen kijken
Zien wij veen en hei rondom,
Vogelzang kan ons niet verblijden,
Bomen staan er kaal en stom.
\endverse

\beginchorus
\textbf{Refrain}
Wij zijn de moorsoldaten,
En zwoegen heelder dagen,
In 't veen.
\endchorus

\beginverse
Heen en weer zo gaan de posten,
Niemand kan er langs voorbij!
Vluchten zou ons het leven kosten,
Prikkeldraad, vier op een rij.
\endverse

\beginverse
Toch zal voor ons ook het uur gaan komen,
't Kan niet eeuwig winter zijn!
Dan roepen en zingen w' in alle tonen:
Land van mij, ge zijt weer vrij.
\endverse
\endsong