\beginsong{DE RUNER VON HAMBORG\footnotemark}[
ititle={Runer von Hamborg (De)}]

\footnotetext{Chanson à hisser en Plattdeutsch parlant des "marchands d'homme" qui vendaient à crédit (la solde
	n'était touchée qu le lendemain du débarquement au port) aux matelots des long-courriers, tout ce dont
	ils avaient besoin pour passer un agréable séjour à terre. Ces matelots se retrouvaient donc très souvent
	endettés (Source: Florilège de la chanson de mer - Jacques Yvart - Editions maritimes & d'Outre-mer -
	page 92 - D88/23528 F - 55/87 - 1988).}

\beginverse
De see geiht hoch, de wind de blast,
Oh, kööm un beer for mi!
Jan Maat, de fleit, is nie verbaast,
Oh, kööm un beer for mi!
\endverse

\beginverse
Reise aus quartier un all'an deck, ...
De Ool de fiert de marssails weg, ...
\endverse

\beginverse
Un wenn wi nu na Hamborg kaamt, ...
Denn süüt man all' de sneiders staan, ...
\endverse

\beginverse
Elias röppt, dor büst du ja, ...
Ik see di nich tom eersten mal, ...
\endverse

\beginverse
Du bruukst gewiss een' neen hoot, ...
If heff weck von neeste mood, ...
\endverse

\beginverse
Un ok gewiss een taschendook, ...
Un 'n neen slips, den bruukst du ok, ...
\endverse

\beginverse
Un ook een beeten seep un tweern, ...
Un denn one pound to'n amuseern, ...
\endverse

\beginverse
Wat is dat een lutjen kööm, ...
Un een zigarn, dat smeckt doch schöön, ...
\endverse

\beginverse
Amuseert wart, dat is mol klor, ...
Wie gaat von bord un schreet Hurroh! ...
\endverse
\endsong
