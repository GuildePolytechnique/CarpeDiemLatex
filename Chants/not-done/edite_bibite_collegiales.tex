\beginsong{EDITE, BIBITE, COLLEGIALES}\footnotemark [
ititle={Edite, bibite, collegiales}]
\footnotetext{Chanson du XVIème siècle. P. : Christian Wilh Kindleben (1871).}

\beginverse
Ca, ca geschmauset,
Lasst uns nicht rappelköpfisch sein!
Wer nicht mithauset,
Der bleibt daheim.
\endverse

\beginchorus
\textbf{Refrain}
Edite, bibite, collegiales!
\ter{Post multa saecula, pocula nulla!}
\endchorus

\beginverse
Der Herr Professor
Liesst heat' kein Kollegium;
Drum ist es besser,
Man trinkt eins 'rum.
\endverse

\beginverse
Trinkt nach Gefallen,
Bis ihr die Finger danach leckt,
Dann hat's uns allen
Recht wohl geschmeckt.
\endverse

\beginverse
Auf, auf, ihr Brüder!
\emph{On lève son verre.}
Erhebt den Bacchus auf den Thron
Uns setzt euch nieder,
Wir trinken schon.
\endverse

\beginverse
So lebt man immer,
So lang der junge Lenz uns blinkt,
Und Jugendschimmer
Die Wangen schminkt.
\endverse

\beginverse
So lebt man lustig,
Weil es noch flotter Bursche heibt,
Bis dass man rüstig
Ad patres reist.
\endverse

\beginverse
Denkt oft, ihr Brüder,
An unsre Jugendfröhlickkeit,
Sie kehrt nie wiedern
Die goldne Zeit!
\endverse
\endsong