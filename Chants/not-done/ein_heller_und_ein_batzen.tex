\beginsong{EIN HELLER UND EIN BATZEN}\footnotemark [
ititle={Heller und ein Batzen, ein}]

\footnotetext{Autre titre : Leichter Wanderer. Chanson populaire de Albrecht, comte de Schlippenbach.}

\beginverse
Ein Heller und dein Batzen, de waren beide mein, ja mein.
\bis{Der Heller ward zu Wasser, de Batzen ward zu Wein, ja Wein;}
\endverse

\beginchorus
\textbf{Refrain}
\ter{Heidi heido heida}
Ha ha ha ha ha ha ha
\ter{Heidi heido heida}
\endchorus

\beginverse
Die Mädel und die Wirtsleut' die dufen beid : o weh! o weh!
\bis{Die Wirtsleut, wenn ich komme, die Mädel wenn ich geh, ja geh;}
\endverse

\beginverse
Mein' Stiefel sind zerrissen, mein Schuh, die sind entzwei, entzwei,
\bis{Und draussen auf der Heide, da singt der Vogel frei, ...}
\endverse

\beginverse
Das war 'ne rechte Freunde, als mir der Herrgott schuf, ja schuf;
\bis{'N Kerl wie Samt und Seide, nur schade, das er suff, ...}
\endverse
\endsong