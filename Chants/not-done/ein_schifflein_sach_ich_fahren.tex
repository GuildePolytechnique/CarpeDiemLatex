\beginsong{EIN SCHIFFLEIN SAH ICH FAHREN}\footnotemark [
ititle={Schifflein sah ich fahren, ein}]

\footnotetext{Mélodie populaire allemande de 1787.}

\beginverse
Ein Schifflein sah ich fahren,
Kapitän und Leutenant;
Darinnen waren geladen
Zwei brave Kompanien Soldaten.
\endverse

\beginchorus
\textbf{Refrain}
Kapitän, Leutenant, Fähndrich, Sergeant,
\bis{Nimm das Mädel} bei der Hand,
\bisdeux{{Soldaten, Kameraden,},
	{\bis{Nimm das Mädel} bei der Hand.}}
\endchorus

\beginverse
Was sollen die Soldaten essen?
Kapitän und Leutenant;
Gebratene Fisch mit Kressen,
Das sollen die Soldaten essen.
\endverse

\beginverse
Was sollen die Soldaten trinken?
Kapitän und Leutenant;
Den besten Wein, der zu finden,
Den sollen die Soldaten trinken.
\endverse

\beginverse
Wo sollen die Soldaten schlafen?
Kapitän und Leutenant;
Bei ihren Gewehr' und Waffen,
Da müssen die Soldaten schlafen.
\endverse

\beginverse
Wo sollen die Soldaten tanzen?
Kapitän und Leutenant;
Vor Harburg auf der Schanzen,
Da müssen die Soldaten tanzen.
\endverse

\beginverse
Wie kommen die Soldaten in den Himmel?
Kapitän und Leutenant;
Auf einem weissen Schimmel,
Da reiten die Soldaten in den Himmel.
\endverse

\beginverse
Wie komm'n die Offiziers in die Höllen?
Kapitän und Leutenant;
Auf einem schwarzen Fohlen,
Da wird sie der Teufel schon holen.
\endverse
\endsong