\beginsong{EINE SEEFAHRT}\footnotemark [
ititle={Seefahrt, eine}]

\footnotetext{Chanson estudiantine allemande.}

\beginverse
Eine Seefahrt die ist lustig
Eine Seefahrt die ist schön,
Ja, da kann man was erleben
Ja, da kann man etwas sehen.
\endverse

\beginchorus
\textbf{Refrain}
\bisdeux{{Hollahi, hollaho},
	{Hollahia hia hia, hia, ho.}}
\endchorus

\beginverse
Einen schrecklich langen Bartsack
Den hat unser Kapitän
Raucht dafür auch starken Tabak
Das man gar nichts mehr kann seh'n.
\endverse

\beginverse
Und der erste Offizier
Ist besoffen wie ein Stier,
Und der zweite Offizier
Trinkt noch immer so viel Bier.
\endverse

\beginverse
Und der Koch in der Kambüse
Diese gottverdammte Sau,
Dauernd spuckt er ins Gemüse,
Manchmal auch in den Kakau.
\endverse

\beginverse
Eine Tonne Öl genommen,
Drin ein kleiner Pinsel ist,
Und ne riesengrosse Schnauze,
Fertig ist der Maschinist.
\endverse

\beginverse
Steh nur auf, du faules Luder,
Steh nur auf, de faules Schwein,
Kohlen willst du kleine trimmen
Aber Heizer willst du sein.
\endverse

\beginverse
Und er haut ihn vor den Dassel
Dass er an die Reling fällt,
Und die heil'gen zwölf Apostel
Für' ne Räuberbande hällt.
\endverse

\beginverse
Und die schönen weissen Möwen
Sie erfüllen ihren Zweck,
Und sie scheissen, scheissen, scheissen,
Auf das frischgewasch'ne Deck.
\endverse

\beginverse
In der Heimat angekommen
Fängt ein neues Leben an,
Eine Frau wird sich genommen
Kinder bringt der Weihnachtsmann.
\endverse
\endsong