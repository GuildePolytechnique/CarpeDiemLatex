\beginsong{FILIA HOSPITALIS} [
ititle={Filia hospitalis}
tu={O wonnevolle Jugendzeit (P. : Otto Kamp, 1885 - M. : O. Lob)}]

\beginverse
O heerlijke studententijd
Met talloze vermaken,
Met minnetochten wijd en zijd,
De schoonsten te genaken,
Wees mij gegroet, O schone jeugd!
Elk aardig meisje baart me vreugd,
\bisdeux{En toch is niets aequalis}
{Aan Filia Hospitalis.}
\endverse

\beginverse
Ik kwam als eerstejaars hier aan
En speurde door de straten
Waar ik een stoel en bed vond staan
Om mij daar neer te laten,
'k Vond luie stoel noch canapé
En toch viel mij de kamer mee
\bisdeux{Want niemand is aequalis}
{Aan Filia Hospitalis.}
\endverse

\beginverse
Het is zo 'n alleraardigst kind
Met zacht blauwe ogen,
De voetjes trippelend gezwind
Komt zij naar mij gevlogen,
Het mondje lacht zo lief zo blij,
Geen tweede komt haar ooit nabij;
\bisdeux{Neen, niemand is aequalis}
{Aan Filia Hospitalis.}
\endverse

\beginverse
Drie huurders heeft ze : de jurist
Bezoekt slechts fijne kringen,
De medicus heeft zich vergist
Als hij zich op wil dringen
Doch mij, slechts de philologus
Gaf zij in eer en deugd een kus
\bisdeux{Daarom is niets aequalis}
{Aan Filia Hospitalis.}
\endverse

\beginverse
O liefste blonde, wist ik toch,
Wat God wil met ons beiden,
Een laatste kusje geef me nog,
Vóór ik van u moet scheiden.
En zijt gij mij niet toebedacht,
Aan u denk ik bij dag en nacht,
\bisdeux{Want niemand is aequalis}
{Aan Filia Hospitalis.}
\endverse
\endsong