\beginsong{GEUZENLIED}\footnotemark [
ititle={Geuzenlied}]

\footnotetext{Auteur : Julius De Geyter (1873)}

\beginverse
Zij brullen : "Leeuw van Vlaanderen!"
En huilen tegen ons
Zij die de Leeuw doen kruipen
Doen kruipen voor Bourbons
O! Breydel en De Coninck,
Gent en Brugge van weleer
Heeft Vlaanderen dan geen kerels 
Hebt gij gen Klauwaerts meer?
Op Geuzen, wreekt uw vaadren
Zwaait gij de Leeuwenvaan!
\bisdeux{Wee, wee den landverraadren}
{Wanneer hun uur zal slaan!}
\endverse

\beginverse
Blikt om u heen, O Broedren!
Trekt gans den wereld rond
Weer rijzen als paleizen
De kloosters uit den grond
Het glanzend licht der rede
Het licht moet uitgedoofd
Voor bedevaart, mirakels
En spoken in het hoofd
Ach! over Leie en Schelde
Hangt zulk een sombere nacht...
\bisdeux{O land van Artevelde}
{De Geuzen houden wacht!}
\endverse

\beginverse
Jezuïeten zaaien tweedracht
Zij blazen haat en twist
Wij juichen "Recht en rede"
Zij grijnsen "Laag en list"
Hoort! Rome smeedt ons ketens
Voor 't lijf en voor de ziel
Het zwart gespuis zal 't mensdom
Verplettren met den hiel
Dan Geuzen, dan te wapen
De Vrijheidsvlag ter hand
\bisdeux{Van et ongediert der papen}
{Verlost ons vaderland!}
\endverse

\beginverse
Wanneer rijst eens het daglicht
In d' aardse rampenwoestijn
Dat elk zijn eigen koning
Zijn eigen Paus zal zijn?
Geen slaven meer aan ketens
Geen ziel aan boei of band
En 't mensdom, gans het mensdom
Een enkel Broederland
Dat willen wij, O Geuzen
Al kost het goed en bloed
\bisdeux{Wij gaan vooruit als reuzen}
{Vooruit met leeuwenmoed!}
\endverse
\endsong