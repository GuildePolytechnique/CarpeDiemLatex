\beginsong {Le Grenadier de Flandre} [
ititle= {Grenadier de Flandre (Le)}]

\beginverse
C'�tait un grenadier
Qui revenait de Flandre\footnote {Le retour de Flandre des grenadiers (1667-1789) repr�senterait la fin de la guerre de Hollande (1678).}
Qu' �tait si mal v�tu
Qu'on lui voyait son membre
\endverse

\beginchorus
\textbf {Refrain}
Le tambour bat
La g�n�rale
La g�n�rale bat
Ne l'entendez-vous pas ?
La g�n�rale bat
Le r�giment s'en va.
\enchorus

\beginverse
Qu' �tait si mal v�tu
Qu'on lui voyait son membre
Un' dam' de charit�
L' fit monter dans sa chambre
\endverse

\beginverse
... Allum' cinq, six fagots
Pour r�chauffer le membre
\endverse

\beginverse
... Quand le membre fut chaud
Il se mit � s'�tendre
\endverse

\beginverse
... Aussi long que le bras
Aussi gros que la jambe
\endverse

\beginverse
... " Dis-moi, beau grenadier
A quoi te sert ce membre ? "
\endverse

\beginverse
... " Il me sert � pisser
Quand l'envie vient m'en prendre
\endverse

\beginverse
... Et aussi � baiser
Quand l'occasion s' pr�sente. "
\endverse

\beginverse
... " Eh bien ! Beau grenadier
Fous-le moi donc dans l' ventre ! "
\endverse

\beginverse
... " Ah ! non, non, non, Madame
J'aurais peur de vous fendre ! "
\endverse

\beginverse
... " Fendue ou pas fendue
Il faut que tout y entre !
\endverse

\beginverse
... S'il en rest' un p'tit bout
Ce s'ra pour la servante
\endverse

\beginverse
... S'il n'en rest' pas du tout
Elle se bross'ra le ventre !
\endverse

\beginverse
... Elle ira dir' partout :
"Madam' est un' gourmande.
\endverse

\beginverse
... Quand y'a d' la viand' chez nous
Elle se fout tout dans l' ventre !"
\endverse

\endsong
