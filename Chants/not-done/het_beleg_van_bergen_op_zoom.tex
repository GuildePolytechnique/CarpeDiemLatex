\beginsong{HET BELEG VAN BERGEN-OP-ZOOM\footnotemark} [
ititle={Beleg van Bergen-op-Zoom, Het}]

\footnotetext{Auteur : Adriaan Valerius (environ 1626).}

\beginverse
Merck toch hoe sterck nu int werck sich al steld,
Die t' allen tijd soo ons vrijheijt heeft bestreden.
Siet hoe hij slaeft, graeft en draeft met geweld
Om onse goet en ons bloet en onse steden!
Hoor de Spaensche trommels slaen!
Hoor Maraens trompetten!
Siet, hoe komt hij trecken aen
Bergen te besetten!
Berg'-op-Zoom, hout u vroom,
Stut de Spaensche scharen :
Laet 's lands boom end' zijn stroom,
Trouw'lijck toch bewaren.
\endverse

\beginverse
't Moedige bloedige woedige swaerd
Blonck en het klonck dat de voncken daer uyt vlogen.
Beving en leving, opgeving der aerd,
Wonder gedonder nu onderwas, nu boven.
Door al 't mijnen en 't geschut,
Dat men daeglijcx hoorde;
Menig Spanjaert in sijn hut,
In sijn bloet versmoorde.
Berg'-op-Zoom, hout sich vroom,
't Stut de Spaensche scharen :
't Heeft 's lands boom end' zijn stroom,
Trouw'lijck doen bewaren.
\endverse

\beginverse
Die van Oranjen quam Spanjen aen boord,
Om uyt het velt, als een helt, 't gewelt te weeren;
Maer also dra Spinola 't heeft gehoord
Treckt hij flox heen op de been met al zijn heeren.
Cordua kruyd spoedig voort,
Sach daer niets te winnen;
Don Velasco liep gestoort,
't Vlas was niet te spinnen.
Berg'-op-Zoom, hout sich vroom,
't Stut de Spaensche scharen :
't Heeft 's lands boom end' zijn stroom,
Trouw'lijck doen bewaren.
\endverse
\endsong