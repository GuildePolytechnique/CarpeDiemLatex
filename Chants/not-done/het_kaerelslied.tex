\beginsong{HET KAERELSLIED}\footnotemark [
ititle={Kaerelslied, het}]

\footnotetext{Auteurs : Julius De Geyter - Florimond Van Duyse.}

\beginverse
Ic wil van den kaerel singhen,
Al met sinen langhen baert;
Hine laet ghenen ruter hem dwinghen;
Ontembaer soes hi vanaert,
Ende of sine cleeder ontnaeit sijn,
Sin hooft mit een hoetkin ghecapt;
Ende mach sin caproen ooc verdraeit syn,
Sin scoen ende cousen ghelapt.
\endverse

\beginchorus
\textbf{Refrain}
Al eti maer wronghelen, caes ende broot;
Al slaepti up stro ende cruut;
Vri als 'et veulen, en kenti gheen noot;
Ende lacht de ruters uut.
\endchorus

\beginverse
Hi esser een vriman gheboren,
Ter see, upt velt of int bos.
Waerom sou gheen plek hem behoren,
Gheen plek vor een paert oft een os?
Die ruter heeft alles ghenomen;
Men loopet, als hi ghebiet.
Bi dunen, bi vlacten, bi stromen,
Dat en doet die kaerel niet!
\endverse

\beginverse
Hine wert niet den ruter een slave,
Ter kermesse blyfti gaen,
Met ghespen, knijf ende stave,
Sin wyf met een mantelkin aen.
Daer pypen die cornemusen:
Daer springhen ende dansen si ront.
Daer swieren si lanx die husen;
Daer singhen stuut luder mont:
\endverse

\beginverse
Sine wyf is een vroudighe daerne;
Hi ploeght, ende si, si spint;
Hi cust ende blust se so gaerne:
So winnen si kint up kint.
Ter kermesse drinken si wine,
Want wine verhoghet verstant;
Nu esser die waerelt de sine,
Met steden, casteel ende lant.
\endverse

\beginverse
Si willen den kaerel doen greinsen,
Al dravende over 't velt;
Hi es coener dan si wael peinsen;
Hine bevet voor gheen ghewelt.
Si willen hem sleepen ende hangen;
Te lanc es haer sin baert.
O ruter, als hi u moet vanghen,
Wi weter hoe ghi dan vaert!
\endverse
\endsong