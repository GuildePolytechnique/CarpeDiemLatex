\beginsong{HET LIED VAN HERTOG JAN}\footnotemark [
ititle={Lied van Hertog Jan, het}]

\footnotetext{Auteurs : H. Beers - Fl. Van De Putt. Il est fait référence au Duc de Brabant du XIIIème siècle.}

\beginverse
Toen den Hertog Jan kwam varen
Te peerd parmant, al triumfant
Na zevenhonderd jaren
Hoe zong men 't allen kant :
Harba lorifa, zong den Hertog,
Harba lorifa!
Na zevenhonderd jaren
In dit edel Brabants Land.
\endverse

\beginverse
Hij kwam van over 't water :
Den scheldevloed, aan wal te voet,
't Antwerpen op de straten
Zilver veren op zijn hoed :
Harba ...
't Antwerpen op de straten
Lere leerzen aan zijn voet.
\endverse

\beginverse
Och Turnhout, stedeke schone,
Zijn uw ruitjes groen, maar uw hertjens koen :
Laat den Hertog binnenkomen
In dit zomers vrolijk seizoen.
Harba ...
Laat den Hertog binnenkomen;
hij heeft een peerd van doen.
\endverse

\beginverse
Hij heeft een peerd gekregen,
Een schoon wit peerd, een schimmelpeerd.
Daar is hij opgestegen,
Dien ridder onverveerd.
Harba ...
Daar is hij opgestegen
En hij reed naar Valkensweerd.
\endverse

\beginverse
In Valkensweerd daar zaten,
Al in de kast, de zilverkast,
De gulde-koning zijn platen,
Die werden aaneen gelast.
Harba ...
De guilde-koning zijn platen,
Toen had hij een harnas.
\endverse

\beginverse
Rooise boeren, komt naar buiten;
Met de grote trom, met de kleine trom,
Trompetten en cornetten ende fluiten,
Want den Hertog komt weerom.
Harba ...
Trompetten en cornetten ende fluiten
In dit Brabants Hertogdom.
\endverse

\beginverse
Wij reden allemaal samen
Op Oischot aan, door een kanidasselaan,
En Jan riep : "In Gods name!
Hier heb ik méér gestaan."
Harba ...
En Jan riep : "In Gods name!
Reikt mij mijn standaard aan!"
\endverse

\beginverse
De standaard was de gouwe :
Die waaide dan, die draaide dan
Die droeg de leeuw mee klauwen,
Wij zongen alle man :
Harba ...
Die droeg de leeuw mee klauwen,
Ja, de leeuw van Hertog Jan!
\endverse

\beginverse
Hij is in Den Bosch gekomen
Al in den nacht, niemand die zag 't.
En op de Sint Jan geklommen
Daar ging hij staan op wacht!
Harba ...
En op de Sint Jan geklommen
Daar staat hij dag en nacht!
\endverse
\endsong