\beginsong{HET LIED VAN KOPPELSTOCK}\footnotemark [
ititle={Lied van Koppelstock, het}]

\footnotemark{Auteur : A. Schooleman. Autre titre : De Veerman.}

\beginverse
In naam van Oranje, doet open de poort!
De Watergeus ligt aan de wal :
De vlootvoogd der Geuzen, hij maakt geen akkoord,
Hij vordert Den Briel of uw val
Dat is het bevel van Lumey op mijn eer
En burgers, hier baat nu geen tegenstand meer,
\bis{De Watergeus komt om Den Briel!}
\endverse

\beginverse
De vloot is met vijfduizend koppen bemand,
De mannen zijn kloek en vol vuur.
Een ogenblik nog en zij stappen aan land,
Zij wachten bericht binnen 't uur;
Gij moogt dus niet dralen, doet open de poort,
Dan nemen de Geuzen terstond zonder moord
\bis{Bezit van de vesting Den Briel!}
\endverse

\beginverse
Komt, geeft de verzek'ring, 'k moet spoedig terug,
De klok heeft het uur reeds gemeld.
Ik zeg 't u, geeft gij mij de sleutels niet vlug,
Dan is reeds uw vonnis geveld.
De wakkere Geuzen staan tandenknarsend klaar.
Zij wetten hun zwaarden en maken zich klaar,
\bis{En zweren : "Den dood of Den Briel!"}
\endverse

\beginverse
Hier dringt men naar buiten, daar schuilt men bijeen
En spreekt over Koppelstocks last :
"De stad in hun handen of anders de dood".
't Besluit tot her eerste staat vast!
Maar nauw'lijks is hiermee de veerman gevleid,
Of Simon de Rijk heeft de poort gerammeid
\bis{En zo kwam de Geus in Den Briel!}
\endverse
\endsong