\beginsong{HET LOZE VISSERTJE}\footnotemark [
ititle={Loze vissertje, het}]

\footnotetext{Chanson gantoise du XVIIIème siècle.}

\beginverse
Des winters als het regent,
Dan zijn de paadjes diep, (ja diep)
Dan komt dat loze vissertje
Vissen al in het riet. (ja riet)
\endverse

\beginchorus
\textbf{Refrain}
Met zijnen rijfstok,
Met zijnen strijkstok,
Met zijnen lapzak,
Met zijnen knapzak,
\bistrois{Met zijne lere van,}
{Dire domme dere,}
{Met zijne lere laarsjes aan.}
\endchorus

\beginverse
Dat loze molenarinnetje
Ging in heur deurtje staan, (ja staan)
Opdat dat aardig vissertje
Voorbij haar heen zou gaan. (ja gaan)
\endverse

\beginverse
"Wat heb ik u misdreven,
Wat heb ik u misdaan, (ja daan)
Opdat ik niet met vrede
Voorbij uw deur mag gaan," (ja gaan)
\endverse

\emph{Remplacer au refrain "zijnen" par "mijnen".}

\beginverse
"Gij hebt mij niets misdreven,
Gij hebt mij niets misdaan, (ja daan)
Maar moet mij driemaal zoenen,
Eer gij van hier moogt gaan." (ja gaan)
\endverse

\emph{Remplacer au refrain "zijnen" par "uwen".}
\endsong