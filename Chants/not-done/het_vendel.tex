\beginsong{HET VENDEL}\footnotemark [
ititle={Vendel, het}]

\footnotetext{P. : Jan Hoogensteyn - M. : inconnu (vers 1560).}

\beginverse
Het vendel moet marsjeren
Want Vlaand'ren is in nood
Sint Joris geef on kleren,
Geef ons soldij en brood.
Dat wij geen koude lijden,
Geef ons de boer zijn wijn,
Zijn wolhemd en zijn duiten,
Dat kan geen zonde zijn!
Marsjeer, landsknecht, marsjeer!
\endverse

\beginverse
Wij slikken stof bij 't wand'len,
Verstomd zijn lied en lach;
De Keizer slikt heel Vlaand'ren,
Hij heeft een sterke maag!
Hij denkt al onder 't kauwen
Aan nieuwe roem en eer,
Thuis weent een blonde vrouwe
Als ik niet wederkeer.
Marsjeer, landsknecht, marsjeer!
\endverse

\beginverse
De tamboer slaat parade
Sint Joris, sterke held,
Bescherm ons in genade,
het vendel trekt te veld.
De pijper wil niet fluiten,
Wij trekken stil en stom
Over die groene heide
Opwaarts naar Berg-op-Zoom.
Marsjeer, landsknecht, marsjeer!
\endverse
\endsong