\beginsong {Le hussard de la garde}\footnotemark [
ititle= {Hussard de la garde (Le)}
ititle= {Vivre sans souci}
ititle= {Manon}]

\beginverse
C'�tait un hussard de la garde
Qui revenait de garnison,
De Brian�on.
Portant sa pin' en hallebarde :
Agr�ment�e de deux roustons
Pleins de morpions.
\endverse

\beginchorus
\textbf {Refrain}
Vivre sans soucis,\footnote {Les quatre derniers vers du refrain sont la version scatologique actuelle du refrain d'un cantique : Go�tez, �mes ferventes.}
Boir' du purin, manger d' la merde,
C'est le seul moyen
De ne jamais crever de faim.
� merde, merde divine !
Toi seule a des appas.
La rose a des �pines,
Toi, merde, tu n'en as pas.
\endchorus

\endverse
En descendant la rue Trouss'couille,
Il rencontra la garc' Manon
Qui pue du con.
Il lui dit : " Ma chaste vadrouille,
Le r�giment s'en va demain
La pin' en main. "
\endverse

\beginverse
En vain Manon se d�sesp�re
De voir partir tous ses amis
Avec leurs vits.
Elle va trouver Madam' sa M�re,
Lui dit : " Je veux partir aussi
Sacr�e chipie. "
\endverse

\beginverse
" Ma fill', ma sacr�e garc' de fille,
N' vas pas avec ce hussard-l�,
Il te perdra !
Ils t'ont fendue jusqu'au nombril(e),
Ils te fendraient jusqu'au menton
La peau de con. "
\endverse

\beginverse
" Ma fill', ma sacr�e garc' de fille,
Quand s'ra parti ce hussard-l�,
Tu te branl'ras.
Je t'ach�t'rai une cheville
Avec laquell' tu t' masturb'ras
A tour de bras. "
\endverse

\beginverse
" Ma M�r', mon vieux chameau de m�re,
Quand tu parles de me branler,
Tu m' fais chi-er.
Un vit, �a sort de l'ordinaire,
�a vous laiss' un doux souvenir
Qui fait jou-ir. "
\endverse

\beginverse
La garc' s'est quand m�m' laiss�e faire
Par le hussard qui la pressait
De se donner.
Il lui mit un' si longu' affaire,
Que �a ressortait par le nez ;
�a l'a tu�e.
\endverse

\beginverse
Manon, la sacr�e garc' est morte.
Morte comm' elle avait v�cu :
La pin' au cul.
Le corbillard est � sa porte,
Tra�n� par quatr' morpions en deuil,
La larm' � l'oeil.
\endverse

\beginverse
Ils l'ont conduite au cimeti�re
Et sur sa tombe, ils ont grav�
Tous ces couplets.
Mais le fossoyeur, par derri�re,
L'a d�terr�e et l'a viol�e,
�a lui manquait.
\endverse

\beginverse
L'auteur de cette barcarolle
Est un bon hussard � chevrons,
Foutu cochon !
Quand il mourut de la v�role,
Les asticots qui l'ont bouff�
Ont d�gueul�.
\endverse
\endsong