\beginsong{IM TIEFEN KELLER}\footnotemark [
ititle={Im tiefen Keller}]

\footnotetext{P. : Karl Müchler, 1802 - M. : Ludwig Fischer.}

\beginverse
Im tiefen Keller sitz' ich hier
Bei einem Fass voll Reben
Bin guten Muts und lasse mir
Vom Allerbesten geben.
Der Küfer zieht den Heber vor,
Gehorsam meinem Winke,
Füllt mir das Glas, ich halt's empor
Und trinke, trinke, trinke, ...
\endverse

\beginverse
Mich plagt ein Dämon, Durst genannt,
Doch um ihn zu verscheuchen,
Nehm' ich mein Deckelglas zur Hand,
Und lass mir Rheinwein reichen.
Die ganze Welt erscheint mir nun
Im rosaroter Schminke,
Ich könnte niemand Leides tun,
Und trinke, trinke, trinke, ...
\endverse

\beginverse
Allein mein Durst vermehrt sich nur,
Bei jedem frischen Becher;
Das ist die leidige Natur
Der alten Rheinweinzecher!
Doch tröst' ich mich, wenn ich zuletzt
Vom Fass zu Boden sinke :
Ich hab' keine Pflicht verletzt,
Ich trinke, trinke, trinke, ...
\endverse
\endsong