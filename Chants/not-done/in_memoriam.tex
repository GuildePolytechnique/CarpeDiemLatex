\beginsong{IN MEMORIAM}\footnotemark [
ititle={In memoriam},
tu={De rolders in de nacht}]

\footnotetext{XIIIème festival de la chanson estudiantine ULB-CP, 1987 (Guide Polytechnique)}

\beginverse
Amis, frères et compagnons écoutez-moi
Je m'en vais vous raconter avec émoi
Sous la forme d'une chanson la belle histoire
De sept étudiants qui portaient la penne noire.
\endverse

\beginchorus
\textbf{Refrain}
\bis{Oh oh oh oh oh oh oh}
\endchorus

\beginverse
À l'époque, ce n'était vraiment pas aisé
Pour celui qui voulait apprendre à chanter
Le CP magouillait avec la Médecine
Et chez eux le chant n'avait pas bonne mine.
\endverse

\beginverse
Il y avait Thierry, Stéphane et puis le Jo,
Le Jack et le Ben, Michel et puis le Gauss
Pour qui le folklore ne pouvait pas crever,
Ils se sont réunis pour le ranimer.
\endverse

\beginverse
Sachez donc, amis, que la Guilde s'est créée
Pour qu'enfin en Polytech' on sache chanter;
Rouge et vert, noir et blanc sur son étendard
Sont les couleurs qui nous unissent ce soir.
\endverse

\beginverse
Par l'organisation de nombreux cantus
La Guilde se fit connaître de plus en plus
Au nom du compas, du marteau et du pic,
Elle rendit la voix à la Polytechnique.
\endverse

\beginverse
Que l'on clache, que l'on tonde ou bien que l'on rase,
Je vous invite à méditer cette phrase :
Entre les cercles trop souvent séparés,
La chanson reste un lien, fut-il le dernier…
\endverse

\beginverse
Levez donc votre verre rempli jusqu'au bord;
Buvez-le à la chanson et au folklore,
À la liberté, à la fraternité
Et bien sûr à la santé de l'ULB !
\endverse
\endsong