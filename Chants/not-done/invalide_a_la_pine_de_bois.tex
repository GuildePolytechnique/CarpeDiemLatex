\beginsong {L'invalide � la pine de bois} \footnotemark [
ititle= {Invalide � la pine de bois (L')}]

\footnotetext { Chanson sans doute inspir�e par une fantaisie d'Eug�ne Mouton, Histoire de l'invalide � la t�te de bois (1887) : �lucubrations d'un soldat picard � qui l'on a greff� une t�te en sapin d'Alsace apr�s une bataille contre les Turcs}
\beginverse
Je viens d' voir, c'est un vrai prodige,
Enfonc�s, les fr�r's Siamois,
Je viens d' voir, j'en ai le vertige,
L'invalide � la pin' de bois.
Un homme dont la pin' se d�visse,
Et qui se fout des morpi-ons,
De la v�role, de la chaude-pisse.
C' qui l'emmerde, c'est les bubons.
\endverse

\beginchorus
\textbf {Refrain}
Il faut le voir pour le croire
\bis{Venez donc y voir.}
Il vous �pat'ra bourgeois,
\bis{L'invalide � la pin' de bois.}
\endchorus

\beginverse
Il faut dir' que cet homm' �trange
Est muni de plusieurs �tuis,
Contenant des pin's de rechange
En bois de diff�rents pays.
De sa campagne d'Italie,
Ce brave et vaillant guerrier
A rapport� la plus jolie,
La pin' en bois de laurier.
\endverse

\beginverse
Quand il a cell' en bois d' ch�ne
De dix coups il port' le fardeau.
Quand il a cell' en bois d'�b�ne
Il bais' comm' un moricaud.
Il encule comm' un Kabyle
Quand il a cell' en palmier.
Il bais' comm' un imb�cile
Quand il a cell' en olivier.
\endverse

\beginverse
Quand il a cell' en bois d' charme
Aucun' femm' n' peut lui r�sister.
On le voit bander comm' un carme
Quand il a cell' en poivri-er.
Mais voil� son plus grand vice :
D�s qu'il voit une femm' tousser,
Il met sa pin' en bois d' r�glisse,
Que vit' il va lui fair' sucer.
\endverse

\beginverse
Avec son �tui fid�le,
Il peut toujours se contenter
Veut-il enfoncer un' pucelle,
Il met sa pin' en oranger.
S'il vient � tomber malade,
Il peut lui-m�me se soigner,
Car il piss' de la limonade
Avec sa pin' en citronnier.
\endverse

\endsong