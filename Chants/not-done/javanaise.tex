\beginsong {La Javanaise} \footnotemark[
ititle= {Javanaise (La)}
ititle= {B�bert}
tu= {La java (interpr�t�e par Arletti)}]

\footnotetext {Autre titre : B�bert.}

\beginverse
Quand pour la premi�r' fois
Julot encula
Une Javanaise
Il sentit sur son doigt
Quelque chos' de gras
Comm' d' la mayonnaise
Son con �tait si long,
Si larg' et profond
Si plein de liquide
Qu'il avait l'impression
Que son saucisson
Nageait dans le vide.
\endverse

\beginchorus
\textbf {Refrain}
C'est la java, la bite � papa
Les couilles � Julot.
Sa p'tit' casquett' ses grosses roupettes
Et son p'tit m�got.
Viens mon landru, mon tordu
Fous-la moi dans l' cul ;
Viens mon tr�sor, mon Nestor
Pouss' un peu plus fort.
\endchorus

\beginverse
Mon p�re �tait branleur,
Astiqueur de bites
Dans un bal musette.
Ma m�re �tait putain,
Faisait des pompiers
A tous ceux d' l'orchestre
Non ! Tu ne verras plus
Les poils de mon cul
J'en ai fait des brosses
A vingt francs du kilo-,
C'est du bon boulot
Pour nourrir les gosses.
\endverse

\endsong