\beginsong {Je cherche fortune}\footnotemark [
ititle= {Je cherche fortune}]

\footnotetext {Aristide Bruant d�butera au "Chat Noir" de Rodolphe Salis avec cette chanson, en 1881, mis � part le fait que seule la deuxi�me moiti� du refrain fait encore partie de la version de Aristide Bruant.}

\beginverse
\bis{Chez l' boulanger}
\bis {Fais-moi cr�dit,}
\bis {J' n'ai plus d'argent,}
\bis {J' paierai sam'di. }
\bis {Si tu n' veux pas }
\bis {M' donner du pain,}
\bis {J' te cass' la gueule }
\bis {Dans ton p�trin. }
\endverse

\beginchorus
\textbf {Refrain}
Non, c'est pas moi, c'est ma soeur
Qu' a cass� la machin' � vapeur
\ter {Ta gueule.}
Je cherche fortune
Tout autour du Chat Noir
Au clair de la lune
A Montmartre, le soir.
\endchorus

\beginverse
\bis {Chez l' marchand d' frites \footnote {Sauf avis contraire, tous les couplets sur le mode du premier.}
\bis {... M' donner des frites}
\bis {... Dans tes marmites }
\endverse

\beginverse
\bis {Chez l' cabar'tier }
\bis {... M' donner � boire }
\bis {... Sur ton comptoir.}
\endverse

\beginverse
\bis {Marchand d' tabac}
\bis {... M' donner des s�ches }
\bis {J' fais dans ta gueule}
\bis {Une large br�che. }
\endverse

\beginverse
\bis {Chez la putain }
\bis {... Baiser � l'oeil }
\bis {.. Dans ton fauteuil.}
\endverse

\beginverse
\bis {Chez l'aut' putain }
\bis {... M' pr�ter ton con}
\bis {J' te bouff' le cul }
\bis {Et les nichons. }
\endverse

\beginverse
\bis {Chez l'aubergiste }
\bis {... M' donner un' chambre}
\bis {... Et les cinq membres. }
\endverse

\beginverse
\bis {Chez l' chirurgien}
\bis {... Soigner mon p'tit }
\bis {J' t'enfonc' dans l' cul }
\bis {Ton bistouri. }
\endverse

\beginverse
\bis {Chez l' pharmacien}
\bis {... M' donner d' potion }
\bis {... Dans tes flacons.}
\endverse

\beginverse
\bis {Chez M'sieur l' cur� }
\bis {... Nous mari-er }
\bis {... Dans l' b�nitier.}
\endverse

\endsong