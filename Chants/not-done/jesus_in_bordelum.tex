\beginsong {Jesus in bordelum} \footnotemark [
ititle= {Jesus in bordellum}]

\footnotetext {Voici pr�sent�e ici la version de la chorale de l'ULB d'apr�s une harmonisation de Francis Saint-Hubert (1990). Le Recitant est un t�nor, Jesus une basse, et la Macrella une soprano.}

\beginverse
Recitant :		In eo tempore erat Jesus in calore,
			Dixit discipulis et apostolis suis :
Jesus : 		Cauda mea erecta est. Tempora revoluta sunt ;
			Eamus ad bordelum ut pinare !
Recitant : 		Discipulis responderunt :
Chorale : 		Eamus, eamus.
Recitant : 		Entraverunt in Via Sancti Laurenti, arrivati ad
			portam bordeli. Jesus dixit :
Jesus : 		Silencia, silencia.
Recitant : 		Tocavit que ad portam. Apparuit macrella
			monstruosa cum gueula fardata. Demandans :
Macrella : 		Vultime co�re aut flanellam facere ? Aut flanellam
			facere ?
Recitant : 		Discipulis responderunt :
Chorale : 		Co�re, co�re !
Recitant : 		Entraverunt in salonum ; Jesus vidit in canapeto
			sedente, macrella bene garnita cum gueula
			hospitalata. Dixit Jesus :
Jesus : 		Vultime ludere cum couillonibus meus ?
Recitant : 		Jesus foutavit champagnum, sed Petrus bibit
			grenadinam cum aqua gazata, nam calidam
			pissam habebat. Deboutonavit culottam suam ;
			apparuit cauda immensa cum glando mirabile et
			couillonibus monstruosis. Jesus foutavit caudam
			suam in culo Maria Magdalena qui suscabat
			pinam sancti Matthei. Poussavit ad fundum.
Public : 		Et dechargeatus !
\endverse

\beginverse
Air : Passion selon Saint-Mathieu (Johann Sebastian Bach)

Chorale : 		Erat Jesus in bordellum, erat in bordellum.
			Jouissat Jesus in bordellum, jouissat in bordellum
			Foutavit caudam suam in culo Magdalen� :
			Poussavit ad fundum et dechargeatus.
\endverse

\beginverse
Air : 			Chorale
Recitant : 		Dixit Jesus :
Jesus : 		Consumatum est.
Chorale : 		\bis {Amen Alleluia !}
\endverse

\endsong