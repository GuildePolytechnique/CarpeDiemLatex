\beginsong{KLOKKE ROELAND}\footnotemark [
ititle={Klokke Roeland}]

\footnotetext{Auteurs : Berten Rodenbach, J. De Stoop, A. Veremans}

\beginverse
Boven Gent rijst, eenzaam en grijs,
't Oud Belfort, zinbeeld van 't verleden;
Somber en groots, steeds stom en doods.
Treurt d' oude reus op 't Gent van heden;
Maar soms hij rilt en eensklaps gilt
Zijn bronzen stemme door de stede.
Trilt in uw graf, trilt Gentse helden;
Gij Jan Hyoens, gij Artevelden;
Mijn naam is Roeland, 'k kleppe brand
En luide storm in Vlaanderland.
\endverse

\beginverse
Een bont verschiet schept 't bronzen lied,
Prachtig weertoverd mij voor d' ogen
Mijn ziel herkent het oude Gent;
't Volk komt gewapend toegevlogen.
't Land is in nood : "Vrijheid of dood!",
De gilden komen aangetogen,
'k Zie Jan Hyoens, 'k zie d' Artenvelden,
En stormend roept Roeland den helden :
Mijn naam is Roeland, 'k kleppe brand,
En luide storm in Vlaanderland.
\endverse

\beginverse
O Heldenvolk, O reuzenvolk,
O pracht en praal van vroeger dagen!
O bronzen lied, 'k wete uw bedied,
En ik versta 't verwijtend klagen;
Doch wees gestroost : zie 't Oosten bloost
En Vlaandrens zonne gaat aan 't dagen.
"Vlaanderen die Leeu", tril oude toren,
En paar uw lied met onze koren;
Zing : "Ik ben Roeland, 'k kleppe brand,
Luide triomf in Vlaanderland!".
\endverse
\endsong