\beginsong{KOMT, VRIENDEN, IN HET RONDE} [
ititle={Komt, vrienden, in het ronde}]

\beginverse
Komt, vienden, in het ronde,
Minnaars van enen stiel,
Ik zal u gaan verkonden,
Hoe ik door 't slijperswiel
Den kost verdien voor vrouw en kind,
Schoon bloodgesteld aan weer en wind.
\endverse

\beginchorus
\textbf{Refrain}
Terliererom terla
Van linksom, rechtsom draait mijne steen
Door het roeren van mijn been,
\bis{Ju! Ju! Ju! Ju! Ju! Ju! Ju! Ju!}
\endchorus

\beginverse
De smid die moet hard werken
Gestatig voor het vier;
Hij durft hem niet versterken
Met ene kan goed bier,
Terwijl ik ga op mijn gemak
Soms ook wel met een lege zak.
\endverse

\beginverse
De schoenpik, stijf gezeten
Op ene pikkelstoel
Mag kees en droog brood eten,
Maar als ik nood gevoel
Dan slijp ik tot den avond toe
En zo heb ik nooit arremoe.
\endverse

\beginverse
De kleerfrik maakt ons kleren
Voor acht stuivers per dag
Wil hij den loon vermeren,
Hij snijdt meer dan hij mag
Maar ik met mijne slijpersteen
Ik win meer op een uur alleen.
\endverse

\beginverse
De maalder moet graan malen
Tot in het fijnste meel;
Hij doet dubbel betalen
Voor zijne droge keel
Maar ik door iever en door vlijt
Ik win mijn brood in eerlijkheid.
\endverse

\beginverse
Mijn vrouw die roept victoria
Over den slijpersstiel
Zij vindt haar grootste gloria
In 't draaien van het wiel
Mijn kind'ren hebben geen ongemak
Zij lopen met den bedelzak!
\endverse

\beginverse
Sa, vrienden, voor het leste
All' ambachten zijn goed,
Maar 't mijn is toch het beste,
Schoon ik soms slapen moet
Op hooi of stro in enen stal :
Daar heb ik kost voor niemendal.
\endverse
\endsong