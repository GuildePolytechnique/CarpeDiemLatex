\beginsong{KRAMBAMBOULI}\footnotemark [
ititle={Krambambouli},
tu={Crescentius Koromandel (Chanson estudiantine allemande, 1745)}]

\footnotetext{P. : Wittekind, 1745 - M. : milieu du XVIIIème siècle. Remarque : dans certaines versions, on bisse les deux derniers vers de chaque strophe au lieu du dernier seulement.}

\beginverse
"Krambambouli", zo wordt geheten
Dat schuimend blond studentennat.
Wie zou d'r op aarde iets beter weten
In alle pijn en smart als dat?
Van 's avonds laat tot 's morgens vroeg
Drink ik mijn glas krambambouli,
\bis{Krambimbambambouli, krambambouli!}
\endverse

\beginverse
En brandt mijn hoofd en mijne wangen,
Of breekt mijn herte van verdriet,
Of krult mijn maag in duizend tangen
Of bibbert 't lijf gelijk een riet,
Ik lach met al die medici
En drink mijn glas krambambouli,
\bis{Krambimbambambouli, krambambouli!}
\endverse

\beginverse
War' ik als edelman geboren,
Keizer zoals Maximiliaan,
Ik stichtte een orde uitverkoren
En als devies hing ik daaraan,
Toujours fidèle et sans souci
C'est l'Ordre du Krambambouli,
\bis{Krambimbambambouli, krambambouli!}
\endverse

\beginverse
Is moeders geld nog uitgebleven
En heb ik schulden met de macht,
Heeft 't zoete lief me niet geschreven
De post van thuis droef nieuws gebracht,
Dan drink ik uit melancholie,
Een schuimend glas krambambouli,
\bis{Krambimbambambouli, krambambouli!}
\endverse

\beginverse
En is mijn geld al naar de donder
Dan peezuig ik van elke schacht,
Al heb ik geld, al zit ik zonder,
Eens wordt het heelal tot stof gebracht,
Want dat is de filosofie,
Naar de geest van krambambouli,
\bis{Krambimbambambouli, krambambouli!}
\endverse
\endsong