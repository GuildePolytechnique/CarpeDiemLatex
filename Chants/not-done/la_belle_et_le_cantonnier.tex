\beginsong {La Belle et le Cantonnier}
[ititle={Belle et le Cantonnier, La}, tu={Sur la route de Louviers}]

\beginverse
\bis{Sur la route de Louviers}
\bis{Il y'avait un cantonnier}
\bis{Et qui baisait}
Et qui baisait comm' un voyou
Au lieu d' casser des cailloux.
\endverse
\beginverse
\bis{Un' bell' dam' vint à passer}
\bis{Dans un beau caross' doré}
\bis{Elle y baisait}
Elle y baisait comm' un voyou
A en fair' craquer les roues.
\endverse
\beginverse
\bis{Elle aperçut l' cantonnier}
\bis{Dans le fond d'un grand fossé}
\bis{Qui y baisait}
Qui y baisait comm' un voyou
Un' fillett' aux cheveux roux.
\endverse
\beginverse
\bis{Elle lui dit : " Brav' cantonnier}
\bis{Avec moi veux-tu monter ?}
\bis{Pour me baiser}
Pour me baiser comm' un voyou
Le préfet est mon époux. "
\endverse
\beginverse
\bis{A ces mots, le cantonnier}
\bis{Laiss' la rouss' dans le fossé}
\bis{Et va baiser}
Et va baiser comm' un voyou
La bell' dam' plein' de bijoux.
\endverse
\beginverse
\bis{Le lend'main par arrêté}
\bis{Fut nommé chef cantonnier}
\bis{Parc' qu'il baisait}
Parc' qu'il baisait comm' un voyou
Au lieu d' casser des cailloux.
\endverse
\beginverse
\bis{Voici la moralité}
\bis{Dans la vie pour arriver}
\bis{Il faut baiser}
Il faut baiser comm' des voyous
Les bell's dam's qui ont des sous !
\endverse
\endsong
