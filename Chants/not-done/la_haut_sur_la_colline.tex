\beginsong {Du haut de la montagne}\footnotemark
[ititle={Là-haut sur la colline},ititle={Du haut de la montagne},tu={Malbrough s'en va-t-en guerre}]
\footnotetext{Une version plus proche des origines de cette chanson se trouve dans "69 Chansons d'Etudiants" (1984).}

\beginverse
Du haut de la montagne,
Nom de Dieu ! Joséphin', laisse-toi faire !
Du haut de la montagne,
Descendait un gros cu- (bis)
\endverse
\beginverse
Un gros curé d' campagne
Nom de Dieu ! Joséphin', laisse-toi faire !
Un gros curé d' campagne
\bis{Suivi de son long vi-}
\endverse
\beginverse
Suivi d' son long vicaire ...
\bis{Qui tenait son gros bou-}
\endverse
\beginverse
Son gros bouquin d' prières ...
\bis{Qui était plein de jus-}
\endverse
\beginverse
Plein de justic' divine ...
\bis{Pour entrer dans un con-}
\endverse
\beginverse
Dans un confessionnal(e) ...
\bis{Pour y tirer un cou-}
\endverse
\beginverse
Un couple de l'enfer(e) ...
\bis{Qui avait mal occu-}
\endverse
\beginverse
Occupé sa jeunesse ...
\bis{Et avait trop été}
\endverse
\beginverse
Trop été à la messe ...
\bis{Où il allait quêter}
\endverse
\beginverse
Quêter l'aumôn' pour pauvres ...
\ter {Et pour la Trinité.}
\endverse
\endsong
 
