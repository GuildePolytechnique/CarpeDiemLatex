\beginsong{Lamentations de Chimène}
[ititle={Lamentations de Chimène},tu={Ma Normandie (F. Bérat)}]

\beginverse
Au pied du Li-ban dans l'Asie,
Assise et trom-pant sa douleur,
La belle Chi-mène attendrie,
D'Ulric accu-sait la froideur :
Quand j'ai foi, ré-pétait Chimène
Dans la loyau-té de ton coeur,
Brisant le li-en qui t'enchaîne,
Ah ! Tu manqu's, Ul-ric, à l'honneur.
\endverse

\beginverse
Quand j'ai su sé-duire ton âme
Tu m'en fis l'a-veu sans détour
Et grâce à l'ar-deur de ta flamme
Tout sembla qui dans notre amour.
Malgré ton bou-illant caractère,
Je fis sans pei-ne ton bonheur.
Ta bouche a ju-ré de me plaire
Et tu manqu's, Ul-ric, à l'honneur.
\endverse

\beginverse
Un dédain, dont je suis froissée,
A, trop iné-xorable amant,
Montré qu'une u-nique pensée
Soumettait ton coeur inconstant.
Je sais qu'occu-per d'autres belles,
Tu vois mon trou-ble sans douleur ;
Mais j'ai des ven-geances cruelles,
Quand on manqu', Ul-ric, à l'honneur.
\endverse

\beginverse
Quand un vif outr-age m'offense,
Tu vas, au lieu de t'excuser,
Ici, pour sau-ver l'apparence,
Sans preuve aucu-ne m'accuser.
Tyran, ton cou-pable langage
Fait tout pour ir-riter mon coeur.
Crois à ma tris-tesse, à ma rage,
Si tu manqu's, Ul-ric, à l'honneur.
\endverse
\endsong