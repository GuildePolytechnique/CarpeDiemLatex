\beginsong {Le Bandeur}\footnotemark
[ititle={Bandeur, le}, tu={Le clairon (P. : Paul Déroulède - M. : Émile André, 1878)}]

\footnotetext{Chanson postérieure à la guerre franco-allemande de 1870-1871 (fin du Second Empire), date après laquelle Paul Déroulède consacra son temps à la poésie et à la politique. Musique sans doute composée pour un spectacle Mlle Amiati à l'Eldorado (Source : Le Soir, Sam. 14 et Dim. 15 août 1993, p. 33, "Il peut le dire").}

\beginverse
Il fait nuit, le lit est large,
En songeant à la décharge,
Il se réveille en bandant.
Et c'est alors que Rosine
Doucement lui prend la pine.
Ça fait du bien sur l' moment !
\endverse
\beginverse
Le bandeur est un vieux brave.
S'il se présente un coup grave,
C'est un rude compagnon,
Il a fait maintes ripailles
Et porte plus d'une entaille
De la quéquett' au croupion.
\endverse
\beginverse
On branle, on suce, on active,
La décharge devient vive
Et tous les deux sont adroits.
Rosin', étant très coquette,
Veut lui branler la quéquette :
Il décharge dans ses doigts.
\endverse
\beginverse
Il est là couché, superbe,
Bandant toujours comm' un Serbe
Et, dédaignant tout secours,
Sa bit' est toute gluante,
Mais dans sa fureur ardente,
Il bande, il bande toujours.
\endverse
\beginverse
\textit {Débâcle :}
Mais la momich' éreintée
De foutr' est tout engluée,
Elle ne peut plus jou-ir.
Le bandeur, avec ivresse,
Lui saisissant les deux fesses,
L'encul' alors pour finir.
\endverse
\endsong


