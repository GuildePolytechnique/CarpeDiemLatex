\beginsong {Le Bleu revint}\footnotemark
[itile={Bleu revint, le}, tu={Grand-Papa Nicolas}]

\footnotetext{Guilde polytechnique, ULB ; Festival international de la chanson estudiantine CP ULB, 1990.}

\beginverse
C'est avec le troupeau qu'arriv' le bleu nouveau,
Il pique-niqu' dans le bois, il fait des jeux très cras,
Il divertit les poils, on le clach', c'est bestial :
Bière, huile et farine, et bien encore ... MAIS
\endverse
\beginchorus
\textbf{Refrain}
Le bleu revint le lendemain matin,
Le bleu revint, le fait en est certain.
Nul ne saura ni comment ni pourquoi,
À la guindaille suivant', le bleu était là.
\bis{Po-ro-po-pom-pom-pom-pom-pom}
\endchorus
\beginverse
Puis un beau soir, on luit dit : " Viens donc boire,
On va chanter quelques chansons bien gaies ;
Un' foll' ambiance, être bleu est un' chance. "
Au TP guindaille, le bleu déraille ... MAIS
\endverse
\beginchorus
Le bleu revint ... ... le bleu est encore là.
\endchorus
\beginverse
A ce stade-là, un pot ne suffit pas ;
Le bleu n'est pas sérieux s'il ne peut en boire deux.
Et même bien davantage, il le peut à son âge.
Il va tout gerber, tout dégueuler ... MAIS
\endverse
\beginchorus
Le bleu revint ... ... le bleu est toujours là.
\endchorus
\beginverse
De café en café, notre bleu va ramper :
Il va ingurgiter des moules périmées,
De savoureux cocktails lui seront proposés.
Et son estomac n'en revient pas ... MAIS
\endverse
\beginchorus
Le bleu revint ... ... le bleu est toujours là.
\endchorus
Après bien des déboires, arrive le grand soir.
Par la sall' déchaînée le bleu se fait clacher,
Tout's les délégations jur'nt bien qu'ell's le tondront :
Ses cheveux par terre, nu comme un ver
\endverse

\beginverse

[tu={Bambino (interprétée par Dalida)}]

\emph{À chaque vers, les filles répondent aux garçons par : Petit bleu (bis), à l'exception du dernier où tout le monde chante.}

Tu n'es que l'ombre de toi-même !
Que tu as le visage blême !
T'as pas survécu au baptême !
Mais tu t'en remettras quand même !
\endverse

\beginverse
[tu={Grand-Papa Nicolas}]

Maintenant baptisé et de sa penn' coiffé
Le jour de la St-Vé, il va boir' et chanter.
Fini les gueul's en terre, les brouillards à la bière,
Et, en tablier, il va gueuler ...
\endverse

\beginchorus
\textbf{Dernier refrain}
Je reviendrai car j'aim' bien guindailler ;
Oui, tout compt' fait, je m' suis bien amusé
Et j'apprendrai aux nouveaux arrivés
Comment on devient poil ici à l'ULB.
Le poil revint le lendemain matin,
Le poil revint, le fait en est certain.
Nul ne saura ni comment ni pourquoi,
A la guindaill' suivant', le poil est toujours là.
Po-ro-po-pom-pom-pom-pom-pom (ad libitum)
\endchorus
\endsong


