\beginsong {Le Brick-Goélette} [ititle={Brick-Goélette, Le}]

\beginverse
'L était un brick-goélett', ma doué,
Un brick à grandes voiles,
Qui s'app'lait l'Aquilon,
Amurez la grand-voile,
Qui s'app'lait l'Aquilon,
Amurez l' foc\footnote{Foc (néerl. fok ; 1702) : voile triangulaire placée à l'avant d'un navire à voile. (in Larousse,Dictionnaire de la langue française, Lexis, 1992)} ballon.
\endverse
\beginverse
Il était commandé, ma doué,
Par un grand capitaine,
Un grand gars de Couëron,
Amurez la grand-voile,
Un grand gars de Couëron,
Amurez l' foc ballon.
\endverse
\beginverse
Après trois mois passés, ...
Il arriva t'en rade,
En rade de Toulon ...
\endverse
\beginverse
Il descendit à terr', ...
S'en alla sur la darse,
Rencontr' Mamzell' Suzon ...
\endverse
\beginverse
Y t' la prend, y t' l'embrass' ...
Lui largue les bonnettes,
Lui r'trouss' ses goémons ...
\endverse
\beginverse
Puis quand il l'eut mouillé, ...
Y t' lui sortit un membre,
Comm' un' vergu' d'artimon, ...
\endverse
\beginverse
Puis y t' le lui enfonce ...
D' la longueur de trois brasses,
Jusqu'aux tréfonds du fond ...
\endverse
\beginverse
Mais au fin fond du fond ...
Il y avait un chancre
Qui ne sentait pas bon ...
\endverse
\beginverse
Il en naissait un flux ...
Épais comme de l'huile,
Plus jaun' que du citron ...
\endverse
\beginverse
Et quand il eut fini, ...
Ça t' lui bouffit son membr'
Au ras du caneçon ...
\endverse
\beginverse
Le pauv' gars en est mort ...
Et le beau capitaine
N'a jamais r'vu Couëron ...
\endverse
\endsong