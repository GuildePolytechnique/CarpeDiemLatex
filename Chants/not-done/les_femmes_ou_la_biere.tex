\beginsong{LES FEMMES OU LA BIERE}\footnotemark [
ititle={Femmes ou la bière, les},
tu={La complainte de Mandrin (trad.)}]

\footnotetext{P. : E. Yvergneaux (1986).}

\beginverse
Dans tous les coins du monde,
Des villages aux cités,
Qu'elle soit brune ou blonde
Ou rousse, sûr'ment, vous m'entendez,
Qu'elle soit brune ou blonde, à coup sûr vous l'aimerez.
\endverse

\beginverse
Sur le visage des hommes
Elle inscrit la gaieté,
Et que Dieu me pardonne
On ne peut pas, vous m'entendez,
Et que Dieu me pardonne, on ne peut s'en lasser.
\endverse

\beginverse
Des hommes, elle est compagne
Depuis l'Antiquité
L'Occident, l'Orient,
L'Egypte ancienne, vous m'entendez,
L'Occident, l'Orient aimaient la célébrer.
\endverse

\beginverse
Orge, malt ou framboise,
Cerise ou bien houblon
Qu'importe sa teneur
Ou bien son goût, vous m'entendez
Qu'importe sa saveur, amis, vous la boirez.
\endverse

\beginverse
Elle est dev'nue symbole
De la Fraternité
Fraternité qui règne
Depuis toujours, vous m'entendez,
Qui règne depuis toujours à l'Université.
\endverse
\endsong