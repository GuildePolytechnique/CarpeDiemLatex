\beginsong {Les Légionnaires} [ititle={Légionnaires, Les}]

\beginverse
Il est sur la terr' africaine,
Un régiment dont les soldats, (dont les soldats)
Sont tous des gars qu' on pas eu d' veine,
C'est la légion et nous voilà ! (et nous voilà)
Pour ce qui est d' la discipline,
Faut êtr' passé par Biribi ! (par Biribi)
Avoir goûté de la praline,
Et travaillé du bistouri. (du bistouri)
\endverse
\beginchorus
\textbf{Refrain}
Et on s'en fout, et après tout,
Qu'est-ce que ça fout -out -out -out ?
En marchant sur la grand'-route,
Souviens-toi, oui souviens-toi, ah ! Ah ! Ah !
Les anciens l'ont fait sans doute,
Avant toi, oui avant toi, ah ! Ah ! Ah !
De Gabès à Tataouine,
De Tanger à Tombouctou, -ou -ou -ou !
Sac à dos dans la poussière,
Marchons les légionnaires.
\endchorus
\beginverse
J'ai vu mourir un pauvre gosse,
Un pauvre goss' de dix-huit ans, (de dix-huit ans)
Fauché par les balles féroces,
Il est mort en criant : " Maman ! ", (criant : " Maman ! ")
Je lui ai fermé les paupières,
Recueilli son dernier soupir, (dernier soupir)
J'ai écrit à sa pauvre mère,
Qu'un légionnair', ça sait mourir. (ça sait mourir)
\endverse
\beginverse
Et puisqu'on n'a jamais eu d' veine,
Pour sûr qu'un jour, on y crèv'ra ! (on y crèv'ra)
Sur cett' putain d' terre africaine,
Enterrés sous le sable chaud ! (le sable chaud)
Avec pour croix un' baïonnette
A l'endroit où l'on est tombé ! (l'on est tombé)
Qui voulez-vous qui nous regrette,
Puisqu'on est tous des réprouvés ? (des réprouvés)
\endverse
\endsong