\beginsong {Les Mères d'à présent}\footnotemark [ititle={Mères d'à présent, Les}]
\footnotetext{Cette chanson semble dater du XVIIIème siècle.}

\beginverse
Ah ! que les mères d'à présent
Ont du tourment avec leurs filles ;
Elles ont toutes un amant,
Surtout lorsqu'elles sont gentilles.
Pour un amoureux
Jeun' et vigoureux
Elles briseraient fers et grilles
Et s'échapperaient d' la Bastille.
\endverse
\beginverse
Pauline, un soir à son amant
Qu'elle désirait à la folie,
Donnait un rendez-vous charmant
Pour satisfaire son envie.
" Ah ! Viens donc ce soir,
Tu es mon espoir ;
\bis{Colin n'y manqu' pas, je t'en prie.}
\endverse
\beginverse
Et tiens, voilà mon pass'-partout,
Je suis au cinqui-ème étage ;
Eh ! Colin, attention surtout
De ne pas faire de tapage.
De mon cabinet,
Tu sais le secret
\bis{Je ne t'en dis pas davantage. "}
\endverse
\beginverse
La mèr' avait quelque soupçon
Car elle avait été gentille ;
Se doutant bien qu'un beau garçon
Était couché près de sa fille.
Elle mont' doucement
Et frappe : pan, pan !
\bis{Colin, dans les draps, s'entortille.}
\endverse
\endsong
