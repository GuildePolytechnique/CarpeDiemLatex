\beginsong {Lolotte}\footnotemark [ititle={Lolotte}]
\footnotetext{Auteur : Jacques Bertrand (1865)}

\beginverse
Su l' bord de Sampe et pierdu dins l' fumée,
Weyè Couyet avec sclotchi crawyeu ?
C'est là qu' demeure em' matante Dorothée,
Veuve de m' nonque Adrien du Crosteu :
À s' neuv' maiso nos avons fait ribote
Lundi passet tot en pindant l' crama.
\bis{Pou l' premi coup c'est là qu' djai vu Lolotte}
Ré qu' d'y pinsé sintet comme em' coeur bat ! 
\endverse
\beginverse
Gn' avait drolà les pus gais du villatche
En fait d' coumères on n'avait qu'à chwesi
On a r'sinet à l'ompe du feuillatche
Devant l' maiso padzou l' gros cérégi.
Em' bonn' matante a del bière en bouteille,
C' n'est né l' faro qu'est jamais si bon qu' çà,
\bis{Dins s' chiq' Lolotte estait si bé vermeye}
Qu' ré qu' d'y pinsé sintet comme em' coeur bat ! 
\endverse
\beginverse
Y d' allait mieux, les panses es'tant rimplies
D' Jean l' blanchisseu tinguel es'violon,
Et dit : " zefans nos avons ci des fies
Qui n' demand' nu qu'à danser l'rigodon. "
Ah ! qué plaigi ! quet Lolotte è contenne !
Après l' cadrie on boutte enne mazurka
\bis{Djet triannais en pressant l'main dins l' menne,}
Ré qu' d'y pinsé sintet comme em' coeur bat ! 
\endverse
\beginverse
V'là l' soër venu, pou danser chacun s' presse,
L' violoneu raclait avec ardeur,
L' bière et l'amour em'faye tourner l' tiesse,
Vingt noms di chniq ! djet nad'geais dins l'bonheur
Mais l' pa Lolotte en weyant qu'elle m'embrasse
D'in coup d'chabot m'fait plondgi dins l' puria,
\bis{L' coumère est sauffe eyet mi djet m' ramasse,}
Ciel qué coup d' pid ! sintet comme em' coeur bat ! 
\endverse
\beginverse
D'jet m' souvairai du crama del m' matante,
Dj' croës qu' dj'ai l' cripet casset ou bé desmis,
D'jet prinds des bains à l' vapeur d'eau bouillante,
Grignant les dins tous les coups quet dj' m'achis
Mais quand dj' devrais squetté m' derrenne culotte
En m'empoëgnant avec smame eyet s'pa,
\bis{Plutôt mori quet viquet sins Lolotte,}
Ré qu' d'y pinsé sintet comme em' coeur bat ! 
\endverse
\endsong