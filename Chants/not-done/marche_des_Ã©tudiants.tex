\beginsong {La Marche des Etudiants} \footnotemark
[ititle={Marche des Etudiants, La}, ititle={Chant des étudiants}, tu={Les Gueux (P. : Paul Vanderborght, 1919)}]
\footnotetext{Ce titre était renseigné sous Chant des Étudiants dans Les Fleurs du Mâle-Geuzenliederboek (1967).}

\beginverse
Nous sommes ceux qu'anime la folie
Et qui s'en vont ivres de liberté ;
Nous faisons guerr' à la mélancolie
Ou la cachons sous des cris de gaieté.
Bourgeois sans feu, votre vie est banale :
Les préjugés guident vos fronts tremblants ;
\bis{Chez nous, l'on a l'humeur paradoxale,}
Le coeur léger, et le gosier brûlant. 
\endverse
\beginverse
Des vieux gaulois nous gardons la mémoire
En les chantant perchés sur nos tonneaux ;
Si le bourgeois veut nous payer à boire,
Nous le suivrons jusqu'au fond des caveaux.
Fraternité, tu nais entre les verres ;
Ami, buvons à la Fraternité !
\bis{Haro ! Haro sur les mines sévères !}
Pourquoi Bacchus n'est-il pas député ? 
\endverse
\beginverse
Si nous avons parfois la bourse plate,
Nous possédons bien des coeurs de trottins ;
Car, en amour, nous sommes des pirates
Braquant partout leurs regards assassins.
Souvent, pourtant, nous devons en rabattre
De nos grands airs de riche Don Juan :
\bis{Dans les bouquins nous allons nous ébattre}
Pour oublier les suppôts de Satan. 
\endverse
\beginverse
Quand nous serons amis de doctes sages,
Nous sourirons doucement au passé
En regrettant, malgré tout, ce bel âge
D'enthousi-asme à jamais effacé.
Alors, tirant sur nos vieilles bouffardes,
Nous redirons à mi-voix nos chansons ;
\bis{Elles étaient peut-être un peu gaillardes}
Mais on hurlait si bien à l'unisson ! 
\endverse
\endsong