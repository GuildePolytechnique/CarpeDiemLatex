% !TEX encoding = UTF-8 Unicode
\beginsong {La marseillaise anticléricale\footnotemark} [
ititle = {Marseillaise anticléricale (la)},
tu = {La marseillaise}]

\footnotetext {Chanson écrite par Léo Taxil, en 1881.}

\beginverse
Allons ! Fils de la République,
Le jour du vote est arrivé !
Contre nous de la noire clique
\bis {L'oriflamme ignoble est levé.}
Entendez-vous tous ces infâmes
Croasser leurs stupides chants ?
Ils voudraient encore, les brigands,
Salir nos enfants et nos femmes !
\endverse

\beginchorus
Aux urnes, citoyens, contre les cléricaux !
Votons, votons et que nos voix
Dispersent les corbeaux !
\endchorus

\beginverse
Que veut cette maudite engeance,
Cette canaille à jupon noir ?
Elle veut étouffer la France
\bis {Sous la calotte et l'éteignoir !}
Mais de nos bulletins de vote
Nous accablerons ces gredins,
Et les voix de tous les scrutins
Leur crieront : A bas la calotte !
\endverse

\beginverse
Quoi ! Ces curés et leurs vicaires
Feraient la loi dans nos foyers !
Quoi ! Ces assassins de nos pères
\bis {Seraient un jour nos meurtriers !}
Car ces cafards, de vile race,
Sont nés pour être inquisiteurs...
À la porte, les imposteurs !
Place à la République ! Place !
\endverse

\beginverse
Tremblez, coquins ! Cachez-vous, traîtres !
Disparaissez loin de nos yeux !
Le Peuple ne veut plus des prêtres,
\bis {Patrie et Loi, voilà ses dieux}
Assez de vos pratiques niaises !
Les vices sont vos qualités.
Vous réclamez des libertés ?
Il n'en est pas pour les punaises !
\endverse

\beginverse
Citoyens, punissons les crimes
De ces immondes calotins,
N'ayons pitié que des victimes
\bis {Que la foi transforme en crétins}
Mais les voleurs, les hypocrites,
Mais les gros moines fainéants,
Mais les escrocs, les charlatans...
Pas de pitié pour les jésuites !
\endverse

\beginverse
Que la haine de l'imposture
Inspire nos votes vengeurs !
Expulsons l'horrible tonsure,
\bis {Hors de France, les malfaiteurs !}
Formons l'union radicale,
Allons au scrutin le front haut :
Pour sauver le pays il faut
Une chambre anticléricale.
\endverse

\endsong