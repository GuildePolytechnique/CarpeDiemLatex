\beginsong {Masculin Singulier}\footnotemark [ititle={Masculin Singulier}, tu={Le rire du sergent (Michel Sardou)}]

\footnotetext{Les Six Filles Lésées, ULB ; Festival de la chanson estudiantine CP ULB, 1986.}

\beginverse
Je vais vous conter l'histoire des filles de l'ULB,
Celle de la p'tite garce qui se savait pas mal roulée,
Qui voudrait pourtant
Pouvoir de temps en temps
S'envoyer en l'air
Sans devoir s' marier pour autant
Mais à chaque fois
Qu'elle tombe sur un gars
Elle voit poindre la bague au doigt
Qu'elle ne souhaitait pas ...
\endverse

\beginchorus
\textbf{Refrain}
Je cherche un gars charmant
Bon, chic, intelligent
Et qui ne soit pas trop pédant
Pour autant
Je veux pouvoir draguer
Sans qu'il se sente coincé
Voire agressé ou carrément violé
Pour moi l'amour, tu vois
C'est pas maman-papa
Le p'tit train-train bourgeois
Ni l'homme qui fait le pas
La Marie-couche-toi-là
Qu'on prend à peine entre ses bras
\endchorus

\beginverse
J'ai fais publier partout ce merveilleux portrait
Dans Ciné-Revue ou dans Libé.
C'est pas le pied
Soit à l'ULB
Même à l'UAE
Jamais j' n'ai trouvé
Le p'tit gars qui veut s'amuser
Et voilà pourquoi
Ce soir, je suis là
J'ai voulu te dire enfin
Ce que je pensais de toi.
\endverse

\beginverse
On dit qu'à l'unif., y'a pas de folklore féminin
Normal si on croit que toutes les femmes sont des putains
Pourtant je te dis
Qu' j'ai ma place ici
Pour chanter autre chose
Que les cons ou les zizis
Si tu voulais bien
Me tendre la main
On pourrait construire ensemble
Le Folklore de demain
\endverse

\beginverse
Le Rire et l'Amitié,
L'Humour et la Gaieté
M'ont fait comprendre que guindailler
C'est pas con
Notre Féminité pourrait être explicitée
Pour faire germer l'Esprit de notre Maison
Simplement en voyant les plumes
Autrement qu'en salopes confirmées
En cessant pour de bon
De hurler comme des cons
Le cul, les nichons au balcon ...
\endverse
\endsong