\beginsong {Les Moines de Saint Bernardin} [ititle={Moines de Saint Bernardin, Les}]

\beginverse
\bis{Nous sommes les moines de Saint-Bernardin}
\bis{Qui nous couchons tard et nous levons matin}
Pour aller, à matines, vider quelques flacons
Voilà c' qu' est bon, et bon, et bon.
\endverse
\beginchorus
\textbf{Refrain}
\bis{Et voilà la vie, voilà la vie, la vie chérie, ah ! Ah !}
Et voilà la vie que les moines font. 
\endchorus
\beginverse
\bis{Pour notre déjeuner du bon chocolat}\footnote{Dans la version originale - française -, ce couplet est inexistant.}
\bis{Et du bon café que l'on nomme moka}
Et d' la tarte sucrée, et des marrons d' Lyon,
Voilà c' qu' est bon, et bon, et bon.
\endverse
\beginverse
\bis{Pour notre dîner de bons petits oiseaux}
\bis{Que l'on nomme caille, bécass', ou perdreau}
Et la fin' andouillette, et la tranch' de jambon
Voilà c' qu' est bon, et bon, et bon.
\endverse
\beginverse
\bis{Pour notre coucher dans un lit aux draps blancs}
\bis{Une belle nonne de quinz' à seize ans}
À la taille bien faite et aux nichons bien ronds
Voilà c' qu' est bon, et bon, et bon.
\endverse
\beginverse
\bis{La nuit, tous ensemble, nous nous enculons}
\bis{Jusqu'au jour, ensemble, nous buvons, buvons}
Après, dessous la table, nous roulons et dormons
Voilà c' qu' est bon et bon et bon.
\endverse
\beginverse
\bis{Si c'est ça, la vie que tous les moines font}
\bis{Je me ferai moin' avec ma Jeanneton}
Et couché sur l'herbette, j' lui chatouill'rai l' bouton.
Voilà c' qu' est bon, et bon, et bon.
\endverse
\endsong
