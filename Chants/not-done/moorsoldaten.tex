% !TEX encoding = UTF-8 Unicode
\beginsong {Die Moorsoldaten\footnotemark} [
ititle= {Moorsoldaten (Die)}]

\footnotetext {Ecrit au camp de concentration Papenburg de Börgermoor, 1933}

\beginverse
Wohin auch das Auge blicket,
Moor und Heide nur ringsum.
Vogelsang uns nicht erquicket,
Eichen stehen kahl und krumm.
\endverse

\beginchorus
Wir sind die Moorsoldaten,
Und ziehen mit dem Spaten,
Ins Moor.
\endchorus

\beginverse
Hier in dieser öden Heide
Ist das Lager aufgebaut,
Wo wir fern von jeder Freude
Hinter Stacheldraht verstaut.
\endverse

\beginverse
Morgens ziehen die Kolonnen
In das Moor zur Arbeit hin.
Graben bei dem Brand der Sonne,
Doch zur Heimat steht der Sinn.
\endverse

\beginverse
Heimwärts, heimwärts jeder sehnet,
Zu den Eltern, Weib und Kind.
Manche Brust ein Seufzer dehnet,
Weil wir hier gefangen sind.
\endverse

\beginverse
Auf und nieder gehn die Posten,
Keiner, keiner, kann hindurch.
Flucht wird nur das Leben kosten,
Vierfach ist umzäunt die Burg.
\endverse

\beginverse
Doch für uns gibt es kein Klagen,
Ewig kann's nicht Winter sein.
Einmal werden froh wir sagen:
Heimat, du bist wieder mein:
\endverse

\beginchorus
Dann ziehn die Moorsoldaten
Nicht mehr mit dem Spaten
Ins Moor! 
\endchorus

\endsong