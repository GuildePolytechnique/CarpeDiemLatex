% !TEX encoding = UTF-8 Unicode
\beginsong {Nini-peau-d'-chien\footnotemark} [
ititle = {Nini-peau-d'-chien}]

\footnotetext {Chanson composée par Aristide Bruant, en 1889, à l'occasion du centenaire de la prise de La Bastille.}

\beginverse
Quand elle était p'tite
Le soir, elle allait,
A Saint'-Marguerite,
Où qu'a s' dessalait ;
Maint'nant qu'alle est grande,
Elle marche, le soir,
Avec ceux d' la bande
Du Richard-Lenoir.
\endverse

\beginchorus
\textbf {Refrain}
A la Bastille
On aime bien
Nini-peau-d'-chien :
Elle est si bonne et si gentille !
On aime bien
Nini-peau-d'-chien,
A la Bastille.
\endchorus

\beginverse
Elle a la peau douce,
Aux taches de son,
A l'odeur de rousse
Qui donne un frisson,
Et de sa prunelle,
Aux tons vert-de-gris,
L'amour étincelle
Dans ses yeux d' souris.
Quand le soleil brille
Dans ses cheveux roux,
\endverse

\beginverse
L' génie d' la Bastille
Lui fait les yeux doux,
Et quand a s' promène,
Du bout d' l'Arsenal,
Tout l' quartier s'amène
Au coin du Canal.
\endverse

\beginverse
Mais celui qu'elle aime,
Qu'elle a dans la peau,
C'est Bibi-la-Crème
Parc' qu'il est costaud,
Parc' que c'est un homme
Qui n'a pas l' foie blanc,
Aussi faut voir comme
Nini l'a dans l' sang !
\endverse

\endsong