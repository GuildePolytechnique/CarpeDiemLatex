% !TEX encoding = UTF-8 Unicode
\beginsong {Nunc est bibendum\footnotemark} [
ititle= {Nunc est bibendum},
ititle= {Arthur, la raclure},
tu= {Raoul, mon pitbull (Oldelaf et Monsieur D.)}]

\footnotetext {XXXVème Festival de la chanson estudiantine CP ULB 2009 (Apéro). Autres titres: Arthur, la raclure.}

\beginverse
Mais où est donc passée ma bière ?
J'en ai rêvé toute la journée,
J'l'avais laissée dans l'frigidaire
Entre les restes et le pâté.
Qui a pu être assez avide
Pour me priver de ma boisson ?
Je sens déjà fondre mon bide
Dès le début de cette chanson...
Aurait-elle pu tomber par terre,
Ou bien rouler sous le sofa ?
L?a-t-on planquée dans les waters ?
Mon colloc m'a dit : "détends-toi ?".
\endverse

\beginchorus
C'est Arthur, la raclure,
Qui l?a trouvée rafraîchissante,
C'est Arthur, la raclure,
Il est petit mais quelle descente !
Il affone, c'est un dur
Mais quand il sent vibrer sa glotte,
Il boirait même de la flotte,
C'est Arthur, la raclure.
\endchorus

\beginverse
Si j'avais su que ce rapace
Allait encore squatter chez moi,
J'aurais pris une bière en terrasse
Et j'lui aurais filé celle-là.
De toute façon, elle était tiède
Vu que mon frigo ne marche pas,
Elle traînait là depuis les fêtes
Et en plus c'était une Cara !
Tirons un trait sur ces déboires,
Au cercle on n'fait pas le Carême,
Le barman sert toujours à boire
Mais il me dit : "Y'a un problème..."
\endverse

\beginchorus
Y'a Arthur, la raclure,
Qui est passé vers midi trente
Y'a Arthur, la raclure,
Il est petit mais quelle descente !
Il affone, c'est un dur
Mais quand il sent vibrer sa glotte
Il ne laisse rien pour ses potes
C'est Arthur, la raclure.
\endchorus

\beginverse
Ah oui mais là ça part en vrille
Si le folklore ne m'abreuve pas,
Je m?en retourne dans ma famille
Là au moins la bière coulera.
Je vais enfin me prendre une murge,
Je n'en puis plus de ce pochtron,
J'ai besoin d'une goutte et ça urge,
Il n'viendra pas jusqu'à Arlon.
"Papa, fais péter la picole
J'ai besoin de me mettre une mine."
"Sorry fiston, y'a plus d?alcool..."
"N'en dis pas plus, je crois qu'j'devine?"
\endverse

\beginchorus
C'est Arthur, la raclure,
Qui m'a suivi et qui me hante,
C'est Arthur, la raclure,
Il est petit mais quelle descente !
Il affone, c'est un dur
Mais quand il sent vibrer sa glotte
Il tuerait pour faire ribote
C'est Arthur, la raclure.
\endchorus

\beginverse
Mais non enfin (c'est ridicule) :
C'est Esther ta belle-mère
Qui était déjà alcoolique,
C'est Esther ta belle-mère,
Elle sort à peine de la clinique,
Elle retrouve ses repères
Mais au premier Zizi Coin Coin,
Elle libère tous ses instincts,
C'est Esther ta belle-mère.
\endverse

\beginverse
Si cette vieille maquerelle
A pompé le jus de mon père,
Je rentre à vélo à Bruxelles,
Je pète un plomb, ça dégénère.
J'arrive à passé 23 heures,
Ma dernière chance c'est le Paki,
J'y trouverai bien mon bonheur :
Une bonne Trappiste et c'est parti...
V?là le métèque qui me déclare
Dans un français un peu douteux
Qu'un individu très bizarre
Vient de dévaliser les lieux...
\endverse

\beginchorus
C'est Arthur, la raclure,
Qui a vidé toutes mes 50
C'est Arthur, la raclure,
Il est petit mais quelle descente !
Il affone, c'est un dur
Mais quand il sent vibrer sa glotte
Il boirait même l'eau des chiottes
\bis {C'est Arthur, la raclure }
C'est Arthur, quelle enflure !
\endchorus

\endsong    