% !TEX encoding = UTF-8 Unicode
\beginverse {On est là\footnotemark !} [
ititle = {On est là},
tu = {Y'a d' la joie (Charles Trenet)}]

\footnotetext {Festival de la chanson estudiantine CP ULB, 1986. (Corporatio Bruxellensis, ULB)}

\beginverse
On est là !
On n' connaît rien à la musique,
On est là !
C'est vot' petit' Saint Nicolas
On est là !
Vous ne trouvez pas ça comique
Mais on est là ! On est là ! On n' part pas !
\endverse

\beginverse
Cette chanson
Que nous vous présentons ce soir
Cette chanson
Nous l'avons composée hier soir
Cette chanson
Ne comptez pas les pieds, les rimes
De tout's façons on chante pour la frime
\endverse

\beginverse
Faut oser
Se présenter ainsi d'vant vous
Faut oser
Venir chanter là comme nous
Faut oser
Vouloir gagner le festival
Sans chanson ce n'est pas banal !
\endverse

\beginverse
Faut du sexe
Pour faire une chanson qui marche
Faut du sexe
Pour que ça plaise aux patriarches
Faut du sexe
On parle de motte et de biroute
Tout à coup, l' jury nous écoute
\endverse

\beginverse
La Corpo.
C'est eux, c'est moi et c'est le gros
La Corpo.
Elle pue, elle pète, elle fait des rots
La Corpo.
Et si elle gagne le festival
Cett' année ça n' s'ra pas moral !
\endverse

\beginverse
Après ça
On cherche une rime en "a",
Parlé : les Montois ?
On cherche et on ne trouve pas
Parlé : les Montois ?
On tourne en rond ... ah ! Quel émoi ...
Eurêka : la rime c'est caca ! (ad libitum)
\endverse

\endsong