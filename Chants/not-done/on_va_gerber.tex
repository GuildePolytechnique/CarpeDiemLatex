\beginsong{ON VA GERBER}\footnotemark [
ititle={On va gerber},
tu={On va s'aimer (M : Gilbert Montagné, P : Smetje)}]

\footnotetext{Polytechnique ERM. Festival de la Chanson Estudiantine ULB-CP, 2000}

\beginverse
Une belle nuit, au mois d'octobre
On est parti, on était sobre
Plus pour longtemps, nos estomacs vides
Se préparaient souvent la modération
Est une des faiblaisses de chaque étudiant
Dont le gosier déborde trop souvent
\endverse

\beginchorus
On va gerber
Dans un vieux seau ou dans un sale W-C
Dans un tonneau ou dans un bénitier
On va r' découvrir les merveilles qu'on vient de manger
On va gerber
Tous ces délices un peu trop arrosés
Ce paquet d' frites à moitié digéré
Pour avoir enfin le ventre  vidé prêt à r'commencer
\endchorus

\beginverse
Car une panse sans occupation
Est comme une chope après un à-fond
Ne demande qu'une chose c'est refaire le plein
Et pour cela il n'y a qu'un moyen
C'est comme un bousier vivant sur un tas
De merde et de fumier, bouffer du ca-
Car nous on n'a pas le goût très délicat
\endverse

\beginchorus
On va gerber
Un' flasque difforme qui s'app'lait "déjeuner"
Que, dans mon délire j'essaie de réanimer,
On va r'décorer les trottoir de l'Av'nue Héger
On va gerber
La pitta-sperme de la P'tite Planète
Fait' à la main et à coups de quéquette,
Qui est le plus délicat des mets de l'étudiant-gourmet
\endchorus

\beginverse
J'avoue que l'alcool embrouille les choses
Mes mots ne sont guère plus vers ni prose
Mais l' dragueur cochon en moi s'éveille
Envie d' butiner comme les abeilles
J'ai accosté cette chouette donzelle
Mais quand j'ai voulu m'approcher d'elle
Pour lui rouler de suite une grosse pelle
\endverse

\beginchorus
Elle a vomi
Quand elle a senti mon haleine pourrie
Qui puait tellement à cause du dégueuli
On peut l' comparer à l'effet d'un à-fond aïoli
Elle a gerbé
Elle a recouvert d'un coup tous mes vêtements
De taches de bile puante ainsi que de sang
Tout un éventail de belles couleurs sortait en dégueulant
\endchorus

\beginchorus
Et la morale
C'est qu'on est adonné à boire et bouffer
Et qu'après une fois on n'en a jamais assez
Qu'il vaut mieux goûter un' deuxième fois au lieu de digérer
\endchorus
\endsong