% !TEX encoding = UTF-8 Unicode
\beginsong {La patrouille\footnotemark} [
ititle= {Patrouille (La)},
ititle= {Voilà la Patrouille}]

\footnotetext {Autre titre : Voilà la Patrouille (milieu du XIXème siècle). Voici la version chantée actuellement à l'ULB; la version originale étant d'Henry Monnier ; poème en 4 strophes et sans rengaine appelé "La Pierreuse" qui, après transformation, deviendra "La Patrouille."}

\beginverse
Viens par ici, viens mon p'tit homme
Y'a pas tant d' monde, on n'y voit rien.
Débraguett'-toi, tu verras comme
Je s'rai gentille et j' t'aim'rai bien.
Tu m' donn'ras cent sous pour ma peine,
Béni soit le noeud qui m'étrenne,
\endverse

\beginchorus
\textbf {Refrain}
Ah ! Ah ! Ah ! Ah !
C'est un' patrouille, attends-moi là
Entretiens-toi pendant qu'elle passe,
C'est un' patrouille, attends-moi là
Entretiens-toi pendant c' temps-là !
\endchorus

\beginverse
C'est des boueux, n'y prends pas garde,
Viens, que j' te magn' ton p'tit outil.
Vrai ! Qu' j'avais cru qu' c'était la garde.
Y bande encor', est-y gentil ?
Allons ! Et que rien ne t'arrête,
Fais-moi cadeau d' ta p'tit' burette,
\endverse

\beginverse
Vrai ! j'en ai t-y d' la vein' quand même
T'as du beau linge. Es-tu marié ?
T'es beau, et t'as des yeux que j'aime !
Tu dois au moins être épicier,
Ou mêm' représentant d' la Chambre.
Jouis donc, cochon ! Ah, le beau membre !
\endverse

\beginverse
J'ai beau manier ta p'tite affaire
Qu'est-ce que t'as donc t'en finis pas ?
C'est-y qu' t'aurais trop bu d' la bière,
Ou bien ma gueul' qui te r'vient pas ?
Pense à un' femme qu' aurait d' bell's cuisses
Ou bien pens' à l'Impératrice,
\endverse

\beginverse
Qu'est-c' que tu dis ? Capote anglaise ?
Mon cul est aussi propr' que l' tien.
Je me fous pas mal de ta braise,
Tu peux r'tourner d'où c'est qu' tu viens.
Qui m'a foutu c'tt' espèc' d'andouille
Qu'à seul'ment rien dans l' fond d' ses couilles,
\endverse

\beginverse
T'es rien poireau si tu supposes
Que j' vais t' la sucer pour vingt ronds !
Allons aboule encore que'qu' chose
Tu verras si j' te pompe à fond.
Tiens ! y'a le fils à M'sieur Augusse
Qui m' donn' trent' sous quand j' la lui suce,
\endverse

\beginverse
C'étaient des marlous d' connaissance.
Mais par où donc est-y passé ?
Que j'y finiss' sa p'tit' jouissance ?
C'est-y vous, M'sieur, qu' j'ai commencé ?
Ah ! merd' ça c'est pas chouett' tout de même
Sûr, il a dû s' finir soi-même,
\endverse

\beginverse
\textbf {Dernier refrain}
Cré nom de Dieu ! Cré nom de d' là !
Ah, j' suis volée pour ce coup là
Cré nom de Dieu ! Cré nom de d' là !
Faut pas d' crédit dans c' métier-là !
\endverse

\endsong