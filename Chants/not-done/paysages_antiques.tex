% !TEX encoding = UTF-8 Unicode
\beginverse {Paysages Antiques\footnotemark} [
ititle= {Paysages Antiques},
ititle = {Tremblements de terre (Les)},
tu=  {Scène de chasse}]

\footnotetext {Autre titre : Les tremblements de terre. Dans "69 Chansons d'Étudiants" (1984), on trouve un refrain à cette chanson: Adoremus, adoremus, adoremus nostrum (bis). Voici reproduite ici la version que chantent les taupins, toutefois sans le couplet sur Neptune.}

\beginverse
Les tremblements de terre,
La foudr' et le tonnerre,
\bis {Ne sont pas ce qu'on dit}
\endverse

\beginverse
Lorsque la terre tremble,
Ce sont les dieux qui s' branlent
\bis {Au fond du Paradis.}
\endverse

\beginverse

C'est le beau Ganymède
Qui tient la pine raide
\bis {Du puissant Jupiter ;}
\endverse

\beginverse
Il la branl' en cadence
Ses couilles se balancent
\bis {Jusqu'au fond des enfers.}
\endverse

\beginverse
La belle Diane lasse
Des plaisirs de la chasse,
\bis {Dort au fond d'un vallon ; }
\endverse

\beginverse
Elle sent, avec délices,
Glisser entre ses cuisses,
\bis {Le long vit d'Apollon.}
\endverse

\beginverse
Les trois Parques fileuses,
Sont trois filles péteuses,
\bis {Qui tienn'nt entre leurs mains,}
\endverse

\beginverse
En guise de quenouille,
Le fin poil noir des couilles
\bis {Du maître des humains.}
\endverse

\beginverse
Dans un boxon d'Athènes,
Le savant Demosthène
\bis {Enculait Cicéron :}
\endverse

\beginverse
Le jus philosophique
De sa pine hellénique
\bis {Coulait à gros bouillons.}
\endverse

\endsong