% !TEX encoding = UTF-8 Unicode
\beginsong {Le petit vin blanc\footnotemark} [
ititle= {Petit vin blanc (Le)}]

\footnotetext {P. : Jean Dréjac - M. : Ch. Borel-Clerc.}

\fbeginverse 
Voici le printemps :
La douceur du temps
Nous fait des avances.
Partez mes enfants !
Vous avez vingt ans,
Partez en vacances !
Vous verrez, agiles
Sur l'onde tranquille,
Les barques dociles
Aux bras des amants,
De fraîches guinguettes,
Des filles bien faites,
Les frites sont prêtes
Et y'a du vin blanc...
\endverse

\beginchorus
\textbf {Refrain}
Ah ! Le petit vin blanc
Qu'on boit sous les tonnelles
Quand les filles sont belles
Du côté de Nogent.
Et puis, de temps en temps,
Un air de vieille romance
Semble donner la cadence
{Pour fauter}
\endchorus

\beginverse
Dans les bois, dans les prés,
Du côté, du côté de Nogent.
Suivons le conseil,
Monsieur le Soleil
Connaît son affaire.
Cueillons, en chemin,
Ce minois mutin
Cette robe claire.
Venez belle fille,
Soyez bien gentille.
Là, sous la charmille,
L'amour nous attend.
Les tables sont prêtes,
L'aubergiste honnête,
Y'a des chansonnettes
Et y'a du vin blanc...
\endverse

\beginverse
A ces jeux charmants
La taille, souvent,
Prend de l'avantage.
Ça n'est pas méchant,
Ça finit tout l' temps
Par un mariage.
Le gros de l'affaire,
C'est lorsque la mère
Demande, sévère,
A la jeune enfant :
" Ma fille, raconte,
Comment, triste honte,
As-tu fait ton compte ?
Réponds, je t'attends ... "
\endverse

\beginchorus
\textbf {Refrain puis}
\endchorus

\beginverse
\textbf {Coda}
Car c'est toujours pareil,
Tant qu'y'aura du soleil
On verra les amants au printemps
\endverse

\beginverse
S'en aller, pour fauter
Dans les bois, dans les prés,
Du côté, du côté de Nogent.
\endverse

\endsong