% !TEX encoding = UTF-8 Unicode
\beginsong {La petite crotte de nez} [
ititle= {Petite crotte de nez (La)},
tu= {Le petit pont de bois (Y. Duteil)}]

\beginverse
La petite crotte de nez qui ne tenait plus guère
Que par un grand mystère et quelques bouts de glaires
La petite crotte de nez ainsi que les bouts d'glaires
Tombèrent un beau matin dans un soutien ouvert.
\endverse

\beginchorus
\textbf {Refrain}
Prout mou, prout mou, prout mou
\endchorus

\beginverse
C'est par le froid du glaire que les tétons pointèrent
Et que la crotte de nez fut d'nouveau propulsée
La glaire dans l'aventure, goûta la vergeture
La crotte fut en péril en tombant dans l'nombril
\endverse

\beginverse
C'est là qu'elle rencontra des crottes fossilisées
Qui dans cet orifice se sont embourgeoisées.
Par crainte des macchabées, elle s'est vite envolée
Par l'odeur alléchée d'un fromage avancé.
\endverse

\beginverse
Quand elle eut atterri dans un clito durci
C'est là qu'elle a glissé dans un con vérolé
Avec de l'urine, elle s'est désaltérée
Et voulant se nourrir, elle va dans le sphincter
\endverse

\beginverse
A l?entrée de l'anus, se cabrait un phallus
Qui d'sa vitalité la renvoya dans l'nez.
La morale de l'histoire, c'est que les crottes de nez
Doivent être bien accrochées pour(re) ne pas tomber.
\endverse

\endsong
