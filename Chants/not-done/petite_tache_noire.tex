% !TEX encoding = UTF-8 Unicode
\beginsong {La petite tache noire\footnotemark} [
ititle= {Petite tache noire (La)}]

\footnotemark {C'est Alfred Haring (mort en 1905), chef de musique au 32e régiment d'infanterie, qui publia la version originale de La petite tache noire ; elle consiste en des refrains de route arrangés en pas-redoublé avec clairons et tambours.}

\beginverse
Un jour la p'tit' Jeannette
Se baignant le cul nu
Aperçut dans un' glace
Son petit chat velu, ohu !
\endverse

\beginchorus
\textbf {Refrain}
\bisdeux {Ah ! Petite tache noire} {Jamais je ne t'avais vue.}
\endchorus

\beginverse
Ah ! Ah ! s'écria-t-elle,
Il est noir et poilu,
Et elle a décidé
Qu'il serait ras tondu, ohu !
\endverse

\beginverse
Avec de grands ciseaux
Tout de frais rémoulus,
Mais en voulant le tondre
Elle se l'est fendu, ohu !
\endverse

\beginverse
Tous les méd'cins d' la ville
Sont bien vite accourus,
Et dirent tous en choeur :
" Encor' un cul d' foutu ! " Ohu !
\endverse

\beginchorus
\textbf {Deuxième refrain}
\bisdeux {Ah ! Petite tache noire.} {Je ne te reverrai plus !}
\endchorus

\beginverse
Oui, mais le cousin Blaise
Lui aussi est venu,
Et sans perdre un' minute
Il lui a recousu, ohu !
\endverse

\beginchorus
\texbf {Troisième refrain}
\bisdeux {Ah ! Petite tache noire} {Moi, je te l'ai recousu.}
\endchorus

\beginverse
Avec sa grande aiguille
Qui lui pendait au cul
Et les deux p'lot's de fil
Qui y sont suspendues, ohu !
\endverse

\beginchorus
\textbf {Dernier refrain}
\bisdeux {Ah ! Petite tache noire} {Jamais je ne t'avais vue.}
\endchorus

\beginverse
Il fallait un' morale
A cett' histoir' de cul :
Fillett's ne tondez plus
Les poils de votre cul ;
\endverse

\beginverse
\textbf {Parlé :} Oh, non !
Ou si vous les coupez
Car vous êtes têtues
N'oubliez pas l'aiguille
Qui seule a recousu, ohu !
\endverse

\endsong
