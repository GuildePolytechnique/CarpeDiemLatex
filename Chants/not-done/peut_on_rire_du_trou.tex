% !TEX encoding = UTF-8 Unicode
\beginsong {Peut-on rire du trou?\footnotemark} [
ititle= {Peut-on rire du trou?},
tu= {Les filles de bord de mer (Arno)}]

\footnotetext {Chanson présentée au Festival de la chanson estudiantine, mais qui ne passera pas les éliminatoires}

\beginverse
Je ne suis qu'un petit ver de terre
Qui fréquentait les taupinières
Dans un jardin de Charleroi
Je rencontrais deux étrangères
\endverse

\beginverse
Elles venaient d'être mises en bière
Ce n'était pas pour me déplaire
Leurs corps n'étaient pas encore froids
Que j'y pénétrais par derrière
\endverse

\beginchorus
En douceur, en douceur
En douceur et profondeur
C'était chouette les filles d'Sart la Buissière
(tiens, tiens, tiens)
C'était chouette par devant, par derrière
\endchorus

\beginverse
Je croyais être ver solitaire
Quand j'aperçus quelques confrères
Ils étaient la depuis des mois
A glander comme de vieux pervers
\endverse

\beginverse
Je ne fis donc pas de manière
Je laissai là mes congénères
Pour aller voir une plus bas
Un trou qui m'était tout aussi cher
\endverse

\beginchorus
Refrain x2
\endchorus

\beginverse
J'étais peinard dans le sphincter
Quand je sentis vibrer la terre
Le ciel s'ouvrit au d'ssus de moi
Je vis deux gueules de fonctionnaires
\endverse

\beginverse
De ministère en ministère
Ils se passaient les teenagers
Ils recherchaient je ne sais quoi
Ma vie devenait un enfer
\endverse

\beginverse
Je regrettais ma vie d'hier
Où je n'bouffais que des mémères
Un trop jeune cul c'n'est pas la joie
Ca vous conduira aux galères
\endverse

\endsong