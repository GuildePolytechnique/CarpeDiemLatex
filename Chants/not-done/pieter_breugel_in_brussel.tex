\beginsong{PIETER BRUEGEL IN BRUSSEL}\footnotemark [
ititle={Pieter Bruegel in Brussel}]

\footnotetext{Auteur : Wannes Van De Velde.}

\beginverse
Pieter Bruegel de Oude
Zou opstaan uit de dood
Om d' wereld te aanschouwen :
Was 't bloed er nog zo rood als karmijn?
Zou er nog oorlog zijn?
\endverse

\beginverse
Als eerst ging hem naar Brussel,
Naar zijnen atelier
En hij nam zijnen bussel
Penselen en wat houtskool mee
Naar zijn Brabantse stee.
\endverse

\beginverse
Hij was nog niet vergeten
Waar dat zijn woonhuis was
Het was wel wat versleten
De memel woonde in zijn kas
Kapot was 't vensterglas.
\endverse

\beginverse
Eerst vroeg hem aan de mensen :
Is Spanje hier nog baas?
Leef' de naar eigen wensen?
Zijn ze nog even dwaas in ons land?
Of kregen ze verstand?
\endverse

\beginverse
De mensen wouden Bruegel
Zijn Brabants niet verstaan
Dus is hem stil en treurig
Naar een café gegaan die daar in
Zijn jeugd al had gestaan.
\endverse

\beginverse
Hij vroeg in 't zuiver Brabants
De kastelein om drank
Maar de patron die zei : "Pardon,
Je ne comprends pas l' flamand" - Emmerdant,
Dans le coeur du Brabant!-
\endverse

\beginverse
Pieter Bruegel den Ouwe
Die dacht 't is weer zover
Da'ze hier den Geuze nog brouwen
Da's fij maar dat 't in 't Frans nu moet zijn
Da vin'k een groot sjagrijn.
\endverse

\beginverse
Het Spaans is nu verdreven
Uit ons klein vaderland
Maar nu hebben we gekregen
Het Frans aan de Marollenkant
Da's boven mijn verstand.
\endverse

\beginverse
Piet' Bruegel is dan droevig
Terug naar zijn graf gegaan
Nadat hem op zijn kamer
Een heel klein maar een fijn schilderij
Vol kleur had doen ontstaan.
\endverse

\beginverse
En daarop stond geschilderd
Ne Vlaming in 't gevang
't Gevang van zijn kompleksen
De sleutel ligt erbij aan zijn zij
Doet open, maakt hem vrij!
\endverse
\endsong