% !TEX encoding = UTF-8 Unicode
\beginsong {Le pou et l'araignée\footnotemark} [
ititle = {Pou et l'araignée (Le)}]


\footnotetext {La "Confrérie des Dignitaires de l'Ordre de Saint Eloy" a repris cette chanson comme chant d'ordre.}

\beginverse
Un jour, un pou dans la rue,
Y rencontra chemin faisant, (chemin faisant)
Un' araignée bon enfant
Qui était toute velue.
Elle vendait du verr' pilé,
Pour s'ach'ter des p'tits souliers.
\endverse

\beginchorus
\textbf {Refrain}
Là, tu, là, tu m'emmerdes.
Là, tu, là, tu m' fais chier.
\bisdeux {Tu nous emmerdes, } {Tu nous fais chier.}
Et l'on entend dans les champs
S?masturber les éléphants,
Et l'on entend dans les prés
Gazouiller les chimpanzés.
Et l'on entend sous les ormeaux
Battr' la merd' à coups d' marteaux,
Et l'on entend dans les plumards
Battr' le foutr' à coups de braqu'marts,
\bis {Non, non, non, non, Saint Éloi n'est pas mort}
\bis {Car il band' encore.}
\endchorus

\beginverse
Le pou qui voulait la séduire
L'emm'na chez l' mastroquet du coin, (troquet du coin)
Lui fit boir' cinq, six coup d' vin
L'araignée ne fit qu'en rire.
La pauvrett' ne s' doutait pas
Qu'elle courait à son trépas.
\endverse

\beginverse
Le pou lui offrit une prise
En lui disant d'un air joyeux, (d'un air joyeux)
" Coll'-toi ça dans l' trou des yeux
Et mouch's-toi avec ta ch'mise. "
L'araignée qu' en avait pas.
Lui fit voir tous ses appas.
\endverse

\beginverse

Le pou, une franche canaille
Lui proposa trois francs six sous, (trois francs six sous)
" Ah ! qu'elle dit, c'est pas l' Pérou
Ce n'est qu'un fétu de paille.
Si tu m' donn's quatr' sous de plus
J' te f'rai voir le trou d' mon cul ! "
\endverse

\beginverse
C'est là que les horreurs commencent
Le pou monta sur l'araignée, (sur l'araignée)
Il n' pouvait plus s' retirer
Tant il eut de jouissance.
Si bien qu' la pauvr' araignée
Écope la maternité.
\endverse

\beginverse
Le pèr' d' l'araignée en colère
Lui dit : " Tu m'as déshonoré, (déshonoré)
Tu t'es laissée enceinter,
T' es aussi putain qu' ta mère ! "
L'araignée de désespoir
S'est foutu trois coups d' rasoir.
\endverse

\beginverse
Le pou, le désespoir dans l'âme,
S'arracha des poignées d' cheveux ; (poignées d' cheveux)
" Ah !, qu'il dit, y'a plus d' bon Dieu ! "
Il monta sur Notre-Dame,
Et c'est là qu'il s'a foutu
Les quatr' doigts et l' pouc' dans l' cul.
\endverse

\beginverse
Alors, les poux du voisinage,
Se réunir'nt pour l'enterrer, (pour l'enterrer)
Au cim'tière d' Levallois-Perret
Tout comm' un grand personnage
Et c'était bien trist' à voir
Tous ces poux en habit noir !
\endverse

\endsong