% !TEX encoding = UTF-8 Unicode
\beginsong {La puce\footnotemark} [
ititle= {Puce (La)},
tu= { Amphigouri (M. : Exaudet)}] 

\footnote {P. : Voltaire. On appelait "amphigouri" une chanson ou un poème qui n'avait aucun sens, l'auteur se laissait diriger par les besoins de la rime. Voltaire, sacrifiant à la mode, composa plusieurs chansons amphigouriques. Presque tous les amphigouris se chantent sur l'air du Menuet d'Exaudet (Source : Chansons insolites - Guy Breton - Disques Omega 143.022 - 1978).}

\beginverse
Au dortoir, sur le soir, la soeur Luce
En chemis' et sans mouchoir
Cherchant du blanc au noir
A surprendre une puce ;
A tâtons, du téton
A la cuisse,
L'animal ne fait qu'un saut
Ensuite, un peu plus haut,
Se glisse.
\endverse

\beginverse
Dans la petit' ouverture,
Croyant sa retraite sûre,
De pincer sans danger
Il se flatte.
Luce, pour se soulager,
Y porte un doigt léger
Et gratte.
\endverse

\beginverse
En ce lieu, par ce jeu, tout s'humecte ;
A force de chatouiller,
Venant à se mouiller,
Elle noya l'insecte.
Mais enfin, ce lutin
Qui rend l'âme
Veux fair' un dernier effort ;
Luce, grattant plus fort,
\endverse

\endsong