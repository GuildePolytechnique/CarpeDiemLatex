% !TEX encoding = UTF-8 Unicode
\beginsong {La pucelle malicieuse\footnotemark} [
ititle = {Pucelle malicieuse (La)},
tu= {Ma Belle Gazelle}]

\footnotetext {XIXème Festival de la chanson estudiantine CP ULB 1993}

\beginverse
J'ai vu une p'tite fille, débarquer d'là bas,
D'un petit village dont on ne parle pas,
Elle était jolie, elle chantait en la
Pour sa voix superbe, il la remarqua.
\endverse

\beginchorus
\ter {Une belle pucelle}
Qui cachait son jeu
\ter {Une belle pucelle}
Une belle aux airs malicieux.
\endchorus

\beginverse
Elle était nouvelle, dans la capitale,
Dans sa bouche sensuelle, pas de chanson triviale
Voulant se la faire lors du FESTIVAL,
Il lui offrit des fleurs, c'étaient celles du mâle.
\endverse

\beginverse
Naïve et sensible, elle fût très touchée,
Notre heureux bonhomme en fût bien remercié,
Les mains de la belle glissèrent vers ses roustons
Car elle aimait les fleurs autant que les bonbons
\endverse

\beginverse
La belle pucelle qui l'eût, qui l'eût dit
Était nymphomane et l'attacha au lit.
Cette voix superbe avait un secret
C'est qu'avec du sperme, elle se gargarisait.
\endverse

\beginverse
Gisant sur sa couche, l'organe ramolli,
Il avait les couilles complètement démolies.
Aujourd'hui il chante d'une voix de castré
À qui veut l'entendre
\endverse

\beginchorus
\textbf {Refrain final}
\ter {Une nymphomane}
Au regard vicieux
\ter {Une nymphomane}
Une nympho m'a bouffé la queue.
\endchorus

\endsong