% !TEX encoding = UTF-8 Unicode
\beginsong {Les quatre-vingts chasseurs\footnotemark}[
ititle= {Quatre-vingts chasseurs},
ititle= {Marquise (La)}]

\footnotetext {Autre titre : La marquise. Dériverait des paroles d'un Noël de Henri Murger (1850) qui, lui-même,
s'inspirerait d'un poème de Victor Hugo, Chanson de pirate (1828) (in Les Orientales).}

\beginverse
À l'ouverture de la chasse
Dans un château riche en gibier, (riche en gibier)
Une marquis' sans héritiers
Invita des chasseurs en masse.
Bientôt l'on vit tous les chasseurs
Accourir sans mêm' qu'on leur dise
\endverse

\beginchorus
\textbf {Refrain}
Au rendez-vous de la marquise,
Nous étions quatre-vingts chasseurs
\bisdeux { \quater {Quatre-vingts } {Quatre-vingts chasseurs}
\endchorus

\beginverse
Encouragés par notre belle
Nous abattîmes plus d'un faisan, (plus d'un faisan)
Quand un sanglier menaçant
Vint à s'élancer dessus elle.
Malgré sa rage et sa fureur
Nous l'obligeâm's à lâcher prise
Et, pour défendre la marquise, ...
\endverse

\beginverse
" Pour célébrer cette victoire,
Dit la marquis', il faut rentrer ; (il faut rentrer)
Ce n'est pas tout de s'illustrer,
Il faut aussi manger et boire.
En avant les vins, les liqueurs ! "
Et la nappe était déjà mise.
À la table de la marquise, ...
\endverse

\beginverse
Quand on eut savouré l' champagne
Nous fûmes dispos à l'amour, (-spos à l'amour)
Chacun voulut, chacun son tour,
Embrasser l'aimable compagne.
Nous étions tous de belle humeur,
Et la belle était déjà grise
Et, dans le lit de la marquise, ...
\endverse

\beginverse
Après cette histoire mémorable
Notre marquis' neuf mois plus tard, (neuf mois plus tard)
Nous mit au monde un beau bâtard
Un homme aujourd'hui redoutable.
De ses jours ignorant l'auteur
Il demanda qu'on l'en instruise
" Tu es, lui dit notre marquise,
Le fils de quatre-vingts chasseurs ... "
\endverse

\endsong