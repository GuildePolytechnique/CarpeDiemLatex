% !TEX encoding = UTF-8 Unicode
\beginsong {Regrets Bruxellois\footnotemark} [
ititle= {Regrets Bruxellois},
tu = {Le temps des colonies (J. Revaux , M. Sardou , P. Delanoë)}]
\footnotetext {VIIème Festival de la chanson estudiantine CP ULB, 1981. (Cercle des étudiants bruxellois, ULB)}

\beginverse
Moi monsieur, j'ai bu de la Kriek
Elle était vraiment magnifique
Elle était encor' naturelle
Elle goûtait pas l'eau de Javel
Mais d'puis qu'on leur a mis l' métro
Les Bruxellois préfèrent l'eau
Heureus'ment qu' y'a les étudiants
Pour boir' les bonnes vieill's bièr's d'antan.
\endverse

\beginchorus
\textbf {Refrain}
On pense encore à toi, Bruxellois
Au temps où le Faro était roi
Y'a plus d' bistrot, plus de Lambic, plus de zwanze, malchance !
Mais des regrets, ça, on en a plein la panse
On pense encore à toi, Bruxellois
Au temps où le Faro était roi.
\endchorus

\beginverse
Moi monsieur, j'ai connu l' Bruxelles
Où pour se fair' une pucelle
Tous les students faisaient la file
Dans cett' bonn' vieill' rue du Persil.
Au Diable Au Corps, les étudiants
Buvaient abominablement
À moins qu' ce n' soit au Nez-qui-Pend.
Où sont tous ces bistrots d'antan ?
\endverse

\beginverse
Messieurs ne vous souv'nez-vous pas
Des très héroïques combats
Des étudiants d' la rue des Sols
Contre les Apach's des Marolles ?
En ces temps presque z-oubliés
Y'avait encore de l'amitié
Car à cett' époque l'ULB
Signifiait la Fraternité.
\endverse

\endsong