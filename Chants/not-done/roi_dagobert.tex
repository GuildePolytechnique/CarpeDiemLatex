% !TEX encoding = UTF-8 Unicode
\beginsong {Le roi Dagobert\footnotemark} [
ititle= {Roi Dagobert (Le)}]

\footnotetext {Cette version est tirée en partie du disque de Colette Renard "Chansons "très" libertines" (Disques Vogue, LD 61630) et de "Les Fleurs du Mâle" (1983) ; elle est chantée par la Chorale de l'ULB (1935).}

\beginverse
Le bon Roi Dagobert
Enfilait sa femme à l'envers.
Le grand Saint Éloi
Lui dit : " Oh ! Mon Roi,
Vous êtes entré
Du mauvais côté. "
" Crétin, lui dit le Roi,
Tu sais bien qu' l'envers vaut l'endroit. "
\endverse

\beginverse
Le bon Roi Dagobert
Avait toujours sa queue à l'air.
Le grand Saint Éloi
Lui dit : " Oh ! Mon Roi,
Au mois de décembre
Faut rentrer son membre. "
Le Roi lui dit, très fier :
" Rien ne vaut le vit au grand air. "
\endverse

\beginverse
Le bon Roi Dagobert
Bandait toujours comm' un grand cerf.
Le grand Saint Éloi
Lui dit: " Oh ! Mon Roi,
On voit votre gland,
C' n'est pas élégant. "
Le Roi dit aussitôt :
" J'y vais accrocher mon chapeau. "
\endverse

\beginverse
Le bon Roi Dagobert
Se faisait sucer au dessert.
La Reine choquée
Lui dit : " C'est assez,
Devant tout l' palais
C'est vraiment très laid. "
Le Roi lui dit : " Souv'raine,
On n' doit pas parler la bouch' pleine. "
\endverse

\beginverse
La Reine Dagobert
Choyait un galant assez vert.
Le grand Saint Éloi
Lui dit : " Oh ! Mon Roi,
Vous êtes cocu,
J'en suis convaincu. "
" C'est vrai, lui dit le Roi,
Mon pèr' l'était bien avant moi. "
\endverse

\endsong