% !TEX encoding = UTF-8 Unicode
\beginsong {Le roi de Bavière\footnotemark} [
ititle= {Roi de Bavière (Le)}]

\footnotetext {Allusion serait faite à Louis II de Wittelsbach.}

\beginverse
Il était naguère
Un Roi de Bavière,
Toujours suivi d'un mortel ennui
Qui ne le quittait guère.
Un soir sous l'ombrage,
Seul avec son page,
Il entendit dans la forêt
Une voix qui chantait :
\endverse

\beginchorus
\textbf {Refrain}
" Moi je suis putain,
Sacré nom d'un chien !
Et pour un écu
Je fais voir mes fesses.
Moi je suis putain,
Sacré nom d'un chien !
Et pour un écu
Je fais voir mon cul. "
\endchorus

\beginverse
" Page, quell' est cette voix de fauvette ? "
" Sir', c'est Agnès qui se branle seulette
Et qui s'en va chantant
Ce refrain si charmant : "
\endverse

\beginverse
" Gentille bergère,
Ta voix sut me plaire.
Viens dans mon palais avec moi,
Mes trésors sont à toi. "
" Sir', vos trésors ne me tentent guère,
Vous pouvez bien vous les foutr' au derrière ! "
Et le Roi l'épousa,
Et le soir il chanta :
\endverse

\beginchorus
\textbf {Dernier refrain}
" Ah ! petit' putain
Que tu baises bien !
Ton con chauss' mon vit
Comm' une châsse.
Ah ! petit' putain
Que tu baises bien !
Ton con chauss' mon vit
Comme un écrin. "
\endchorus

\endsong