% !TEX encoding = UTF-8 Unicode
\beginsong {La romance du 14 Juillet\footnotemark} [
ititle= {Romance du 14 Juillet (La)},
ititle = {Fête nationale (La)}]

\footnotetext {Autre titre : La fête nationale.}

\beginverse
Elle avait ses quinz' ans à peine
Quand elle sentit battr' son coeur
Un beau soir, près du mec Gégène
Marinette a cru au bonheur.
C'était l' jour d' la fêt' nationale
Quand la bomb' éclat' en l'air
Elle sentit comm' une lame
Qui lui pénétrait, dans la chair.
\endverse

\beginchorus
Par-devant, par-derrière,
Tristement comme toujours,
Sans chichis, sans manières,
Elle a connu l'amour.
Les oiseaux dans les branches
En les voyant s'aimer
Entonnèr'nt la romance
Du quatorze juillet.
\endchorus

\beginverse
Mais quand refleurit l'aubépine,
Au premier souffl' du printemps,
Fallait voir la pauvre gamine
Mettr' au monde un petit enfant.
Mais Gégène, qu' est l' mec à la coule
Lui dit : " Ton goss', moi, j' m'en fous !
Si tu savais comm' je m' les roule
A ta plac', moi, j' lui tordrais l' cou. "
\endverse

\beginchorus
Par-devant, par-derrière,
Tristement comm' toujours,
Faillait voir la pauvr' mère,
Avec son goss' d' huit jours,
En fermant les paupières
Elle lui tordit l' kiki
Et dans l' trou des water(e)
Elle jeta son petit.
\endchorus

\beginverse
Mis' au banc de la cour d'assises
Et de c'ui d' la société
Elle fut traitée de fill' soumise
A la veill' du quatorze juillet.
Elle entendait son petit gosse
Qui appelait sa maman
Tandis que le verdict atroce
La condamnait au bagn' pour vingt ans.
\endverse

\beginchorus
Par-devant, par-derrière,
Tristement comme toujours,
Elle est mort' la pauvr' mère
A Cayenne un beau jour
Mort' avec l'espérance
De revoir son bébé
Dans la fosse d'aisance
Où elle l'avait jeté.
\endchorus

\beginverse
Elle avait ses quinze ans à peine
Quand elle sentit batt' son coeur
Un beau soir près du mec Gégène
Marinette a cru au bonheur.
\endverse

\endsong