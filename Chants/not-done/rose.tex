% !TEX encoding = UTF-8 Unicode
\beginsong {La rose\footnotemark} [
ititle= {Rose (La)},
tu = {Résiste-moi, belle Aspasie.}]

\footnotetext {P. : Robespierre. Alors qu'il était encore avocat à Arras, le futur révolutionnaire entra à la Société Littéraire et Philosophique des Rosati (Rosati est l'anagramme du mot "Artois") dont le but était de chanter les femmes, les roses et le vin. La cérémonie d'intronisation avait ses rites. On lisait un compliment en vers au récipiendaire, puis on lui offrait une rose qu'il devait respirer trois fois avant de l'attacher à sa boutonnière. Après quoi, il vidait un verre de vin rosé et devait écrire trois couplets sur un thème donné. La Rose est la chanson que Maximilien de Robespierre composa à cette occasion (Source : Chansons insolites - Guy Breton - Disques Omega - 143.022 - 1978).}

\beginverse
Je vois l'épine avec la rose
Dans les bouquets que vous m'offrez
Et lorsque vous me célébrez
Vos vers découragent ma prose.
Tout ce qu'on m'a dit de charmant
Messieurs a droit de me confondre,
La rose est votre compliment
L'épine est la loi d'y répondre.
\endverse

\beginverse
Dans cette fête si jolie
Règne l'accord le plus parfait
On ne fait pas mieux un couplet
On n'a pas de fleur mieux choisie.
Moi seule j'accuse mes destins
De ne m'y voir pas à ma place
Car la rose est dans nos jardins
Ce que vos vers sont au Parnasse.
\endverse

\beginverse
A vos bontés lorsque j'y pense
Ma foi, je n'y vois pas d'excès,
Et le tableau de vos succès
Affaiblit ma reconnaissance.
Pour de semblables jardiniers
Le sacrifice est peu de chose,
Quand on est si riche en lauriers
On peut bien donner une rose.
\endverse

\endsong