\beginsong{RUITERSLIED}\footnotemark [
tu={Tübinger Burschenschaft (P. : G. Herwegh, 1835 - M. : J. W. Lyra, 1843)},
ititle={Ruiterslied}]

\footnotetext{P. : Emiel Vereecken (début XXème siècle). Cette chanson est essentiellement chantée par les étudiants de dernière année.}

\emph{Les verres sont remplis.}

\beginverse
De bange nacht is weeral om,
We rijden stil, we rijden stom,
We rijden ten verderve!
Hoe koud waait toch de morgenwind!
Waardin, nu nog een glas gezwind
\bis{Voor 't sterven.}
\endverse

\beginverse
Hoe staat het jonge gras nu groen,
Maar bloeien zal het morgen doen,
Mijn eigen bloed zal 't verven
De eerste slok, met 't zweerd in de hand,
Gedronken voor het vaderland
\bis{Voor 't sterven.}
\endverse

\emph{On boit un première gorgée.}

\beginverse
De tweede slok van d' edele wijn,
Zal voor de heilige vrijheid zijn,
Voor vrijheid, land en erve;
\endverse

\emph{On boit une seconde gorgée.}

\beginverse
De rest zij nog een huldeblijk,
De laatst' voor 't oud Romeinse rijk
\bis{Voor 't sterven.}
\endverse

\emph{On vide le verre.}

\beginverse
Voor 't liefken, maar mijn glas is uit,
De spere blinkt, de kogel fluit,
Draag aan mijn kind de scherven!
\endverse

\emph{On brise le verre par terre.}

\beginverse
Vooruit nu naar de laatste slag.
O ruiterslust in vroege dag
\bis{Te sterven.}
\endverse
\endsong