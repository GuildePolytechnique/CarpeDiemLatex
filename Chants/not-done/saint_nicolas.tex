% !TEX encoding = UTF-8 Unicode
\beginsong {Saint-Nicolas\footnotemark} [
ititle= {Saint-Nicolas},
tu= {France (interprétée par Michel Sardou)}]

\footnotetext {P. : Yves Dulieu.}

\beginverse
V'là dix ans que je suis en taule
Avant j'étais Saint-Nicolas
Avec ma barbe j'étais drôle
Tous les goss's venaient près de moi.
Des lettr's affluaient de partout
Je recevais plein de courrier
Pour des bonbons des scoubidous
Ou pour des camions de pompiers.
\endverse

\beginchorus
\textbf {Refrain}
Ça s'est passé un 6 décembre
On m'accuse d'avoir vi-olé
Une fill' de six ans aux yeux tendre
C'est elle qui m'avait provoqué.
\endchorus

\beginverse
Quand je repens' à tous ces gosses
Qui défilaient sur mes genoux
Et qui jouaient avec ma cross'
C'est ambigu, oui, je l'avoue.
Quand je repens' à tous ces gosses
Accompagnés de leurs parents
Oui, j'ai commis ce gest' atroc'
C'était elle ou bien sa maman.
\endverse

\beginverse
Quand je paradais dans les rues
Suivi du Père Fouettard
Promené dans une voitur'
Plus gross' que celle d'une star.
Un tas de mignonnes fillettes
Ensembles se jetaient sur moi
Pour recevoir une sucette
A six ans ell's aim'nt déjà ça
\endverse

\beginverse
Je traînais dans les galeries
Dans les halls des supermarchés
On prenait des photographies
Toutes les famill's défilaient.
C'est ce jour-là qu'une salope
M'a d'mandé un' pair' de patins
J' lui filé, elle est en cloqu'
C'est de sa faute moi j'y peux rien.
\endverse

\beginverse
Il paraît qu'elle était pucelle
La press' racont' n'importe quoi
Qu'elle avait du sang dans les selles
Que prouv' qu'il y en avait déjà
Et si c'est vrai que puis-je faire ?
C'est bien regrettable pour elle
Qu'elle demand' dans ses prières
Des spermicides au Père Noël.
\endverse

\endsong