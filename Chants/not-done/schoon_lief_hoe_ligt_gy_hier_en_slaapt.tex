\beginsong{SCHOON LIEF, HOE LIGT GY HIER EN SLAEPT}\footnotemark [
ititle={Schoon lief, hoe ligt gy hier en slaept}]

\footnotetext{Chanson brabançonne datant d'environ 1610. Certains cercles VUBistes entonnent les couplets impairs par les hommes et les pairs par les femmes, sauf le dernier qui est entonn� par tout le monde.}

\beginverse
"Schoon lief, hoe ligt gy hier en slaept
In uwen eersten droome?
Wilt opstaan en den mei ontfaen
Hy staet hier al zoo schoone."
\endverse

\beginverse
" 'k En zou voor geenen mei opstaen
Myn vensterken niet ontsluiten :
Plant uwen mei waer 't u gerei
Plant uwen mei daer buiten!"
\endverse

\beginverse
"Waer zou 'k hem planten of waer doen?
't Is al op heeren strate;
De winternacht is koud en lang,
Hy zou zyn bloeyene laten.
\endverse

\beginverse
"Schoon lief, laet hy zyn bloeyen staen,
Wy zullen hem begraven,
Op 't kerkhof by den eglantier,
Zyn graf zal roosjes dragen."
\endverse

\beginverse
"Schoon lief, en om die roozekens
Zal 't nachtegaelken springen,
En voor ons bei in elken mei
Zyn zoete liedekens zingen."
\endverse
\endsong