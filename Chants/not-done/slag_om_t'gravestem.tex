% !TEX encoding = UTF-8 Unicode
\beginsong {De slag om t' Gravesteen\footnotemark} [
ititle = {Slag om t'Gravesteem (De)},
tu= {Preud'homme}]

\footnotetext {Paroles : Eugene De Ridder}

\beginverse
Te Gent, de oude stede,
Daar lag het Gravensteen
Sinds eeuwen als vergeten,
Verlaten en alleen.
Tot plots studentenkeerlen,
Belust op leute en lach,
Met list de burcht verov'ren,
Zo zonder stoot of slag.
\endverse

\beginchorus
\textbf {Refrain}
Spuiters van Vlaanderen!
Gent brult van pret:
"'t Gravenkasteel door studenten bezet!"
Ze zitten er binnen! wie krijgt z'er uit.
Ze vrezen noch knuppel, noch water, noch spuit!
Belegeraars zo ge ten aanval wilt gaan.
Past op, past op, past op, past op!
Uilenspiegel, Uilenspiegel voert hen aan?
\endchorus

\beginverse
't Pandoerenheir, zeeghaftig,
Rolt ladders bij de muur,
En neemt met waterlansen,
De ruïne onder vuur.
Maar appels als granaten,
Ontploffen op de grond,
En 't slijmig schroot zaait pletsend,
Verwarring in het rond.
\endverse

\beginverse
't Studentengild, verbeten
Bedekt met stof en as,
Verschoot zijn laatste appel,
Zijn laatste zode gras.
\endverse

\beginverse
Toen was hun strijd gestreden...
Maar, door de eeuwen heen,
Zal Vlaadrens lach herdenken,
De slag om 't Gravensteen!
\endverse

\endsong