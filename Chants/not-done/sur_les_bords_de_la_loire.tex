% !TEX encoding = UTF-8 Unicode
\beginsong {Sur les bords de la Loire\footnotemark} [
ititle= {Sur les bords de la Loire}]

\footnotetext {Cette chanson a été reprise par la Guilde Turquoise en tant que leur chant de guilde.}

\beginverse
\bis {La belle se promène, au fond de son jardin,}
Au fond de son jardin, sur les bords de La Loi-oi-re,
Au fond du jardin, sur les bords du ruisseau.
\endverse

\beginchorus
\textbf {Refrain}
Tout auprès du vaisseau, charmant matelot...
\endchorus

\beginverse
\bis {Sur le grand fleuve passe un brick de marinier}
Un brick de marinier, sur les bords de La Loi-oi-re,
Un brick de marinier, sur les bords du ruisseau.
\endverse

\beginverse
Le plus jeune des mousses chantait une chanson,...
\endverse

\beginverse
" Je voudrais, dit la belle, savoir votre chanson,...
\endverse

\beginverse
" Montez dedans le brick(e) et je vous l'apprendrai, ...
\endverse

\beginverse
Quand elle fut sur le brick(e), elle se mit à pleurer, ...
\endverse

\beginverse
" Qu'avez-vous donc, la belle, qu'avez-vous à pleurer ?...
\endverse

\beginverse
" Je pleure mon puc'lage qu'un gabier m'a volé, ...
\endverse

\beginverse
" Ne pleurez pas, la belle, je vous le retrouv'rai, ...
\endverse

\beginverse.
" Ça n' se rend pas, dit-elle, comm' de l'argent prêté ...
\endverse

\beginverse
" Sans ça, toutes les filles trouv'raient à se marier,...\endverse

\endsong