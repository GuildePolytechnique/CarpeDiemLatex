\beginsong{'T ROS BEYAERT}\footnotemark [
ititle={Ros Beyaert, 't}]

\footnotetext{P. : Kl. Wijtsman - M. : Vieille mélodie populaire (XIIème siècle) de Dendermonde. Recueillie par Florimond Van Duyse en 1868.}

\beginverse
't Ros Beyaard doet syn ronde
In de stad van Dendermonde;
Die van Aalst die syn soo quaet
Als bij ons hier 't Ros Beyaard gaet.
\endverse

\beginchorus
\textbf{Refrain}
De vier Aymonskinderen jent met blancken sweirt in d' handt
Ziet ze ryden : 't syn de schoonste al van ons land!
\endchorus

\beginverse
't Ros Beyaerts ooghen vonck'len,
Syne breede manen kronck'len,
En hy wend hem fraey en vlugh,
Met vier broers op synen rugh.
\endverse

\beginverse
Hun harnas schild en lancen
Blincken by de sonneglansen.
En den Beyaert 't voisken geeft,
Daer het ros syne eer in leeft.
\endverse

\beginverse
't Ros Beyaert is verheven,
Heeft hem in het vuur begeven,
En het week op 't oorlogsveld,
Alles voor syn groot geweld.
\endverse
\endsong