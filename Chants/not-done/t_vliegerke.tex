\beginsong{'T VLIEGERKE}\footnotemark [
ititle={Vliegerke ('T)}]

\footnotetext{Chanson de rue gantoise.}

\beginverse
'k Ben nie al te zot van 't spel,
Maar 'k vange gere musschen;
Marblen, toppen kan ik wel,
Maar daarin ben ik ni fel!
'k Zie tegenwoordig overal,
En ook al in mijn strate,
Jongens schuppen op nen bal,
Maar 'k spele 't liefst van al :
\endverse

\beginchorus
\textbf{Refrain}
Mee mijne vlieger, mee zijne steert,
Hij gaat omhoge, 't is 't ziene weerd;
'k Geve maar klauwe, op mijn gemak,
'k Heb nog drij bollekes, in mijn zak!
\endchorus

\beginverse
Mietje van de koolmarsjant,
Een meisje uit mijn strate,
Keurde mijne cerf-volant,
En ze had er 't handje van.
Want zo rap alsof de wind
Was ze aan 't spelen mee mijn klauwen,
En ze zei : "'t Es 't spele weerd,
Want hij heeft ne goeie steert!"
\endverse

\emph{Dans le refrain, drij bollekes est remplacé par twee bollekes.}

\beginverse
Tsjeef liet zijne vlieger op
Van tsjoepe, tsjoepe, tsjoepe,
Maar hij stuikt op zijne kop,
En muile dat hij trok;
Zijne spankoorde was te kort
En mee zijn tsietse klauwe,
En daarbij was zijne steert
Geen sjieke toebak weerd.
\endverse

\emph{Dans le refrain, drij bollekes est remplacé par één bolleke.}

\beginverse
Overlaatst op 't Sint Denijsplein,
Mijne vlieger was aan 't zweven,
Er kwam een wijf, een groot venijn,
En ze zei : "Da mag nie zijn,
Hij hangt te veel in mijne weg!";
Ze begost eraan te sleuren,
En op één, twee, drij, pardaf,
De koorde schoot eraf!
\endverse

\beginchorus
\textbf{Dernier refrain}
Hij was gaan vliegen, al mee de wind,
'k Stond te schriemen, 'k was maar een kind,
Mijne bol klauwe, diene ging ne gang,
Da zal 'k onthouden, mijn leven lang!
\endchorus
\endsong