% !TEX encoding = UTF-8 Unicode
\beginsong {Tape ta pine} [
ititle= {Tape ta pine}]

\beginverse
En revenant de la fête,
De la fêt' de Charenton\footnote {Variantes : "De la fêt' de Besançon," & "De la fête de chez nous,".}
J'ai rencontré trois fillettes,
Tap' ta pine,
Qui se chatouillaient l' bouton.
Tap' ta pine contre mon con.
\endverse

\beginverse
J'ai rencontré trois fillettes
Qui se chatouillaient l' bouton.
Je demande à la plus belle
Tap' ta pine :
" Comment vous appelle-t-on ? "
Tap' ta pin' contre mon con.
\endverse

\beginverse
... " On m'appelle Gabrielle ...
Gabrielle, c'est mon nom " ...
\endverse

\beginverse
... Je la prends et je l'embrasse ...
Je la couche sur le gazon ...
\endverse

\beginverse
... Je relève sa jupette ...
Lui fais voir mon Jean-Luron ...
\endverse

\beginverse
... Jean-Luron fort en colère ...
Crach' au nez de Brabançon ...
\endverse

\beginverse
... Brabançon qu' est fou de rage ...
Avala mon Jean-Luron ...
\endverse

\beginverse
... Mes deux couill's rest'nt à la porte ...
À la porte, en facti-on ...
\endverse

\beginverse
... Un poil du cul leur demande : ...
" Que faites-vous là, couillons ? " ...
\endverse

\beginverse
... " Nous attendons notre maître ...
Qu' est entré chez Brabançon " ...
\endverse

\beginverse
... " Il est entré l'vant la tête ...
Il en sortira couillon " ...
\endverse

\endsong