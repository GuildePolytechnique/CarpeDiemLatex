% !TEX encoding = UTF-8 Unicode
\beginsong {Les trente brigands\footnotemark} [
ititle= {Trente brigands (Les)},
tu= {Le complainte de Mandrin}]

\footnotetext {Chanson traditionnelle de la fin du XVIIIème siècle.}

\beginverse
Ils étaient vingt ou trente brigands dans une bande
Chacun sous le préau voulait m' toucher, vous m'entendez,
Chacun sous le préau voulait m' toucher ... un mot.
\endverse

\beginverse
Un beau jour sur la lande, l'un deux se fit prétendre
Et un p'tit guill'ret vint pour me trousser, vous m'entendez,
Et un p'tit guill'ret vint pour me trousser ... un couplet.
\endverse

\beginverse
Comme j'étais dans ma chambre, un matin de septembre
Un autre vint tout à coup pour me sauter, vous m'entendez,
Un autre vint tout à coup pour me sauter ... au cou.
\endverse

\beginverse
Un soir dans une fête un autr' perdit la tête
Et jusqu'au lendemain voulut m' baiser, vous m'entendez,
Et jusqu'au lendemain voulut m' baiser ... les mains.
\endverse

\beginverse
Le vent soul'vait ma robe quand l'un d'eux d'un air noble
S'approcha min' de rien et m' caressa, vous m'entendez,
S'approcha min' de rien et m' caressa ... mon chien.
\endverse

\beginverse
Comme je filais la laine un autr' avec sans gêne
Sans quitter son chapeau vint me p'loter, vous m'entendez,
Sans quitter son chapeau vint me p'loter ... mon éch'veau.
\endverse

\beginverse
Celui qui sut me prendre c'est un garçon de Flandre
Un soir entre deux draps, ce qu'il me fit, vous m'entendez,
Un soir entre deux draps, je ne l' vous dirai pas.
\endverse

\endsong