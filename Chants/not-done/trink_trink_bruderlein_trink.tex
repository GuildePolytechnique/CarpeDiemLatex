\beginsong{TRINK, TRINK, BRÜDERLEIN, TRINK}\footnotemark [
ititle={Trink, trink, Brüderlein, trink}]

\footnotetext{Auteur : Wilhelm Lindemann.}

\beginverse
Das Trinken das soll man nicht lassen,
Das Trinken regiert doch die Welt,
Man soll auch den Menschen nicht hassen,
Der stets eine Lage bestellt.
Ob Bier oder Wein, ob Champagner,
Nur lasst uns beim Trinken nicht prahlen;
Es trank den Champagner schon mancher,
Und konn ihn nachher nicht bezahlen.
\endverse

\beginchorus
\textbf{Refrain}
Trink, trink, Brüderlein, trink!
Lass doch die Sorgen zu Haus!
Trink, trink, Brüderlein, trink!
Zieh doch die Stirn nicht zu Grauss.
\bisdeux{{Meide den Kummer und meide den Schmerz},
	{Dann ist das Leben ein Scherz.}}
\endchorus

\beginverse
Das Lieben, das Trinken, das Singen
Schafft Freude und fröhlichen Mut.
Den Frauen, den musst du eins bringen,
Sie sind doch so lieb und so gut.
Verlieb dich solange du jung bist,
Die Hauptsach', du bist noch nicht blau,
Denn wenn man beim schönsten Trunk ist,
Bekommt man sehr leicht eine Frau.
\endverse

\beginverse
Der Moses, der hat, gar nicht übel,
Ein elftes Gebot noch erdacht,
Das steht nicht in de Bibel
Und hat so viel Freude gemacht.
Man hatte es uns unterschlagen,
Weil Trinken und Saufen es preist;
Ich aber, ich darf es euch sagen,
Ja wisst ihr denn auch wie es heisst?
\endverse
\endsong