% !TEX encoding = UTF-8 Unicode
\beginsong {Les trois orfèvres\footnotemark} [
ititle= {Trois orfèvres (Les)},
ititle = {Orfèvres (Les)}]

\footnotetext {Autre titre : Les Orfèvres. Cette chanson figurait dans l' "Anthologie Hospitalière et Latinesque" 1913). Ne sont renseignés ici que les couplets actuellement chantés en guindaille à Bruxelles.}

\beginverse
Trois orfèvres, à la Saint-Éloi,
S'en allèr'nt dîner chez un autre orfèvre.
Trois orfèvres, à la Saint-Eloi,
S'en allèr'nt dîner chez un bon bourgeois.
Ils ont baisé toute la famille :
La mèr' aux nichons
Le pèr' au cul, la fill' au con..
\endverse

\beginchorus
\textbf {Refrain}
Relevez, belles, votre blanc jupon,
Qu'on vous voie le cul, qu'on vous voie les fesses,
Relevez, belles, votre blanc jupon,
Qu'on vous voie le cul, qu'on vous voie le con !
\endchorus

\beginverse
La servante, qui avait tout vu,
Leur dit : " Foutez-moi votre pine aux fesses ! "
La servante qui avait tout vu,
Leur dit : " Foutez-moi votre pine dans l' cul ! "
Ils l'ont baisée, tous trois, sur un' chaise,
La chais' a cassé,
Ils sont tombés sans débander.
\endverse

\beginverse
Les orfèvres, non contents de ça
Montèr'nt sur le toit pour baiser Minette ;
Les orfèvres non contents de ça,
Montèr'nt sur le toit, pour baiser le chat.
" Chat, petit chat, chat tu m'égratignes,
Petit polisson,
Tu m'égratignes les roustons ! "
\endverse

\beginverse
Les orfèvres chez un pâtissier,
Entrèr'nt pour manger quelques friandises ;
Les orfèvres, chez un pâtissier,
Par les p'tits mitrons se fir'nt enculer.
Puis, voyant leurs vits pleins de merde,
Ils ont bouffé ça,
En guis' d'éclairs au chocolat.
\endverse

\beginverse
Les orfèvres, au son du canon,
Se retrouveront tous à la frontière ;
Les orfèvres, au son du canon,
En guis' de boulets, lanc'ront des étrons,
Bandant tous ainsi que des carmes,
À grands coups de vit,
Repousseront les ennemis.
\endverse

\endsong