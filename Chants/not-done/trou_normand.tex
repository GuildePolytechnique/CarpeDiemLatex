% !TEX encoding = UTF-8 Unicode
\beginsong {Le trou normand \footnotemark} [
ititle= {Trou normand (Le)},
ititle = {A-fond liégeois},
ititle = {Petit conduit (Le)},
ititle = {Pour bien chanter l'amour}]

\footnotetext {Autres titres : A-fond liégeois , Le petit conduit , Pour bien chanter l'amour.}

\beginverse
Amis, il existe un moment
Où les femmes, les fill's, et les mères.
Amis, il existe un moment
Où les femm's ont besoin d'un amant
\endverse

\beginverse
Qui les chatouille
Jusqu'à c' qu'ell's mouillent,
Et qui les baise
Le cul sur un' chaise.
\endverse

\beginverse
Mes amis, pour bien chanter l'amour,
\ter {Il faut boire.}
Mes amis, pour bien chanter l'amour,
Il faut boire, la nuit et le jour.
\endverse

\beginverse
À la santé du petit conduit
Par où Margot fait pipi.
Margot fait pipi par son p'tit con-, con-,
Par son p'tit -duit, -duit, par son p'tit conduit.
À la santé du petit conduit
Par où Margot fait pipi.
Il est en face du trou,
Laï trou laï trou laï trou la laire.
Il est en face du trou,
Laï trou laï trou laï trou la la.
\endverse

\beginverse
Il est en haut du trou ...
Il est en bas du trou ...
Il est à gauche du trou ...
Il est à droite du trou ...
Il est très loin du trou ...
Il est tout près du trou ...
Il va passer par l' trou ...
\endverse

\beginverse
\textbf {Parlé} : Attention ! Verre aux lèvres ! Un instant de silence !
Une minute de recueillement ! Une seconde d'abnégation !
Un, deux, trois : À fond !
\endverse

\beginverse
Il est passé par le trou ...
Il descendra par le trou ...
Il sortira par le trou ...
\endverse

\endsong