% !TEX encoding = UTF-8 Unicode
\beginsong {Valete Studia\footnotemark} [
ititle= {Valete studia},
ititle= {A, a, a, valete studia}]

\footnotetext {Chant médiéval de Leuven probablement, et donc ultérieur à 1423 (date des premiers cours qu'on y ait fait). Pour plus d'information sur ce sujet, consulter le "Codex Studiosorum Latino-Gallicus" (1986) à la page 193. Autre titre : A, a, a, valete studia.}

\beginverse
A, a, a,
\bis {Valete studia !}
Studia relinquimus,
Patriam repetimus,
A, a, a,
\bis {Valete studia !}
\endverse

\beginverse
E, e, e,
\bis {Ite, miseriæ !}
Bacchus nunc est dominus,
Consolator optimus,
E, e, e,
\bis {Ite, miseriæ !}
\endverse

\beginverse
I, i, i,
\bis{Bibant philosophi !}
Studiosi parvuli
Etiam sunt bibuli, ...
\endverse

\beginverse
O, o, o,
\bis {Nil est in poculo !}
Repleatur denuo,
Nummi sunt in sacculo ! ...
\endverse

\beginverse
U, u, u,
\bis {Ingente spiritu,}
Celebremus epulas,
Cras habemus ferias, ...
\endverse

\endsong