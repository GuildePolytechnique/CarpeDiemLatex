% !TEX encoding = UTF-8 Unicode
\beginverse {Le vieux morpion\footnotemark} [
ititle= {Vieux morpions (Le)}]

\footnotetext {Les français chantent cette chanson sur l'air du Plaisir des dieux, même si la chanson a son propre air. Daterait de la fin du Second Empire (1852-1870); la princesse Eugénie épousa l'empereur Napoléon III en 1853 et devint impératrice en 1856.}

\beginverse
Sur les débris d'une motte princière
Que la vérole emportait en lambeaux,
Un vieux morpions, plusieurs fois centenaire,
A ses enfants, disait ses derniers mots :
" Suivez enfant, le chemin de ma vie
De tous les cons, soyez les conquérants.
\endverse

\beginchorus
\textbf {Refrain}
\bisdeux {Car Dieu rêva dans sa philosophie}{De réunir les petits et les grands.}
\endchorus

\beginverse
J'ai vu le jour sur le vit d'un sauvage
Qui, du soleil, se disait rejeton.
Je suis venu de ces lointains rivages
Sur le prépuc' de Christophe Colomb ;
Comme il donnait un monde à sa patrie,
Je la peuplai de nombreux habitants.
\endverse

\beginverse
Depuis bientôt près de trois cents années
J'ai fréquenté les plus hauts potentats
J'ai poursuivi les pines couronnées,
J'ai vu des cons engendrer des prélats,
Plus d'un Saint-Pèr' sur ses couilles bénies
Sentit grouiller mes arpions\footnote{2. Arpion (1827) : pop. Pied. (in Dictionnaire alphabétique et analogique de la langue française par PaulRobert 1979)} triomphants.
\endverse

\beginverse
De Louis XIV, j'ai sucé la pine
Et j'ai vécu dix mois sur son bâton.
Frédéric II avait la chaude-pisse
Marie-Thérèse avait un chancr' au con,
J'ai vu briller le soleil d'Italie
Au-dessus du trône des papes branlants.
\endverse

\beginverse
Depuis, j'ai eu des heures malheureuses
Bien peu de cons me fur'nt hospitaliers
Et le vagin d'une religieuse
Puait si fort que j'ai failli crever.
Dans les bidets, j'allais de compagnie
Avec les spermatos agonisants.
\endverse

\beginverse
A Austerlitz, à Friedland et à Rome,
Partout enfin où le poussa le sort
J'ai poursuivi la pine du Grand Homme
Mais il est mort et moi je vis encore
J'habit' le con d' la princesse Eugénie
Et je le lègue à vous, mes chers enfants. "
\endverse

\beginverse
J'ai vu baiser la reine d'Angleterre
Par les sous-offs de tout's ses garnisons,
J'ai buriné les couilles du Saint-Père
Quand tous les soirs, il allait au boxon.
Suivez, enfants, le chemin de ma vie,
De tous les cons, soyez les conquérants.
\endverse

\beginverse
Le vieux morpion voulait parler encore
Mais dans sa bouch', sa langue se glaça.
Un froid mortel envahit tout son être
Et lentement le morpion expira.
Du bout d'un poil qu'agitait l'agonie
Il se raidit et dit à ses enfants :
" Oui, Dieu rêva ... "
\endverse

\endsong