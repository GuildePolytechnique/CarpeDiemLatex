\beginsong{VIVA LA COMPAGNEIA}\footnotemark [
ititle={Viva la compagneia}]

\footnotetext{Traduction d'un chant populaire allemand du XVIIème siècle.}

\beginverse
Ik steek mijn pintje naar omlaag,
Viva la compagneia!
En druk het teder aan mijn maag,
Vive la compagneia!
\endverse

\beginchorus
\textbf{Refrain}
Viva la, viva la, vive la va
Viva la, viva la, hopsasa,
Vive la compagneia!
\endchorus

\beginverse
Mijn pintje voor het sterven moet, ...
Geef ik een kus tot afscheidsgroet, ...
\endverse

\beginverse
Ik hef mijn pint tot aan mijn mond, ...
En drink ze leeg tot op de grond, ...
\endverse

\emph{Les verres seront bus ad fundum après le refrain.}

\beginverse
Het pintje heeft zijn dienst gedaan, ...
En 't onderste moet boven staan, ...
\endverse

\beginverse
Ei schachtje, kom nog eens langs hier ...
En vul mijn pint met schuimend bier! ...
\endverse

\emph{Le Senior boit sa pinte et, après, il la fait circuler chacun en buvant un peu. Pendant ce temps, il dit : }

\beginverse
Nu gaat mijn pint van keel tot keel; ...
Dat ieder drinkt maar niet te veel! ...
\endverse

\emph{Et la corona continue :}

\beginverse
Nu gaat de pint van keel tot keel; ...
Dat ieder drinkt maar niet te veel! ...
\endverse

\emph{Ce couplet est répété jusqu'à ce que la pinte soit de retour chez le Senior.}
\endsong