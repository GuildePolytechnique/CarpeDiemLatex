\beginsong{VOOR OUTER EN HEERD}\footnotemark [
ititle= {Voor outer en heerd}[

\footnotetext{Auteurs : Jef Simons, Armand Preud'homme.}]

\beginverse
Geen roekeloze wagers :
Stil volk dat zich beraadt,
Aleer het zijn belagers
Manhaft te lijve gaat.
Zij wisten wat zij wilden,
Toen zij tot stout verweer
De piek of zeis optilden,
Of grepen naar 't geweer.
\endverse

\beginchorus
\textbf{Refrain}
\bisdeux{Voor vrijheid en recht, ongeknecht,}
{Onverveerd voor outer en heerd!}
\endchorus

\beginverse
Zij steunden op Oranje's :
De Nederlanden één!
En juichten toen Brittanni's
Beloofde vloot verscheen.
Kloekmoedig in de gouwen
Van Diets Zuid-Nederland,
Zijn alle sterk en trouwe,
Gesprongen in den band.
\endverse

\beginverse
Rollier, Corbeels, Van Gansen,
Bevochten onverveerd,
Met wisselende kansen,
Den vijand van hun heerd.
Zij kampten koen als leeuwen,
En, werden ze overmand,
Hun namen staan voor eeuwen,
In 't hart van 't volk gebrand.
\endverse
\endsong