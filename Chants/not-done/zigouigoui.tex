% !TEX encoding = UTF-8 Unicode
\beginsong {Le zigouigoui\footnotemark} [
ititle= {Zigouigoui (Le)},
ititle = {Zigwigwi (Le)}]

\beginverse
Elle naquit un jour de fête
Avec un retard d'un an.
Un garçon, une fillette ?
Se demandaient ses parents
Une fille assurément
Car elle avait le plus grand...
\endverse

\beginchorus
\textbf {Refrain}
Zigouigoui, zigouigoui
Qu'elle tenait de sa mère
Zigouigoui, zigouigoui
Qu'elle gardait pour son mari.
\endchorus

\beginverse
A douz' ans fallait voir comme
Elle s'occupait d' l'avenir
Embrasser un beau jeune homme
Était son plus cher désir
En attendant l' grand frisson
Elle trifouillait dans son...
\endverse

\beginverse
A seiz' ans fut la maîtresse
La maîtress' d'un artilleur
Et dans ses moments d'ivresse
Elle rêvait avec ardeur
Qu' l'artilleur et son canon
Pourraient bien entrer dans son...
\endverse

\beginverse
Elle fut heureuse en ménage
Car son mari l'adorait
Et quand le vent faisait rage
C'est elle qui le réchauffait
Car son mari, sans façon,
Mettait les deux pieds dans son...
\endverse

\beginverse
Elle mourut de son vieil âge
Estimée de tout l' pays
Et les gens du voisinage
Sur sa tomb' gravèr'nt ceci :
"Ici gît assurément
Cell' qui avait le plus grand..."
\endverse

\endsong